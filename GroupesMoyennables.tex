\documentclass[a4paper,12pt]{article}
\usepackage[utf8]{inputenc}
\usepackage[french]{babel}
\usepackage{amssymb}
\usepackage{amsmath}
\usepackage{bbm}
\usepackage{amsthm}
\usepackage{a4wide}
\usepackage{mathrsfs}
\usepackage{stmaryrd}
\usepackage{mathtools}
\usepackage{graphicx}
\usepackage{hyperref}
\usepackage{faktor}
\usepackage{enumerate}
\usepackage{xcolor}
\usepackage{tikz, tkz-tab, tikz-cd}
\usepackage[T1]{fontenc}

\newtheorem{theorem}{Théorème}[section]
\newtheorem{proposition}[theorem]{Proposition}
\newtheorem{definition}[theorem]{Définition}
\newtheorem{corollary}[theorem]{Corollaire}
\newtheorem{lemma}[theorem]{Lemme}
\newtheorem{remark}[theorem]{Remarque}

\renewcommand{\i}{\mathrm{i}}
\newcommand{\R}{\mathbb{R}}
\newcommand{\N}{\mathbb{N}}
\newcommand{\Q}{\mathbb{Q}}
\newcommand{\Z}{\mathbb{Z}}
\newcommand{\C}{\mathbb{C}}
\newcommand{\K}{\mathbb{K}}
\newcommand{\F}{\mathcal{F}}
\newcommand{\G}{\mathcal{G}}
\newcommand{\U}{\mathcal{U}}
\newcommand{\ev}{\mathrm{ev}}
\newcommand{\Bor}{\mathcal{B}}
\newcommand{\norm}[1]{\left\Vert #1\right\Vert}
\newcommand{\abs}[1]{\left\vert#1\right\vert}
\newcommand{\card}[1]{\abs{#1}}
\newcommand{\ket}[1]{\left\langle #1 \right\rangle}
\newcommand{\floor}[1]{\left\lfloor #1 \right\rfloor}
\newcommand{\halfilon}{{\frac\varepsilon2}}
\newcommand{\set}[1]{\left\{ #1 \right\}}
\newcommand{\indic}{\mathbbm{1}}
\newcommand{\integral}[4]{\int_{#1}^{#2} #3~\mathrm{d}#4}
\newcommand\fundef[3]{#1: \left\{\begin{array}{ccc}#2\\#3\end{array}\right.}
\newcommand\funlam[2]{\left\{\begin{array}{ccc}#1\\#2\end{array}\right.}
\newcommand{\tq}{\;\middle|\;}
\newcommand{\ssi}{si et seulement si }
\newcommand{\interior}[1]{\mathring{#1}}
\newcommand{\closure}[1]{\overline{#1}}
\newcommand{\transpose}[1]{\prescript{t}{}{#1}{}{}}
\newcommand{\inv}{^{-1}}
\newcommand{\compl}{^c}
\newcommand{\infi}{\bigwedge}
\newcommand{\supr}{\bigvee}
\newcommand{\comp}{\circ}
\newcommand{\nhds}{\mathcal{N}}
\renewcommand{\implies}{\Rightarrow}
\renewcommand{\iff}{\Leftrightarrow}
\newcommand{\blank}{{-}}
\newcommand{\invop}{\mathrm{inv}}
\newcommand{\divop}{\mathrm{div}}
\newcommand{\parts}{\mathfrak{P}}
\newcommand{\finparts}{\mathfrak{P}_{\mathrm{fin}}}
\newcommand{\TODO}[1]{{\color{red}TODO :} #1}

\DeclareMathOperator{\sgn}{sgn}
\DeclareMathOperator{\ess}{ess}
\DeclareMathOperator{\Id}{id}
\DeclareMathOperator{\Supp}{supp}
\DeclareMathOperator{\id}{id}
\DeclareMathOperator{\Mat}{Mat}
\DeclareMathOperator{\Vect}{Vect}
\DeclareMathOperator{\Ima}{Im}
\DeclareMathOperator{\solset}{Sol}
\DeclareMathOperator{\Sp}{Sp}

\begin{document}

\begin{titlepage}
\title{Groupes Moyennables}
\author{Anatole \textsc{Dedecker}}
\maketitle
\thispagestyle{empty}
\end{titlepage}

\tableofcontents
\thispagestyle{empty}

\clearpage

\pagenumbering{arabic}

\section*{Conventions et remarques préliminaires}

\TODO{Réorganiser cette section}

Conformément à la convention dans le monde anglophone, et pour éviter toute confusion, nous dirons qu'un espace topologique
$X$ est \emph{compact} s'il vérifie la propriété de Borel-Lebesgue, sans hypothèse de séparation. 
Nous dirons que $X$ est \emph{localement compact} si, pour tout $x\in X$, le filtre $\nhds_x$ des voisinages de $x$ admet
une \textbf{base} formée d'ensembles compacts. Si $X$ est séparé, on retrouve que $X$ est localement compact 
\ssi tout point admet \textbf{un} voisinage compact.

\paragraph{}
Si $G$ est un groupe et $g\in G$, nous noterons $\invop$ l'inversion et $\ell_g$, $r_g$ les applications de tranlations:
\begin{equation*}
    \fundef{\invop = (\blank)\inv}{G&\to& G}{x&\mapsto& x\inv}\text{;}\quad\fundef{\ell_g=(g\blank)}{G&\to& G}{x&\mapsto& gx}\text{;}\quad\fundef{r_g=(\blank g)}{G&\to& G}{x&\mapsto& xg}
\end{equation*}

Si le groupe $G$ est muni d'une $\sigma$-algèbre stable par les tranlations droite et gauche, ce qui sera notamment le cas
si $G$ est un groupe topologique muni de sa tribu borélienne $\Bor(G)$, nous dirons qu'une mesure $m$ sur $G$ est \textit{invariante
(par translations) à gauche (resp. à droite)} si $\forall g\in G, {\ell_g}_*m = m$ (resp. ${r_g}_*m = m$). Si les deux conditions
sont vérifiées, nous parlerons simplement de mesure \textit{invariante par translations}.

Une \textit{mesure de Haar à gauche (resp. à droite)} sur un groupe topologique $G$ séparé et localement compact est une mesure de
Radon sur $G$ invariante par tranlations à gauche (resp. à droite). Rappelons le théorème fondamental concernant les mesures de Haar, que nous 
utiliserons à de nombreuses reprises.

\begin{theorem}\label{theorem_Haar}
    Tout groupe topologique $G$ séparé et localement compact admet une mesure de Haar à gauche, unique à multiplication par un réel strictement 
    positif près. 
    
    Si $\mu$ est une telle mesure, tout ouvert $U$ non-vide de $G$ est de mesure strictement positive,
    et $\mu(G)<+\infty$ \ssi G est compact.

    Enfin, si $G$ est compact, les mesures de Haar à gauche sont exactement les mesures de Haar à droite.
\end{theorem}

\TODO{Repréciser définition de mesure de Radon}
\TODO{Absolue continuité des mesures de Haar entre elles (via caractère modulaire)}
\TODO{Définir $\lambda$, $\rho$ les actions régulières de $G$ sur les espaces de fonctions}
\TODO{Notations $\mathscr{L}^p$, $\mathrm{L}^p$}

\section{Premières définitions}

Fixons $\Gamma$ un groupe topologique séparé et localement compact. Sauf mention explicite du contraire, $\mu$ désigne une mesure de Haar arbitraire
sur $\Gamma$.

La notion qui va nous intéresser dans ce mémoire est la suivante.

\begin{definition}\label{amenable_def}
    Le groupe séparé et localement compact $\Gamma$ est \emph{moyennable} s'il existe une forme linéaire positive $m : \mathrm{L}^\infty(G)\to\C$
    vérifiant $m(1) = 1$ et $\forall g\in\Gamma, m\comp\lambda_g = m = m\comp\rho_g$.
\end{definition}

Plus généralement, nous appellerons \emph{moyenne} sur un espace mesuré $(X,\mathcal{A},\mu)$ toute forme 
linéaire positive $m : \mathrm{L}^\infty(X, \mu)\to\C$ vérifiant $m(1) = 1$. Si de plus $X$ est un groupe 
et si $\mathcal{A}$ est stable par translations à gauche (resp. à droite), nous dirons qu'une moyenne est \emph{invariante à gauche}
(resp. \emph{à droite}) si $\forall x\in X, m\comp\lambda(x) = m$ (resp. $m\comp\rho(x) = m$). Si ces deux conditions 
sont vérifiées, nous dirons simplement que $T$ est \emph{invariante par translations} ou simplement \emph{invariante}.

Un groupe topologique séparé et localement compact $\Gamma$ est donc moyennable \ssi $(\Gamma, \Bor(\Gamma), \mu)$ admet une 
moyenne invariante, pour toute mesure de Haar $\mu$ sur $\Gamma$, le choix n'ayant aucune importance car toutes les mesures 
de Haar sont absolument continues les unes par rapport aux autres, et donc définissent le même espace $\mathrm{L}^\infty(\Gamma)$.

Notons tout de suite que l'invariance bilatère n'impose pas de restriction suplémentaire. 
\begin{proposition}\label{bilateral_of_left}
    Supposons qu'il existe une moyenne $m$ sur $\Gamma$, invariante par translations \emph{à gauche}.
    Alors $\Gamma$ est un groupe moyennable.
\end{proposition} 

Commençons par prouver le lemme suivant, qui sera utile en lui-même.

\begin{lemma}\label{positive_iff_norm}
    Soit $(X, \mathcal{A}, \mu)$ un espace mesuré et $\varphi:\mathrm{L}^\infty(X,\mu)\to\C$ une forme linéaire. 
    On a l'équivalence :
    \begin{equation*}
        \norm{\varphi} = \varphi(1) \iff \forall a \ge 0, \varphi(a) \geq 0
    \end{equation*}
\end{lemma}

\begin{proof}
    Supposons d'abord $\norm{\varphi} = \varphi(1)$, et soit $a\ge 0$ de norme 1. On a donc, pour presque tout $x\in X$, $\set{a(x), 1-a(x)}\subseteq[0,1]$,
    d'où enfin $\norm{1-a}_{\mathrm{L}^\infty}\le 1$. Mais par ailleurs $\norm{\varphi} = \varphi(a) + \varphi(1-a) \le \varphi(a) + \abs{\varphi(1-a)} \le \varphi(a) + \norm{\varphi}\norm{1 - a}_{\mathrm{L}^\infty}$, et finalement
    $\varphi(a)\ge \norm{\varphi}(1 - \norm{1-a}_{\mathrm{L}^\infty}) \ge0$. 

    Supposons maintenant $\varphi$ positive. Notons déjà qu'on a bien sûr $\norm{\varphi}\ge\abs{\varphi(1)}=\varphi(1)$. Soit donc $a\in\mathrm{L}^\infty(X,\mu)$ quelconque,
    et notons que $-\norm{a}_{\mathrm{L}^\infty}\cdot1\le a\le\norm{a}_{\mathrm{L}^\infty}\cdot1$, de sorte que $-\norm{a}_{\mathrm{L}^\infty}\varphi(1)\le\varphi(a)\le\norm{a}_{\mathrm{L}^\infty}\varphi(1)$, ce qui conclut.
\end{proof}

\begin{proof}[Démonstration de la proposition \ref{bilateral_of_left}]
    Il s'agit donc de construire une \emph{autre} moyenne $n$ sur $\Gamma$ qui soit cette-fois invariante des deux côtés.
    Posons d'abord, pour $f\in\mathrm{L}^\infty(\Gamma)$ quelconque, $\fundef{\widehat{f}}{\Gamma&\to&\C}{g&\mapsto&m(f\comp r_g)}$. On a 
    $\forall g\in\Gamma, \abs{\widehat{f}(g)} \le \norm{f\comp r_g}_{\mathrm{L}^\infty} = \norm{f}_{\mathrm{L}^\infty}$ par le lemme \ref{positive_iff_norm}, 
    donc $\widehat{f}\in B(\Gamma)\subseteq\mathscr{L}^\infty(\Gamma)$.
    Posons alors $n(f) := m\left(\widehat{f}\comp\invop\right)$
    \footnote{Le fait que $\widehat{f}\comp\invop$ soit bornée presque partout provient de ce que $\widehat{f}$ est bornée \emph{partout}.}.

    Notons que, pour $f, f_1, f_2\in\mathrm{L}^\infty(\Gamma)$ et $g, x\in\Gamma$, on a :
    \begin{gather*}
        \widehat{1} = 1 \\
        \widehat{f_1 + f_2} = \widehat{f_1} + \widehat{f_2} \\
        \widehat{f\comp\ell_g}(x) = m(f\comp\ell_g\comp r_x) = m(f\comp r_x\comp\ell_g) = m(f\comp r_x) = \widehat{f}(x) \\
        \widehat{f\comp r_g}(x) = m(f\comp r_g\comp r_x) = m(f\comp r_{xg}) = \widehat{f}(xg) = \left(\widehat{f}\comp r_g\right)(x)
    \end{gather*}
    En précomposant par $\invop$ et en appliquant $m$ à ces relations, on obtient :
    \begin{gather*}
        n(1) = m(1\comp\invop) = 1 \\
        n(f_1 + f_2) = m\left(\widehat{f_1}\comp\invop + \widehat{f_2}\comp\invop\right) = n(f_1) + n(f_2) \\
        (n\comp\lambda(g))(f) = m\left(\widehat{f\comp\ell_{g\inv}}\comp\invop\right) = m\left(\widehat{f}\comp\invop\right) = n(f) \\
        (n\comp\rho(g))(f) = m\left(\widehat{f\comp r_g}\comp\invop\right) = m\left(\widehat{f}\comp r_g\comp\invop\right) = m\left(\widehat{f}\comp \invop\comp\ell_{g\inv}\right) = n(f)
    \end{gather*}
    Ce qui conlut.
\end{proof}

Les premiers exemples de groupes moyennables sont les groupes compacts séparés (et en particulier les groupes finis discrets).
En effet, si l'on note $\mu$ la mesure de Haar normalisée d'un tel groupe, l'intégration selon $\mu$ fournit une moyenne invariante. 

Au vu de cet exemple, il peut être tentant de supposer que toute moyenne est donnée par l'intégration pour une mesure 
de Radon. En fait, si c'était le cas, la notion de moyennabilité ne serait pas très intéressante :
en effet, si $\mu$ est une mesure de Radon sur $\Gamma$ telle que $f\mapsto\integral{}{}{f}{\mu}$ soit une moyenne bien définie et invariante,
alors $\mu$ est automatiquement une mesure de Haar de masse $1$, et le théorème \ref{theorem_Haar} assure qu'une telle mesure n'existe que si
$\Gamma$ est compact. Les groupes moyennables seraient donc exactement les groupes compacts !

Évidemment ce n'est pas le cas, et nous montrerons en temps voulu que la classe des groupes moyennables est bien plus grande que 
celle des groupes moyennables. Cependant, nous disposons bien d'un théorème de représentation des moyennes comme des
\og{}intégrales\fg{}, à condition d'affaiblir la condition de $\sigma$-additivité des mesures.

Un \emph{contenu} sur un espace mesurable $(X, \mathcal{A})$\footnote{On trouve dans la littérature des définitions de \emph{contenu} autorisant $\mathcal{A}$ à n'être qu'une algèbre d'ensembles,
mais nous supposerons toujours qu'il s'agit d'une $\sigma$-algèbre.} est une fonction $m:\mathcal{A}\to\closure{\R_+}$ vérifiant $m(\varnothing) = 0$
et $m(A_1\cup A_2) = m(A_1) + m(A_2)$ pour $A_1, A_2\in\mathcal{A}$ disjoints. Les mesures sur $(X, \mathcal{A})$
sont donc exactement les contenus $\sigma$-additifs. Notons tout de suite que les contenus ne donnent pas lieu à 
une théorie de l'intégration intéressante : en effet, en l'absence du théorème de convergence monotone, il n'est pas clair que l'intégrale 
des fonctions positives \footnote{On rappelle que l'intégrale d'une fonction positive $f$ est définie comme la borne supérieure des 
intégrales des fonctions simples positives inférieures à $f$} soit additive ! On a par contre le résultat suivant en se retreignant à
intégrer des fonctions bornées selon des contenus finis.

\begin{proposition}\label{content_integration_and_repr}
    Soit $(X, \mathcal{A}, \mu)$ un espace mesuré et $m$ un contenu fini\footnote{i.e. $m(X)<+\infty$} sur $(X, \mathcal{A})$ \emph{absolument continu par rapport
    à} $\mu$. Il existe alors une unique forme linéaire positive $I_m : \mathrm{L}^\infty(X, \mu)\to\C$ vérifiant $I_m(\indic_A) = m(A)$ pour tout
    $A\in\mathcal{A}$. 

    De plus, l'application $m\mapsto I_m$ est une bijection de l'ensemble des contenus finis absoluments continus par rapport à $\mu$
    sur l'ensemble des formes linéaires positives sur $\mathrm{L}^\infty(X, \mu)$.
\end{proposition}

\begin{proof}
    On considère le sous-espace vectoriel $\mathscr{S}(X)$ de $\mathscr{L}^\infty(X, \mu)$ formé des fonctions simples, c'est à dire des
    fonctions $f:X\to\C$ mesurables d'image finie, et $\mathrm{S}(X, \mu)$ son image dans $\mathrm{L}^\infty(X, \mu)$. Il est clair que
    $\mathrm{S}(X, \mu)$ est l'espace vectoriel engendré par les classes des indicatrices des éléments de $\mathcal{A}$. Autrement dit, l'application
    $\fundef{\Theta}{\C^{(\mathcal{A})}&\to&\mathrm{S}(X, \mu)}{\delta_A&\mapsto&\indic_A}$ est surjective.
    
    Montrons tout d'abord que $\mathrm{S}(X, \mu)$ est dense dans $\mathrm{L}^\infty(X, \mu)$. Soit donc $f\in\mathscr{L}^\infty(X, \mu)$, 
    que nous supposons d'abord positive, et construisons une suite $g:\N\to \mathscr{S}(X, m)$ de la manière suivante :
    \begin{gather*}
        g_0 := 0 \\
        g_{n+1} := g_n + \frac12\norm{f - g_n}\indic_{\set{x\tq f(x) - g_n(x) \ge \frac12\norm{f - g_n}}}
    \end{gather*}
    Il est clair que chaque $g_n$ est une fonction simple positive et inférieure à $f$, et que la suite $g$ est croissante.
    De plus, pour tout $n\in\N$, et pour presque tout $x\in X$, on est dans l'un des cas suivants : 
    \begin{gather*}
        f(x)-g_{n+1}(x) = f(x)-g_n(x) \le \frac12\norm{f - g_n} \\
        f(x)-g_{n+1}(x) = f(x)-g_n(x)-\frac12\norm{f-g_n} \le \frac12\norm{f-g_n}
    \end{gather*}
    Il vient $\norm{f - g_{n+1}}\le\frac12\norm{f - g_n}$, d'où $\norm{f - g_n}\xrightarrow[n\to+\infty]{} 0$ dans $\mathrm{L}^\infty(X, \mu)$.
    Dans le cas général, il suffit alors de décomposer $f$ en combinaison linéaire de fonctions positives et d'appliquer 
    le résultat à chacune de ces fonctions. 

    Pour $m$ contenu fini avec $m\ll\mu$, posons $\fundef{\widehat{I}_m}{\C^{(\mathcal{A})}&\to&\C}{\delta_A&\mapsto&m(A)}$.
    Soit $\alpha\in\ker\Theta$, i.e tel que $f := \sum_{A\in\mathcal{A}} \alpha_A\indic_A =_\mu 0$. Considérons alors l'ensemble 
    fini $\mathcal{S}:=\alpha\inv\left(\set{0}\compl\right)\subseteq\mathcal{A}$ des $A$ tels que $\alpha_A\ne 0$, et la fonction 
    $\epsilon:X\to 2^\mathcal{S}$\footnote{On identifie 2 à l'ensemble $\set{0, 1}$} dont la composante selon $A\in\mathcal{S}$
    est l'indicatrice de $A$. Notons que chaque $A\in\mathcal{S}$ est l'union disjointe des $\epsilon\inv(b)$ pour $b\in2^\mathcal{S}$, $b(A) = 1$.
    On a donc :
    \begin{align*}
        \widehat{I}_m(\alpha) 
            &= \sum_{A\in\mathcal{S}} \alpha_A m(A) \\
            &= \sum_{A\in\mathcal{S}} \sum_{\substack{b\in2^\mathcal{S} \\ b(A) = 1}} \alpha_A m(\epsilon\inv(b)) \\
            &= \sum_{b\in2^\mathcal{S}} m(\epsilon\inv(b)) \sum_{\substack{A\in\mathcal{S} \\ b(A) = 1}} \alpha_A \\
            &= \sum_{b\in2^\mathcal{S}} m(\epsilon\inv(b)) \sum_{A\in\mathcal{S}} \alpha_A b(A)
    \end{align*}
    Or, pour chaque $b\in2^\mathcal{S}$, la fonction $f$ est constante de valeur $\sum_{A\in\mathcal{S}} \alpha_A b(A)$ sur l'ensemble 
    $\epsilon\inv(b)$. Comme $f$ est $\mu$-presque nulle, cela implique qu'on a ou bien $\sum_{A\in\mathcal{S}} \alpha_A b(A)=0$ ou bien $\mu(\epsilon\inv(b)) = m(\epsilon\inv(b)) = 0$,
    ce qui assure finalement que $\widehat{I}_m(\alpha) = 0$.

    On a donc montré que $\ker\Theta\subseteq\ker\widehat{I}_m$. $\widehat{I}_m$ se factorise donc à travers la surjection $\Theta$ en
    $\widetilde{I}_m:\mathrm{S}(X, \mu)\to\C$ linéaire vérifiant $\forall A\in\mathcal{A}, \widetilde{I}_m(\indic_A) = m(A)$. De plus, pour 
    $f\in\mathscr{S}(X)$, on a :
    \begin{equation*}
        \abs{\widetilde{I}_m(f)} = \abs{\sum_{z\in f(X)} zm(f\inv(z))} \le \sum_{z\in f(X)} \norm{f}_{\mathrm{L}^\infty} m(f\inv(z)) = \norm{f}_{\mathrm{L}^\infty} m(X)
    \end{equation*}
    Comme par ailleurs $\widetilde{I}_m(1)=m(X)$, on a donc $\widetilde{I}_m$ continue de norme exactement $m(X)$.
    Par le théorème de prolongement des applications uniformément continues, elle se prolonge donc de manière unique en $I_m : \mathrm{L}^\infty(X, \mu)\to\C$
    linéaire continue de même norme $m(X) = I_m(1)$. Le lemme \ref{positive_iff_norm} assure alors que $I_m$ est positive, et convient donc. 
    L'unicité de $I_m$ est alors 
    immédiate, la formule $m(A) = I_m(\indic_A)$ imposant la valeur sur $\mathrm{S}(X, \mu)$, puis sur $\mathrm{L}^\infty(X, \mu)$ par densité et continuité des formes linéaires 
    positives.

    Reste à montrer la bijectivité de $m\mapsto I_m$. L'injectivité est immédiate, la formule $m(A) = I_m(\indic_A)$ déterminant
    entièrement $m$. Soit donc $T : \mathrm{L}^\infty(X, \mu)\to\C$ une forme linéaire positive quelconque, posons $\fundef{m}{\mathcal{A}&\to&\closure{\R_+}}{A&\mapsto&T(\indic_A)}$.
    Il est clair que $m$ est alors un contenu, et la définition assure que $m\ll\mu$ car l'indicatrice d'un borélien de $\mu$-mesure nulle 
    est nulle dans $\mathrm{L}^\infty(X,\mu)$. Comme $T$ est positive et vérifie $\forall A\in\mathcal{A}, m(A)=T(\indic_A)$, l'unicité 
    démontrée précédemment assure que $T = I_m$, ce qui conclut.
\end{proof}

\TODO{Commentaire sur lien avec construction de l'intégrale de Bochner ?}

\begin{remark}
    Dans le cas où $m$ est une mesure finie, l'unicité assure que la forme $I_m$ n'est autre que l'intégrale par rapport à $m$, ou plus précisément sa précomposition
    par l'application naturelle $\mathrm{L}^\infty(X, \mu)\to\mathrm{L}^\infty(X, m)\hookrightarrow\mathrm{L}^1(X, m)$.
\end{remark}

Il est clair que la bijection $I$ de la proposition \ref{content_integration_and_repr} fait correspondre les moyennes 
sur $(X, \mathcal{A})$ aux contenus de masse 1 et absoluments continus par rapport à $\mu$. On voudrait maintenant pouvoir exprimer 
la condition d'invariance par translations, et il nous faut pour cela aborder l'aspect fonctioriel de la construction de $I$.

Soient $(X, \mathcal{A})$, $(Y, \mathcal{B})$ deux espaces mesurables, $m$ un contenu sur $(X, \mathcal{A})$ et $\varphi : X\to Y$ mesurable.
On définit alors le \emph{contenu image} de $m$ par $\varphi$ par $\fundef{\varphi_* m}{\mathcal{B}&\to&\closure{\R_+}}{B&\mapsto&m(\varphi\inv(B))}$.
Il est clair qu'il s'agit encore d'un contenu. Si de plus $X$ et $Y$ sont munis de mesures $\mu$ et $\nu$, et si on a 
$m\ll\mu$ et $\varphi_*m\ll\nu$
\footnote{Autrement dit, la préimage par $\varphi$ de tout ensemble $\nu$-négligeable est $\mu$-négligeable. Cette condition assure que la précomposition 
par $\varphi$ est une opération bien définie de $\mathrm{L}^\infty(Y, \nu)$ vers $\mathrm{L}^\infty(X, \mu)$. Enfin, si $\varphi_*\mu\ll\nu$, cette condition découle de l'hypothèse $m\ll\mu$, 
puisque $\varphi_*m\ll \varphi_*\mu\ll\nu$}
, alors :
\begin{equation*}
    \forall B\in\mathcal{B}, I_{\varphi_*m}(\indic_B) = m(\varphi\inv(B)) = I_m(\indic_{\varphi\inv(B)}) = I_m(\indic_B\comp \varphi)
\end{equation*}
La forme linéaire $\funlam{\mathrm{L}^\infty(Y, \nu)&\to&\C}{f&\mapsto&I_m(f\comp\varphi)}$ étant par ailleurs positive, l'unicité dans la proposition
\ref{content_integration_and_repr} assure qu'elle est égale à $I_{\varphi_*m}$, de sorte que :
\begin{equation*}
    \forall f\in\mathrm{L}^\infty(Y, \nu), I_{\varphi_*m}(f) = I_m(f\comp\varphi)
\end{equation*}

En revenant au cas d'un groupe séparé localement compact $\Gamma$ muni d'une mesure de Haar $\mu$, 
cette dernière égalité montre que $I_m$ est une moyenne invariante à gauche (resp. à droite) \ssi 
$m$ est de masse $1$ et $\forall g\in\Gamma, {\ell_g}_*m = m$ (resp. ${r_g}_*m = m$). On dira qu'un tel 
contenu est \emph{invariant à gauche} (resp. \emph{à droite}) s'il vérifie cette dernière condition, et 
\emph{invariant} s'il est invariant à gauche et à droite. En prenant en compte la proposition \ref{bilateral_of_left},
on vient donc de montrer le résultat suivant.
\begin{proposition}
    Un groupe localement compact séparé $\Gamma$ est moyennable \ssi il admet un contenu de masse $1$ et invariant 
    (resp. invariant à gauche, resp. invariant à droite).
\end{proposition}

Utilisons tout de suite ce critère concret pour donner un premier exemple de groupe \emph{non}-moyennable.

\begin{theorem}\label{not_amenable_F2}
    Le groupe libre en deux générateurs $F_2$ (muni de la topologie discrète) n'est pas moyennable.
\end{theorem}

\begin{proof}
    Notons $a, b$ les deux générateurs. Pour $m$ mot réduit en $\set{a, b, a\inv, b\inv}$, notons 
    $S(m)$ l'ensemble des $g\in F_2$ dont l'écriture (unique) comme mot réduit
    commence par $m$. On a par exemple $ab\in S(a)$ mais $a a\inv b \notin S(a)$ car le mot réduit associé 
    à $a a\inv b$ est $b$.

    Remarquons que $a\inv S(a) = S(a\inv)\compl$. En effet, si $m$ est un mot réduit ne commençant pas par $a\inv$, 
    $am$ est un mot réduit commençant par $a$. Réciproquement si $am$ est un mot réduit alors $m$ est réduit et ne commence 
    pas par $a\inv$. \TODO{est-ce que c'est assez clair ?}

    On montre de même que $b\inv S(b) = S(b\inv)\compl$. Supposons alors qu'il existe un contenu $m$ sur 
    $F_2$ de masse 1 et invariant par translation. On a alors :
    \begin{gather*}
        m(S(a)) + m(S(a\inv)) = m(a\inv S(a)\ \amalg\ S(a\inv)) = 1 \\
        m(S(b)) + m(S(b\inv)) = m(b\inv S(b)\ \amalg\ S(b\inv)) = 1
    \end{gather*}
    Mais les ensembles $S(a)$, $S(a\inv)$, $S(b)$ et $S(b\inv)$ sont disjoints, donc :
    \begin{equation*}
        1 = m(F_2) \ge m(S(a)) + m(S(a\inv)) + m(S(b)) + m(S(b\inv)) = 2
    \end{equation*} 
    D'où contradiction.
\end{proof}

\TODO{Mentionner importance historique ?}

%\paragraph{}
%
%Comme constaté dès la preuve de la proposition \ref{bilateral_of_left}, il va sans surprise être très souvent utile d'étudier la quasi-mesure $m$ à travers la théorie de l'intégration.
%On dispose pour cela du résultat suivant.
%\begin{proposition}\label{repr}
%    Soit $X$ un espace mesurable. Pour toute quasi-mesure de probabilité $m$ sur $X$, on définit :
%    \begin{equation*}
%        \fundef{I_m}{B(X)&\to&\C}{f&\mapsto&\integral{}{}{f}{m}}
%    \end{equation*}
%    L'application $I:m\mapsto I_m$ est alors une bijection de l'ensemble $\mathrm{Proba}(X)$ des mesures de
%    probabilités sur $X$ sur l'ensemble des formes linéaires positives sur $B(X)$ de valeur $1$ en $\indic$, la 
%    réciproque étant donnée par $T\mapsto(A\mapsto T(\indic_A))$.
%\end{proposition}
%
%On appellera \emph{moyenne} toute forme linéaire positive sur $B(X)$ de valeur $1$ en $\indic$. On va donc montrer que 
%les moyennes sur $X$ sont en bijections avec les mesures de probabilité sur $X$.
%
%Commençons par prouver le lemme suivant, qui sera utile en lui-même.
%
%\begin{lemma}\label{positive_iff_norm}
%    Soit $X$ un espace mesurable et $\varphi:B(X)\to\C$ une forme linéaire. 
%    Si $\varphi(\indic) = 1$, on a l'équivalence :
%    \begin{equation*}
%        \norm{\varphi} = 1 \iff \forall a \ge 0, \varphi(a) \geq 0
%    \end{equation*}
%\end{lemma}
%
%\begin{proof}
%    Supposons d'abord $\norm{\varphi} = 1$, et soit $a\ge 0$ de norme 1. On a donc, pour $x\in X$, $a(x), 1-a(x)\in[0,1]$,
%    d'où enfin $\norm{\indic-a}\le 1$. Mais on a $1 = \varphi(a) + \varphi(\indic-a) \le \varphi(a) + \abs{\varphi(\indic-a)} \le \varphi(a) + \norm{\indic - a}$, d'où
%    $\varphi(a)\ge 1 - \norm{\indic-a}\ge0$. \\
%    Supposons maintenant $\varphi$ positif. Notons déjà que l'égalité $\varphi(\indic) = 1$ entraîne $\norm{\varphi}\ge1$. Soit donc $a\in B(X)$ quelconque,
%    et notons que $-\norm{a}\indic\le a\le\norm{a}\indic$, de sorte que $-\norm{a}\le\varphi(a)\le\norm{a}$, ce qui conclut.
%\end{proof}
%
%\begin{proof}[Démonstration du théorème \ref{repr}]
%    Toute fonction mesurable bornée étant intégrable par rapport à toute mesure de probabilité, il est clair que $I$ est bien définie. 
%    Les propriétés élémentaires de l'intégration de Lebesgue, qui restent valables dans le cas de mesures finiment-additives,
%    donnent immédiatement que chaque $I_m$ est bien moyenne, et qu'on a
%    $I_m(\indic_A) = m(A)$ pour tout $A\subseteq X$ mesurable.
%
%    Soit maintenant $T$ une moyenne sur $X$.
%    Il faut alors montrer que $m : A\mapsto T(\indic_A)$ est bien une mesure de probabilité sur $X$,
%    puis que $T = I_m$ pour cette mesure $m$. $m$ est bien à valeurs positive par positivité de $T$, et 
%    de plus $m(X) = T(\indic) = 1$ et $m(\varnothing) = T(0) = 0$. Soient enfin $A, B\subseteq X$ deux ensembles
%    mesurables disjoints, de sorte que $\indic_{A\cup B} = \indic_A + \indic_B$. On a alors bien 
%    $m(A\cup B) = m(A) + m(B)$ ce qui assure l'additivité finie. 
%
%    Pour finir, considérons le sous-espace vectoriel $S(X, m)$ de $B(X)$ des fonctions $m$-simples, c'est à dire des 
%    fonctions $f:X\to\C$ mesurables, d'image finie, et vérifiant $\forall c\in\C^*, m(f\inv(x)) < +\infty$ (cette dernière 
%    condition est automatiquement vérifiée dans notre cas d'une mesure de probabilité, mais nécessaire en général).
%    Montrons que $S(X, m)$ est dense dans $B(X)$. Soit donc $f\in B(X)$, que nous supposons d'abord positive, et construisons 
%    une suite $g:\N\to S(X, m)$ de la manière suivante :
%    \begin{gather*}
%        g_0 := 0 \\
%        g_{n+1} := g_n + \frac12\norm{f - g_n}\indic_{\set{x\tq f(x) - g_n(x) \ge \frac12\norm{f - g_n}}}
%    \end{gather*}
%    Il est clair que chaque $g_n$ est une fonction simple positive et inférieure à $f$, et que la suite $g$ est croissante.
%    De plus, pour tout $n\in\N$ et $x\in X$, on est dans l'un des cas suivants : 
%    \begin{gather*}
%        f(x)-g_{n+1}(x) = f(x)-g_n(x) \le \frac12\norm{f - g_n} \\
%        f(x)-g_{n+1}(x) = f(x)-g_n(x)-\frac12\norm{f-g_n} \le \frac12\norm{f-g_n}
%    \end{gather*}
%    Il vient $\norm{f - g_{n+1}}\le\frac12\norm{f - g_n}$, d'où $\norm{f - g_n}\xrightarrow[n\to+\infty]{} 0$.
%    Dans le cas général, il suffit alors de décomposer $f$ en combinaison linéaire de fonctions positives et d'appliquer 
%    le résultat à chacune de ces fonctions. Cela montre donc la densité souhaitée. Or les formes linéaires $T$ et $I_m$
%    coïncident sur les indicatrices, donc sur $S(X, m)$, et elles sont continues par le lemme \ref{positive_iff_norm}. 
%    Par densité, elles sont donc égales, ce qui termine la preuve.
%\end{proof}
%
%\TODO{Commentaire sur lien avec construction de l'intégrale de Bochner ?}
%
%Cela nous amène à donner la définition suivante.
%
%\begin{definition}
%    Soit $\Gamma$ un groupe localement compact et séparé. Une moyenne $T : B(\Gamma)\to\C$ sur $\Gamma$ est dite 
%    \emph{invariante à gauche} (resp. \emph{à droite}) si $\forall g\in\Gamma, T\comp\lambda_g = T$
%    (resp. $T\comp\rho_g = T$). Une moyenne invariante à droite et à gauche sera simplement dite \emph{invariante}.
%\end{definition}
%
%\begin{remark}
%    Les moyennes invariantes à gauche sur $\Gamma$ sont exactement les morphismes de la représentation $(B(\Gamma), \lambda)$ de 
%    $\Gamma$ vers la représentation triviale, avec la condition suppplémentaire que $T(\indic) = 1$.
%\end{remark}
%
%Remarquons que la bijection $I$ de la proposition \ref{repr} fait correspondre les mesures de probabilité 
%invariantes à gauche aux moyennes invariantes à gauche sur $\Gamma$. En effet, si $m$ est invariante à gauche, alors pour tous 
%$f\in B(\Gamma)$ et $g\in\Gamma$, on a 
%$I_m(\lambda_g(f)) = \integral{}{}{f\comp\ell_{g\inv}}{m} = \integral{}{}{f}{m} = I_m(f)$. 
%
%Réciproquement, si $T$ est une 
%moyenne invariante à gauche, alors la mesure $m$ associe vérifie, pour tous $A\in\Bor(\Gamma)$ et $g\in\Gamma$, 
%$m(g\inv A) = T(\indic_{g\inv A}) = T(\indic_A \comp\ell_g) = T(\lambda_{g\inv}(\indic_A)) = T(\indic_A) = m(A)$.
%En prenant en compte la proposition \ref{bilateral_of_left}, on vient donc de montrer le résultat suivant.
%\begin{proposition}
%    Un groupe localement compact séparé $\Gamma$ est moyennable \ssi il admet une moyenne invariante (resp. à gauche, resp. à droite).
%\end{proposition}
%
%Dans la suite, nous supposons toujours que le groupe topologique $\Gamma$ est séparé et localement compact.
%\paragraph{}

\paragraph{}
Nous allons maintenant vouloir appliquer les outils de l'analyse fonctionelle à l'étude des moyennes invariantes.
Pour cela, le lemme \ref{positive_iff_norm} va de nouveau s'avérer crucial.

En effet, la condition $\norm\varphi = \varphi(1)$ qui apparaît dans ce lemme a le bon goût d'être conservée lors du prolongement de l'application 
$\varphi$ par le théorème de Hahn-Banach, alors qu'il n'est pas clair du tout qu'on puisse prolonger une forme linéaire positive en préservant 
la positivité. On va donc pouvoir se restreindre à étudier les moyennes sur des sous-espaces plus concrets de $\mathrm{L}^\infty(\Gamma)$.

Plus précisément, toujours pour $\Gamma$ groupe localement compact et séparé, on introduit $L_0(\Gamma) := \sum_{g\in\Gamma} \Ima(\lambda(g) - \id)$ le sous-espace de 
$\mathrm{L}^\infty(\Gamma)$ engendré par les classes de fonctions de la forme $\lambda(g)(f) - f$ pour $f\in \mathrm{L}^\infty(\Gamma)$. 
Il est alors clair qu'une moyenne sur $\Gamma$ est invariante à gauche \ssi sa restriction à $L_0(\Gamma)$ est nulle. 

\begin{lemma}
    $1\notin L_0(\Gamma)$
\end{lemma}

\begin{proof}
    Supposons d'abord qu'on ait $1 =_\mu \lambda(\gamma)(f) - f$ pour certains $\gamma\in\Gamma$ et 
    $f\in \mathscr{L}^\infty(\Gamma)$. L'ensemble $S:=\set{x\tq f(\gamma\inv x)\ne 1+f(x)}\cup\set{x\tq\abs{f(x)}>\norm{f}_{\mathrm{L}^\infty}}$ est donc $\mu$-négligeable,
    et par conséquent $T := \bigcup_{n\in\N}\gamma^{-n}S$ l'est aussi. $T\compl$ est donc non-vide, choisissons alors $t\in T\compl$. On a alors 
    $\forall n\in\N, \norm{f}_{\mathrm{L}^\infty}\geq \abs{f(\gamma^{-n}t)} = \abs{f(t) + n}$, ce qui est impossible puisque $\abs{f(t) + n}\xrightarrow[n\to+\infty]{}+\infty$.
    %\TODO{Cet argument ne marche pas si l'égalité $\indic = \lambda(\gamma)(f) - f$ n'est 
    %vraie que presque partout !}

    Pour le cas général, on suppose cette fois $1 =_\mu \sum_{i\in I} (\lambda(\gamma_i)(f_i) - f_i)$ pour $I$ fini, $\gamma : I \to\Gamma$ et
    $f : I\to \mathscr{L}^\infty(\Gamma)$. On considère alors le groupe séparé et localement compact $\Gamma^I$ (dont $\gamma : i\mapsto \gamma_i$ est
    un élément), muni de la mesure de Haar produit $\nu:=\mu^{\otimes I}$, et la fonction : 
    \begin{equation*}
        \fundef{F}{\Gamma^I&\to&\C}{x&\mapsto&\frac{1}{\card{I}}\sum_{i\in I}f_i(x_i)}
    \end{equation*}
    $F$ est mesurable, et la majoration
    $\abs{F(x)}\le\frac1{\card{I}}\sum_i\norm{f_i}_{\mathrm{L}^\infty}$, valable pour presque tout $x\in\Gamma^I$
    \footnote{Notons $S_i:=\set{x\in\Gamma\tq \abs{f_i(x)}>\norm{f}_{\mathrm{L}^\infty}}$ et $T:=\set{x\in\Gamma^I\tq \abs{F(x)}>\frac1{\card{I}}\sum_i\norm{f_i}_{\mathrm{L}^\infty}}$. 
        On a clairement $T\subseteq\bigcup_{i\in I}\pi_i\inv(S_i)$, où $\pi_i:\Gamma^I\to\Gamma$ désigne la $i$-ème projection. \\
        Or $\mu^{\otimes I}(\pi_i\inv(S_i)) = \mu^{\otimes I}(\Gamma\times\dots\times S_i\times\dots\times\Gamma)=\mu(\Gamma)\times\dots\times\mu(S_i)\times\dots\times\mu(\Gamma)=0$ puisque $\mu(S_i)=0$.
        Donc $T$ est bien $\mu^{\otimes I}$-négligeable.},
    assure que $F\in \mathscr{L}^\infty(\Gamma^I)$. 
    Enfin, $1 =_{\nu} \lambda(\gamma)(F) - F$, ce qui nous ramène au cas déjà traité. 
\end{proof}

On peut alors considérer l'espace $L(\Gamma) := \C\cdot 1 \oplus L_0(\Gamma)$, qui est intéressant en ce qu'il permet de caractériser entièrement
les moyennes à gauche sur $\Gamma$, au sens du théorème suivant.

\begin{theorem}\label{left_mean_iff}
    Considérons l'application linéaire $\widetilde{T} : L(\Gamma)\to\C$ définie par $\widetilde{T}(1) = 1$
    et $\widetilde{T}_{|L_0(\Gamma)} = 0$. 
    
    Le groupe $\Gamma$ est moyennable \ssi l'application $\widetilde{T}$ est continue et de norme $1$.
    Si c'est le cas, les moyennes à gauche sur $\Gamma$ sont exactement les prolongements de $\widetilde{T}$ à 
    $\mathrm{L}^\infty(\Gamma)$ qui préservent la norme.
\end{theorem}

En remarquant que $\forall c\in\C, \forall f\in L_0(\Gamma), \widetilde{T}(c + f) = c$, et qu'on a toujours 
$\norm{\widetilde{T}}\ge1$ par $\widetilde{T}(\indic) = 1$, on obtient le critère de moyennabilité suivant:

\begin{corollary}\label{amenable_iff_L0}
    $\Gamma$ est moyennable \ssi $\forall c\in\C, \forall f\in L_0(\Gamma), \abs{c}\le\norm{c + f}_{\mathrm{L}^\infty}$.
\end{corollary}

\begin{proof}[Démonstration du théorème \ref{left_mean_iff}]
    Supposons d'abord $\Gamma$ moyennable, et soit $T$ une moyenne à gauche sur $\Gamma$.
    On sait déjà que $T$ prolonge $\widetilde{T}$, puisque $T$ est nulle sur $L_0(\Gamma)$ et 
    $T(1)=1$. On a donc $\norm{\widetilde{T}}\le\norm{T}=1$ par restriction, et en fait $\norm{\widetilde{T}} = 1$
    puique $\widetilde{T}(1)=1$. 

    Supposons maintenant $\norm{\widetilde{T}}=1$, et soit $T$ un prolongement linéaire continu de $\widetilde{T}$ de norme $1$.
    On a alors $T(1)=1=\norm{T}$ et $T_{|L_0(\Gamma)} = 0$, donc $T$ est une moyenne à gauche sur $\Gamma$. Le théorème de Hahn-Banach
    garantissant l'existence d'un tel prolongement, cela termine la preuve.
\end{proof}

Illustrons ce critère sur le cas du groupe libre $F_2$. Dans ce cas, on peut montrer que $\norm{\widetilde{T}}\ge3$.
Reprenons pour cela les notations de la preuve du théorème \ref{not_amenable_F2}, et posons $f := (\lambda(a\inv)-\id)(\indic_{S(a)})$ et 
$g:=(\lambda(b\inv) - \id)(\indic_{S(b)})$, qui sont deux éléments de $L_0(F_2)$. L'égalité $a\inv S(a) = S(a\inv)\compl$ donne que 
$f = \indic_{a\inv S(a)} - \indic_{S(a)} = \indic_{\set{1}\cup S(b)\cup S(b\inv)}$, et de même 
$g = \indic_{\set{1}\cup S(a)\cup S(a\inv)}$, de sorte que $f+g=\indic + \delta_1$. Mais alors 
$-\frac23(\indic+\delta_1)\in L_0(F_2)$, donc $\widetilde{T}(\indic - \frac23(\indic+\delta_1)) = 1$, 
et d'autre part $\norm{\indic - \frac23(\indic+\delta_1)}_{\mathrm{L}^\infty}=\frac13$. 
On a donc bien $\norm{\widetilde{T}}\geq3$.

\paragraph{}

Donnons maintenant, toujours à l'aide du critère \ref{amenable_iff_L0}, notre premier exemple de groupe moyennable non-compact.

\begin{theorem}\label{Z_amenable}
    $\Z$ est moyennable.
\end{theorem}

\begin{proof}
    Commençons par simplifier un peu notre description de $L_0(\Z)$. Un simple argument de somme
    télescopique montre que, pour $n>0$ et $u\in \mathrm{L}^\infty(\Z)=\ell^\infty(\Z)$:
    \begin{equation*}
        (\lambda(n)-\id)(u) = \left(\sum_{1\le i\le n} \lambda(i+1) - \lambda(i)\right)(u) = (\lambda(1)-\id)\left(\sum_{1\le i\le n} \lambda(i)(u)\right)
    \end{equation*}

    Comme de plus $\lambda(0)-\id = 0$ et $\lambda(-n)-\id = -(\lambda(n)-\id)\comp\lambda(-n)$,
    on en déduit que $\Ima(\lambda(n)-\id)\subseteq\Ima(\lambda(1)-\id)$ pour tout $n\in\Z$, et donc $L_0(\Z) = \Ima(\lambda(1)-\id)$. \\

    Soient maintenant $c\in\C$ et $v\in L_0(\Z)$. On peut donc écrire $v = \lambda(1)(u)-u$ pour un certain $u\in\ell^\infty(\Z)$. 
    On veut montrer $\abs{c}\le\norm{c + \lambda(1)(u) - u}_\infty =: M$. 

    Par définition, on a $\forall n\in\Z, -M\le c+u_{n-1}-u_n\le M$. En moyennant ces inégalités pour 
    $n\in\rrbracket N-k, N\rrbracket$, on obtient, encore par un argument de somme télescopique :
    \begin{equation}\label{Z_amenable_eq1}
        \forall N\in\Z, \forall k\in\N,\quad -M\le c-\frac{u_N - u_{N-k}}{k} \le M
    \end{equation}

    Mais $u$ est bornée, donc il existe une suite strictement croissante $\varphi:\N\to\N$ telle que $u\comp\varphi$
    soit convergente. En particulier $\frac{\abs{u_{\varphi(n+1)}-u_{\varphi(n)}}}{\varphi(n+1)-\varphi(n)}\le\abs{u_{\varphi(n+1)}-u_{\varphi(n)}}\xrightarrow[n\to+\infty]{}0$, 
    donc $c-\frac{u_{\varphi(n+1)}-u_{\varphi(n)}}{\varphi(n+1)-\varphi(n)}\xrightarrow[n\to+\infty]{}c$. En passant à la limite dans l'inégalité (\ref{Z_amenable_eq1}),
    on obtient donc $-M\le c\le M$, ce qui conclut.
\end{proof}

Présentée ainsi, la preuve précédente peut sembler très spécifique à $\Z$ et peu généralisable. Pourtant, l'idée clé 
est simplement de considérer une \og{}moyenne\fg{} d'inégalités de la forme $-M\le c + u(\gamma\inv x) - u(x)\le M$, 
ce qui peut tout à fait se généraliser à un groupe discret quelconque. 

Les obstacles sont donc la forme spécifique de $L_0(\Z)$ et le recours à une suite extraite pour forcer la convergence des $u_{n+1}-u_n$. Nous allons voir 
qu'il est possible de les contourner.

\begin{proof}[Deuxième démonstration du théorème \ref{Z_amenable}]
    Soient $c\in\C$ et $v\in L_0(\Z)$, que l'on peut bien sûr décomposer en $v = \sum_{i\in I} (\lambda(n_i)(f_i) - f_i)$ pour certains $I$ fini, $n : I \to\Z$ et
    $f : I\to\ell^\infty(\Z)$. Posons toujours $M := \norm{c + v}_\infty$.

    Par définition, on a $\forall k\in\Z$:
    \begin{equation*}
        -M \le c + \sum_i (f_i(k-n_i) - f_i(k)) \le M
    \end{equation*}
    En moyennant ces inégalités pour $k\in F_N := \llbracket-N,N\rrbracket$, on obient $\forall N\in\N$:
    \begin{equation}\label{Z_amenable_eq2}
        -M \le c + \sum_i \frac1{2N+1} \sum_{k=-N}^N (f_i(k-n_i) - f_i(k)) \le M
    \end{equation}

    Étudions donc, pour $i$ fixé, les moyennes de la forme $\frac1{2N+1} \sum_{k=-N}^N (f_i(k-n_i) - f_i(k))$. On a :
    \begin{align*}
        \abs{\sum_{k=-N}^N (f_i(k-n_i) - f_i(k))} &= \abs{\sum_{k\in F_N - n_i} f_i(k) - \sum_{k\in F_N}f_i(k)} \\
            &= \abs{\sum_{\substack{k\in F_N - n_i \\ k\notin F_N}} f_i(k) -
            \sum_{\substack{k\in F_N \\ k\notin F_N-n_i}} f_i(k)} \\
            &\le \sum_{k\in (F_N-n_i)\triangle F_N} \norm{f_i}_\infty \\
            &=\norm{f_i}_\infty \cdot \card{(F_N-n_i)\triangle F_N}
    \end{align*}
    Or, pour $N>n_i$, l'ensemble $(F_N-n_i)\triangle F_N = \llbracket -N-n_i, -N\llbracket\ \amalg\ \rrbracket N-n_i, N\rrbracket$ est de cardinal $2n_i$ constant.
    On a donc:
    \begin{gather*}
        \frac{\card{(F_N-n_i)\triangle F_N}}{2N+1} \xrightarrow[N\to+\infty]{} 0 \\
        \frac1{2N+1} \sum_{k=-N}^N (f_i(k-n_i) - f_i(k)) \xrightarrow[N\to+\infty]{} 0
    \end{gather*}

    En passant à la limite dans l'inégalité (\ref{Z_amenable_eq2}), on obtient donc $-M\le c\le M$, ce qui conclut.
\end{proof}

Notons que, dans cette deuxième démonstration, l'hypothèse $\Gamma=\Z$ n'a été utilisée que pour établir l'existence d'une suite $F:\N\to\finparts^*(\Z)$ 
    \footnote{On note $\parts(X)$ (resp. $\parts^*(X)$) l'ensemble des parties (resp. parties non-vides) de $X$, et $\finparts(X)$ (resp $\finparts^*(X)$) 
    l'ensemble des parties finies (resp. parties finies non-vides) de $X$.}
de parties finies non-vides de $\Z$,
telle que $\forall n\in\Z, \frac{\card{(F_N-n)\triangle F_N}}{\card{F_N}} \xrightarrow[N\to+\infty]{} 0$. 

Dans le cas général, la mesure de Haar $\mu$ va remplacer le cardinal, et on va donc s'intéresser aux réels $\frac{\mu(\gamma F \triangle F)}{\mu(F)}$ 
pour $F$ partie compacte de $\Gamma$ et $\gamma\in\Gamma$, et à leur comportement asymptotique
lorsque $F$ est grand. C'est l'objet de la partie suivante.

\section{Conditions de F\o{}lner et de Reiter}

Dans cette section, $\Gamma$ est un groupe topologique séparé et localement compact, et on fixe 
$\mu$ une mesure de Haar à gauche sur $\Gamma$.

On s'intéresse à l'\emph{application de F\o{}lner} $\mathfrak{f} : \mathfrak{K}_+(\Gamma) \to \mathcal{F}(\Gamma, \R_+)$, définie sur l'ensemble 
$\mathfrak{K}_+(\Gamma)$
\footnote{On note $\mathfrak{K}(X)$ (resp. $\mathfrak{K}^*(X)$, resp. $\mathfrak{K}_+(X)$) l'ensemble des parties compactes 
(resp. compactes non-vides, resp. compactes d'intérieur non-vide) d'un espace topologique $X$. }
des parties compactes de $\Gamma$ d'intérieur non vide par $\mathfrak{f}(K)(\gamma) = \frac{\mu(\gamma K \triangle K)}{\mu(K)}$.
C'est une application bien définie car tout élément de $\mathfrak{K}_+(\Gamma)$ est de mesure finie non-nulle,
et il est clair qu'elle ne dépend pas de la normalisation choisie pour la mesure de Haar.

Un filtre $\mathscr{F}$ sur $\mathfrak{K}_+(\Gamma)$ est dit \emph{faiblement de F\o{}lner} si le filtre $\mathfrak{f}_*\mathscr{F}$ image directe 
de $\mathscr{F}$ par l'application de F\o{}lner converge vers $0$ pour la topologie de la convergence simple sur $\mathcal{F}(\Gamma, \R_+)$. 
Si la convergence est uniforme sur les compacts, on parle de filtre \emph{fortement de F\o{}lner}.
Une suite $K:\N\to\mathfrak{K}_+(\Gamma)$ est \emph{faiblement (resp. fortement) de F\o{}lner} si le filtre $K_*\nhds_{+\infty}^\N$, image directe par $K$ du filtre des
voisinages de l'infini dans $\N$, est faiblement (resp. fortement) de F\o{}lner. Comme $\mathfrak{f}_*K_*\nhds_{+\infty}^\N = (\mathfrak{f}\comp K)_*\nhds_{+\infty}^\N$,
une suite $K:\N\to\mathfrak{K}_+(\Gamma)$ est donc faiblement (resp. fortement) de F\o{}lner \ssi la suite de fonctions
$n\mapsto\left(\gamma\mapsto\frac{\mu(\gamma K_n\triangle K_n)}{\mu(K_n)}\right)$ converge vers $0$ simplement 
(resp. uniformément sur tout compact).

Notons que si $\Gamma$ est discret, 
la topologie de la convergence compacte coïncide avec celle de la convergence simple, de sorte que tout filtre 
faiblement de F\o{}lner est automatiquement fortement de F\o{}lner.
De plus, les parties compactes d'intérieur non vide sont exactement les parties finies non-vides,
et la mesure de Haar s'identifie (à un scalaire près) à la mesure de comptage. Dans ce cas, un filtre $\mathscr{F}$
est donc de F\o{}lner si et seulement si, pour tout $\gamma\in\Gamma$, la fonction $F\mapsto \frac{\card{\gamma F\triangle F}}{\card{F}}$ 
converge vers $0$ selon le filtre $\mathscr{F}$. De même, une suite $F:\N\to\finparts^*(\Gamma)$
est de F\o{}lner \ssi pour tout $\gamma\in\Gamma$, la suite $n\mapsto\frac{\card{\gamma F_n\triangle F_n}}{\card{F_n}}$ converge 
vers $0$.

Avant d'aller plus loin, donnons un critère plus simple pour l'existence de filtres de F\o{}lner (faibles ou forts) sur $\Gamma$.

\begin{lemma}\label{Folner_filter_of_cond}
    $\Gamma$ admet un filtre faiblement de F\o{}lner \ssi il satisfait la 
    \emph{condition de F\o{}lner faible} :
    \begin{equation*}\label{weak_Folner_cond}
        \forall\varepsilon>0, \forall \gamma\in\Gamma, \exists K\in\mathfrak{K}_+(\Gamma), 
        \frac{\mu(\gamma K\triangle K)}{\mu(K)}<\varepsilon
    \end{equation*}
    Il admet de un filtre fortement de F\o{}lner \ssi il satisfait la \emph{condition de F\o{}lner forte} : 
    \begin{equation*}\label{strong_Folner_cond}
        \forall\varepsilon>0, \forall A\in\mathfrak{K}(\Gamma), \exists K\in\mathfrak{K}_+(\Gamma), \forall\gamma\in A, 
        \frac{\mu(\gamma K\triangle K)}{\mu(K)}<\varepsilon
    \end{equation*}

    \TODO{Dans quel cas a-t-on des suites ?}
    %Si de plus $\Gamma$ est dénombrable, cette condition est équivalent à l'existence d'une suite de F\o{}lner.
\end{lemma}

\begin{proof}
    \TODO{}
\end{proof}

%\TODO{Quelle est la bonne topologie à mettre sur $\mathcal{F}(\Gamma, \R_+)$ pour la définition de filtre de F\o{}lner : convergence simple ou compacte ?
%    En tout cas la convergence simple suffit à faire marcher le théorème suivant.}

Comme anoncé, on peut alors généraliser le théorème \ref{Z_amenable} à tout groupe muni d'un filtre faiblement de F\o{}lner,
en adaptant directement la deuxième preuve de ce théorème.

\begin{theorem}\label{amenable_of_Folner}
    Si $\Gamma$ admet un filtre faiblement de F\o{}lner, alors $\Gamma$ est moyennable.
\end{theorem}

\begin{proof}
    On utilise toujours le critère \ref{amenable_iff_L0}. Soient donc $c\in\C$ et $v\in L_0(\Gamma)$, 
    que l'on écrit encore sous la forme $v = \sum_{i\in I} (\lambda(\gamma_i)(f_i) - f_i)$ pour certains $I$ fini, $\gamma : I \to\Gamma$ et
    $f : I\to \mathscr{L}^\infty(\Gamma)$. Posons aussi $M := \norm{c + v}_{\mathrm{L}^\infty}$.

    Par définition, on a $\forall x\in\Gamma$:
    \begin{equation}\label{amenable_of_Folner_eq1}
        -M \le c + \sum_i (f_i(\gamma_i\inv x) - f_i(x)) \le M
    \end{equation}
    En moyennant ces inégalités sur un $K\in\mathfrak{K}_+(\Gamma)$ quelconque, on obtient:
    \begin{equation}\label{amenable_of_Folner_eq2}
        -M \le c + \sum_i \frac1{\mu(K)} \integral{K}{}{\left(f_i(\gamma_i\inv x) - f_i(x)\right)}{\mu(x)} \le M
    \end{equation}

    Or, pour $i$ fixé, l'invariance par translation de $\mu$ donne :
    \begin{align*}
        \abs{\integral{K}{}{(f_i(\gamma_i\inv x) - f_i(x))}{\mu(x)}} 
            &= \abs{\integral{\gamma_i\inv K}{}{f_i}{\mu} - \integral{K}{}{f_i}{\mu}} \\
            &= \abs{\integral{\gamma_i\inv K\setminus K}{}{f_i}{\mu} - \integral{K\setminus\gamma_i\inv K}{}{f_i}{\mu}} \\
            &\le \integral{\gamma_i\inv K\triangle K}{}{\norm{f_i}_{\mathrm{L}^\infty}}{\mu} \\
            &= \norm{f_i}_{\mathrm{L}^\infty} \cdot \mu(\gamma_i\inv K\triangle K)
    \end{align*}
    
    Soit finalement $\mathscr{F}$ un filtre faiblement de F\o{}lner pour $\Gamma$. L'estimation précédente assure alors que
    la fonction $K\mapsto\frac1{\mu(K)} \integral{K}{}{(f_i(\gamma_i\inv x) - f_i(x))}{\mu(x)}$ converge vers $0$
    selon $\mathscr{F}$, et ce pour chaque $i\in I$. Il suffit enfin de prendre la limite selon $\mathscr{F}$ des inégalités \ref{amenable_of_Folner_eq2}
    pour obtenir $-M\le c\le M$. 
\end{proof}

La condition de F\o{}lner est très intéressante pour exprimer la moyennabilité de groupes discrets en termes combinatoires. Elle permet ainsi 
de lier la moyennabilité à la \emph{croissance} d'un groupe finiment engendré. \TODO{Dire quelques mots de plus ?} 

Cependant, pour établir la théorie, il sera utile de travailler avec une condition un peu plus flexible. Pour voir cela, reprenons une dernière fois la preuve
précédente. Plutôt que de prendre une moyenne uniforme des inégalités \ref{amenable_of_Folner_eq1} sur un compact, observons ce qui se passe 
lorsqu'on considère une moyenne pondérée par une fonction $\varphi\in\mathscr{L}^1(\Gamma)$, positive et de masse $1$ (le cas déjà traité correspondant à $\varphi=\frac1{\mu(K)}\indic_K$).
On obtient alors:
\begin{equation*}\label{intuition_Reiter_eq1}
    -M \le c + \sum_i \integral{}{}{\varphi(x)\left(f_i(\gamma_i\inv x) - f_i(x)\right)}{\mu(x)} \le M
\end{equation*}
Pour $i$ fixé, on a alors :
\begin{align*}
    \abs{\integral{}{}{\varphi(x)(f_i(\gamma_i\inv x) - f_i(x))}{\mu(x)}} 
        &= \abs{\integral{}{}{\varphi(\gamma_i x)f_i(x)}{\mu(x)} - \integral{}{}{\varphi(x) f_i(x)}{\mu(x)}} \\
        &= \abs{\integral{}{}{f_i\cdot\left(\lambda(\gamma_i\inv)(\varphi)-\varphi\right)}{\mu}} \\
        &\le \norm{f_i\cdot\left(\lambda(\gamma_i\inv)(\varphi)-\varphi\right)}_{\mathrm{L}^1} \\
        &\le \norm{f_i}_{\mathrm{L}^\infty} \norm{\lambda(\gamma_i\inv)(\varphi)-\varphi}_{\mathrm{L}^1}
\end{align*}
Pour pouvoir conclure, il faudrait donc cette fois pouvoir faire tendre $\norm{\lambda(\gamma_i\inv)(\varphi)-\varphi}_{\mathrm{L}^1}$ vers $0$, ce qui motive les définitions suivantes.

\paragraph{}

Notons $\mathscr{L}^1_{1,+}(\Gamma)$ l'ensemble convexe des $f\in\mathscr{L}^1(\Gamma)$ positives et de masse $1$, et $\mathrm{L}^1_{1, +}(\Gamma)$ son image dans 
$\mathrm{L}^1(\Gamma)$. On s'intéresse à l'\emph{application de Reiter} $\mathfrak{r} : \mathrm{L}^1_{1, +}(\Gamma)\to\mathcal{F}(\Gamma, \R_+)$, définie par
par $\mathfrak{r}(\varphi)(\gamma) = \norm{\lambda(\gamma_i\inv)(\varphi)-\varphi}_{\mathrm{L}^1}$.

Un filtre $\mathscr{F}$ sur $\mathrm{L}^1_{1, +}(\Gamma)$ est dit \emph{faiblement de Reiter} (resp. \emph{fortement de Reiter}) si le filtre $\mathfrak{r}_*\mathscr{F}$ converge vers $0$ pour la 
topologie de la convergence simple (resp. uniforme sur les compacts) sur $\mathcal{F}(\Gamma, \R_+)$. 
Une suite $\varphi:\N\to\mathrm{L}^1_{1, +}(\Gamma)$ est \emph{faiblement de Reiter} (resp \emph{fortement de Reiter}) si le filtre $\varphi_*\nhds_{+\infty}^\N$ est faiblement (resp. fortement) de Reiter. Comme $\mathfrak{r}_*\varphi_*\nhds_{+\infty}^\N = (\mathfrak{r}\comp \varphi)_*\nhds_{+\infty}^\N$,
une suite $\varphi:\N\to\mathrm{L}^1_{1, +}(\Gamma)$ est donc faiblement (resp. fortement) de Reiter \ssi la suite de fonctions
$n\mapsto\left(\gamma\mapsto\norm{\lambda(\gamma_i\inv)(\varphi)-\varphi}_{\mathrm{L}^1}\right)$ converge vers $0$ simplement 
(resp. uniformément sur tout compact).

On a aussi un analogue du lemme \ref{Folner_filter_of_cond}, qui se prouve de manière similaire.

\begin{lemma}
    $\Gamma$ admet un filtre faiblement de Reiter \ssi il satisfait la 
    \emph{condition de Reiter faible} :
    \begin{equation*}\label{weak_Reiter_cond}
        \forall\varepsilon>0, \forall \gamma\in\Gamma, \exists \varphi\in\mathscr{L}^1_{1,+}(\Gamma), 
        \norm{\lambda(\gamma_i\inv)(\varphi)-\varphi}_{\mathrm{L}^1}<\varepsilon
    \end{equation*}
    Il admet de un filtre fortement de Reiter \ssi il satisfait la \emph{condition de Reiter forte} : 
    \begin{equation*}\label{strong_Reiter_cond}
        \forall\varepsilon>0, \forall A\in\mathfrak{K}(\Gamma), \varphi\in\mathscr{L}^1_{1,+}(\Gamma), \forall\gamma\in A, 
        \norm{\lambda(\gamma_i\inv)(\varphi)-\varphi}_{\mathrm{L}^1}<\varepsilon
    \end{equation*}

    \TODO{Dans quel cas a-t-on des suites ?}
    %Si de plus $\Gamma$ est dénombrable, cette condition est équivalent à l'existence d'une suite de F\o{}lner.
\end{lemma}

Comme prévu, tout filtre de F\o{}lner fort (resp. faible) induit un filtre de Reiter fort (resp. faible). En effet, pour $K\in\mathfrak{K}_+(\Gamma)$, l'application 
$\chi_K:=\frac1{\mu(K)}\indic_K$ appartient à $\mathscr{L}^1_{1,+}(\Gamma)$, et on a $\mathfrak{r}(\chi_K) = \mathfrak{f}(K)$.
Si $\mathscr{F}$ est un filtre de F\o{}lner fort (resp. faible), $\chi_*\mathscr{F}$ est donc un filtre de Reiter fort (resp. faible).

Avant d'en arriver au résultat crucial de cette partie, introduisons une dernière notion. Une moyenne $m$ sur $\Gamma$ est dite 
\emph{topologiquement invariante (à gauche)} si :
\begin{equation*}
    \forall f\in\mathrm{L}^\infty(\Gamma), \forall\varphi\in\mathrm{L}^1_{1, +}(\Gamma), m(\varphi*f) = m(f)
\end{equation*}

Le théorème suivant assure que toutes les notions définies dans cette partie sont en fait équivalentes à la moyennabilité ! À travers la condition de Reiter, il nous 
permettra de caractériser la moyennabilité par la théorie des représentations dans la partie suivante.

\begin{theorem}\label{amenable_TFAE}
    Soit $\Gamma$ un groupe topologique séparé et localement compact. Les assertions suivantes sont équivalentes :
    \begin{enumerate}[(i)]
        \item $\Gamma$ est moyennable \label{amenable_TFAE/amenable}
        \item Il existe une moyenne topologiquement invariante sur $\Gamma$ \label{amenable_TFAE/topological_mean}
        \item $\Gamma$ satisfait la condition de Reiter forte \label{amenable_TFAE/strong_Reiter}
        \item $\Gamma$ satisfait la condition de F\o{}lner forte \label{amenable_TFAE/strong_Folner}
        \item $\Gamma$ satisfait la condition de F\o{}lner faible \label{amenable_TFAE/weak_Folner}
        \item $\Gamma$ satisfait la condition de Reiter faible \label{amenable_TFAE/weak_Reiter}
    \end{enumerate}
\end{theorem}

La preuve nécessite un certain nombre de résultats et définitions préliminaires, que nous détaillons maintenant. 

\TODO{Dans les sections suivantes, motiver les résultats à l'avance. Ou bien les mettre dans la preuve et faire les preuves des lemmes ensuite.}

\subsection{L'espace \texorpdfstring{$UC_b(\Gamma_d)$}{des fonctions uniformément continues bornées}}

Nous notons $UC_b(\Gamma_d)$ l'ensemble des fonctions $f:\Gamma_d\to\C$ qui sont bornées et uniformément 
continues, muni de la norme uniforme. Dans cette définition et dans toute la suite, $\Gamma_d$ désigne l'espace uniforme obtenu 
en munissant le groupe topologique $\Gamma$ de sa structure uniforme \emph{droite}\footnote{On noterait de même $\Gamma_s$
le groupe topologique $\Gamma$ muni de sa structure uniforme \emph{gauche}.}.
Rappelons que cette structure uniforme, dont nous noterons $\mathcal{U}\Gamma_d$ le filtre des entourages,
est définie par l'égalité $\mathcal{U}\Gamma_d=(\divop_d)^*\nhds_1^\Gamma$, pour 
$\fundef{\divop_d}{\Gamma\times\Gamma&\to&\Gamma}{(x, y)&\mapsto&x y\inv}$\footnote{Dans le cas de la structure 
uniforme gauche, on remplaçerait $\divop_d$ par $\fundef{\divop_s}{\Gamma\times\Gamma&\to&\Gamma}{(x, y)&\mapsto&x\inv y}$.}.
Rappelons aussi que la structure uniforme associée à l'espace métrique $(X, d)$ est définie par 
$\mathcal{U}X = d^*\nhds_0$. 
Pour $f:\Gamma\to\C$ bornée, on a donc : 
\begin{align}
    f\in UC_b(\Gamma_d)
        &\iff \mathcal{U}\Gamma_d \le\footnotemark\ (f\times f)^*\mathcal{U}\C \nonumber \\
        &\iff (\divop_d)^*\nhds_1^\Gamma \le (f\times f)^*d^*\nhds_0 \nonumber \\
        &\iff \forall W\in\nhds_0, \exists V\in\nhds_1^\Gamma, \set{(x, y)\tq x y\inv\in V} \subseteq \set{(x, y)\tq \abs{f(x)-f(y)}\in W} \nonumber \\
        &\iff \forall \varepsilon>0, \exists V\in\nhds_1^\Gamma, \forall y\in\Gamma, \forall g\in V, \abs{f(gy)-f(y)} < \varepsilon \nonumber \\
        &\iff \forall \varepsilon>0, \exists V\in\nhds_1^\Gamma, \forall g\in V, \norm{\lambda(g)(f) - f}_\infty < \varepsilon \label{UCB_explicit_cara}
\end{align}
Où on a pu remplacer $\lambda(g\inv)$ par $\lambda(g)$ en remplaçant $V$ par son symétrique. Cela permet de donner la caractérisation suivante de $UC_b(\Gamma_d)$.
\footnotetext{Si $\mathscr{F}, \mathscr{G}$ sont deux filtres sur un ensemble $X$,
on note $\mathscr{F}\le \mathscr{G}$ si $\mathscr{F}$ est \emph{plus fin que} $\mathscr{G}$,
c'est à dire si $\mathscr{G}\subseteq \mathscr{F}$. } 

\begin{lemma}\label{UCB_iff_translate}
    Soit $f\in\ell^\infty(\Gamma)$\footnote{i.e $f$ est bornée \emph{partout}}. $f$ appartient à $UC_b(\Gamma_d)$ \ssi l'application 
    $\fundef{\ev_f\comp\lambda}{\Gamma&\to&\ell^\infty(\Gamma)}{g&\mapsto&\lambda(g)(f)}$
    est continue.
\end{lemma}

On donne une démonstration directe en se basant sur la caractérisation \ref{UCB_explicit_cara}, mais on 
pourrait aussi utiliser le théorème \ref{uniform_continuous_iff_postcomp}, plus général, démontré en annexe.

\begin{proof}
    Remarquons d'abord que $\ev_f\comp\lambda$ est continue \ssi elle est continue en $1$. Le sens direct est immédiat, supposons donc 
    la continuité en $1$, et montrons la continuité en $x\in\Gamma$ quelconque. Soit donc $\varepsilon>0$. La continuité en $1$ fournit 
    $V\in\nhds_1$ tel que $\forall g\in V, \norm{\lambda(g)(f) - f}_\infty<\varepsilon$. Notons alors que :
    \begin{equation*}
        \forall g\in V, \norm{\lambda(xg)(f) - \lambda(x)(f)}_\infty = \norm{\lambda(x)(\lambda(g)(f) - f)}_\infty = \norm{\lambda(g)(f) - f}_\infty < \varepsilon
    \end{equation*}
    Comme $xV\in\nhds_x$ cela conclut, puisqu'on a $\forall y\in xV, \norm{\lambda(y)(f) - \lambda(x)(f)}_\infty < \varepsilon$.

    Or, on remarque que la caractérisation \ref{UCB_explicit_cara} de $UC_b(\Gamma_d)$ exprime précisément la continuité en $1$ de l'application $\ev_f\comp\lambda$.
    Cela conclut la preuve du lemme.
\end{proof}

\subsection{Moyennes provenant de fonctions intégrables}

\TODO{Gérer absence de $\sigma$-finitude}

Rappelons que, sauf mention explicite du contraire, tous les espaces $\mathrm{L}^p$ sont complexes, et
si $E$ est un espace vectoriel topologique, $E^*$ désigne son dual topologique \emph{sur le corps $\C$}. Si
$E$ est un espace vectoriel complexe, nous noterons $\floor{E}$ l'espace vectoriel réel sous-jacent.

Notons $\mathcal{M}(X,\mu)$ l'ensemble des moyennes sur un espaces mesuré $(X,\mathcal{A},\mu)$, telles que définies après la définition \ref{amenable_def}.
Dans le cas d'un groupe localement compact et séparé $\Gamma$ muni d'une mesure de Haar $\mu$, nous noterons
$\mathcal{M}(\Gamma):=\mathcal{M}(\Gamma, \mu)$\footnote{Rappelons que cet ensemble ne dépend pas du choix de $\mu$.}.

Soit $X$ espace topologique séparé et localement compact muni d'une mesure de Radon $\mu$. Le plongement isométrique usuel $\mathrm{L}^1(X,\mu)\hookrightarrow\mathrm{L}^\infty(X,\mu)^*$
(\TODO{pk ?})
induit, par restriction, un plongement isométrique $\mathrm{L}^1_{1, +}(X, \mu)\hookrightarrow \mathcal{M}(X, \mu)$. 
En effet, il est clair que, pour $f\in\mathscr{L}^1_{1, +}(X, \mu)$, la forme linéaire $\funlam{\mathscr{L}^\infty(X,\mu)&\to&\C}{ \varphi &\mapsto&\integral{}{}{f\varphi}{\mu}}$ est une moyenne,
car elle est positive et $\integral{}{}{f\cdot 1}{\mu} = \integral{}{}{f}{\mu} = 1$.

Le résultat principal quant à ce plongement est que son image est $*$-faiblement dense.

\begin{lemma}
    Soit $X$ espace topologique séparé et localement compact muni d'une mesure de Radon $\mu$.

    Munissons l'espace $\mathrm{L}^1(X,\mu)$ de la topologie faible $\sigma(\mathrm{L}^1, \mathrm{L}^\infty) = \sigma\left(\mathrm{L}^1, {\mathrm{L}^1}^*\right)$
    \footnote{Cette égalité provient de ce que ${\mathrm{L}^1}^*\simeq\mathrm{L}^\infty$}, 
    l'espace $\mathrm{L}^\infty(X,\mu)^*$ de la topologie faible-$*$ $\sigma({\mathrm{L}^\infty}^*, \mathrm{L}^\infty)$,
    et les parties $\mathrm{L}^1_{1, +}(X, \mu)$, $\mathcal{M}(X, \mu)$ des topologies induites respectives.

    L'application $\iota:\mathrm{L}^1(X,\mu)\hookrightarrow\mathrm{L}^\infty(X,\mu)^*$ est alors un plongement d'espaces 
    vectoriels topologiques, et le plongement topologique induit $\widetilde{\iota}:\mathrm{L}^1_{1, +}(X, \mu)\hookrightarrow \mathcal{M}(X, \mu)$
    est d'image dense. 
\end{lemma}

\begin{proof}
    Notons que, pour tous $f\in\mathscr{L}^1(X, \mu), \varphi\in\mathscr{L}^\infty(X, \mu)$, on a, pour les dualités usuelles 
    $({\mathrm{L}^\infty}^*, \mathrm{L}^\infty)$ et $(\mathrm{L}^1, \mathrm{L}^\infty)$ :
    \begin{equation*}
        \ket{\iota(f), \varphi} = \iota(f)(\varphi) = \integral{}{}{f\varphi}{\mu} = \ket{f, \varphi}
    \end{equation*}
    Cela assure que $\iota$ est effectivement un plongement topologique pour les topologies faibles associées à ces dualités,
    et donc un plongement d'espaces vectoriels topologiques par linéarité.

    La partie importante est donc la densité, qui va reposer crucialement sur la convexité de $\mathrm{L}^1_{1, +}(X, \mu)$ et sur le théorème de séparation de Hahn-Banach.
    Notre objectif est de montrer l'inclusion $\mathcal{M}(X, \mu)\subseteq\closure{\iota(\mathrm{L}^1_{1, +}(X, \mu))}$ (l'adhérence étant prise dans 
    ${\mathrm{L}^\infty(X,\mu)}^*$ pour la topologie faible-$*$). On suppose donc que ce n'est pas le cas, ce qui fournit
    $m\in\mathcal{M}(X, \mu)$ tel que $m\notin\closure{\iota(\mathrm{L}^1_{1, +}(X, \mu))}$. Appliquons alors le théorème de séparation de
    Hahn-Banach au compact (séparé) convexe $\set{m}$ et au fermé convexe $\closure{\iota(\mathrm{L}^1_{1, +}(X, \mu))}$
    de l'espace localement convexe \emph{réel} $\floor{\mathrm{L}^\infty(X,\mu)^*}$
    \footnote{L'espace vectoriel réel sous-jacent à un espace localement convexe complexe
    est par définition localement convexe. } 
    sous-jacent à l'espace localement convexe \emph{complexe} $\mathrm{L}^\infty(X,\mu)^*$
    \footnote{Toute topologie faible est localement convexe, comme topologie initiale 
    pour une famille d'applications linéaires à valeurs dans l'espace localement convexe $\K$, où $\K\in\set{\R, \C}$ 
    désigne le corps des scalaires. }. 
    Cela fournit une forme $\R$-linéaire continue $f : \floor{\mathrm{L}^\infty(X,\mu)^*} \to \R$ et un réel 
    $a$ tel que $f(m)<a$ et $\forall x\in \closure{\iota(\mathrm{L}^1_{1, +}(X, \mu))}, f(x)>a$. 
    %En particulier, par le théorème des 
    %valeurs intermédiaires, $a = f(\alpha)$ pour un certain $\alpha\in\mathrm{L}^\infty(X,\mu)^*$.

    Écrivons alors $f=\Re g$ où $\fundef{g}{\mathrm{L}^\infty(X,\mu)^* &\to& \C}{x&\mapsto&f(x)-\i f(\i x)}$ est $\C$-linéaire et continue, et 
    supposons que $g$ est de la forme $\ev_\varphi$ pour $\varphi\in\mathrm{L}^\infty(X, \mu)$. On a alors, pour $u\in\mathrm{L}^1_{1, +}(X, \mu)$ :
\begin{align*}
    \ket{u, \Re\varphi - a} 
        &= \Re\ket{u, \varphi - a}\footnotemark \\
        &= \Re\ket{\iota(u), \varphi} - \ket{u, a} \\
        &= \Re g(\iota(u)) - a\ket{u, 1} \\
        &= f(\iota(u)) - a \\
        &> 0
\end{align*}
\footnotetext{Cette égalité provient de ce que $u$ est à valeurs réelles.}Tout élément positif de $\mathrm{L}^1(X, \mu)$ étant nul ou multiple positif d'un élément de $\mathrm{L}^1_{1, +}(X, \mu)$
\footnote{Plus précisément, si $u\in\mathrm{L}^1(X,\mu)$ est positif et non nul, on a $\integral{}{}{u}{\mu}=\norm{u}_{\mathrm{L}^1}\ne 0$. On peut alors poser 
$\widetilde{u}:=\frac1{\integral{}{}{u}{\mu}}\cdot u\in\mathrm{L}^1_{1, +}(X, \mu)$, de sorte que
$u = \left(\integral{}{}{u}{\mu}\right)\cdot\widetilde{u}$.}, cette inégalité s'étend à tout $u\in\mathrm{L}^1(X, \mu)$ positif. 
Autrement dit, $\Re\varphi$ est positive \emph{lorsqu'elle est vue comme forme linéaire sur $\mathrm{L}^1(X, \mu)$}. Si l'on admet
que $\Re\varphi - a$ est en fait positive au sens usuel, on obtient directement une contradiction avec la positivité de $m$, puisque : 
\begin{align*}
    m(\Re\varphi - a) 
        &= m(\varphi - a) - \i m(\Im\varphi) \\
        &= \Re m(\varphi - a)\footnotemark \\
        &= \Re g(m) - a \\
        &= f(m) - a \\
        &< 0
\end{align*}
\footnotetext{Comme $\Im\varphi$ est à valeurs réelles, la positivité de $m$ assure que $\i m(\Im\varphi)\in \i\R$.}

Pour conclure, il s'agit de montrer que les deux hypothèses que nous avons faites au cours de la preuve sont vérifiées. C'est l'objet 
des deux lemmes suivants.

\begin{lemma}\label{dual_of_weak}
    Soit $E, F$ deux $\K$-espaces vectoriels ($\K\in\set{\R, \C}$) en dualité, et supposons $E$ muni de la topologie faible $\sigma(E, F)$ associée à cette dualité. 
    Alors toute forme linéaire continue sur $E$ est de la forme $x\mapsto\ket{x, y}$ pour un certain $y\in F$.
\end{lemma}

\begin{lemma}\label{positive_of_positive_form}
    Soit $\varphi\in\mathscr{L}^\infty(X, \mu)$. Si l'intégrale $\integral{}{}{f\varphi}{\mu}$ est positive 
    pour toute fonction $f\in\mathscr{L}^1(X, \mu)$ positive $\mu$-presque partout, alors $\varphi$ est 
    positive $\mu$-presque partout.
\end{lemma}

Le lemme \ref{dual_of_weak} assure ainsi que l'on peut toujours écrire $g=\ev_\varphi$,
tandis que le lemme \ref{positive_of_positive_form} assure la positivité de $\Re\varphi - a$, ce qui conclut la preuve.
\end{proof}

\begin{proof}[Démonstration du lemme \ref{dual_of_weak}]
    Soit $\varphi$ une forme linéaire continue sur $E$. La continuité assure que l'ensemble $\mathbf{U} := \varphi\inv(B(0, 1))$ est un voisinage
    de $0$ pour $\sigma(E, F)$, ce qui fournit\footnote{Car la topologie $\sigma(E, F)$ est initiale 
    pour la famille des applications $\ket{\blank, y}$, où $y$ parcourt $F$.} $\varepsilon>0$, $I$ fini et $f : I\to F$ tels que
    $\mathbf{V} := \bigcap_{i\in I} \set{x\in E\tq \abs{\ket{x, f_i}}<\varepsilon}\subseteq\mathbf{U}$. 
    
    Pour chaque $i\in I$, notons donc $\psi_i$ la forme linéaire $\ket{\blank, f_i}$. Si $x\in\bigcap_{i\in I}\ker\psi_i$, on a bien sûr 
    $x\in\mathbf{V}\subseteq\mathbf{U}$ donc $\abs{\varphi(x)}<1$. Mais on a aussi $\forall t\in\R_+, tx\in\bigcap_{i\in I}\ker\psi_i$, d'où 
    $\forall t\in\R_+^*, \abs{\varphi(x)}<\frac1t$, d'où $x\in\ker\varphi$ en faisant tendre $t$ vers $0$. 
    On a donc montré $\bigcap_{i\in I}\ker\psi_i\subseteq\ker\varphi$. Pour conclure, on va s'appuyer sur un lemme usuel d'algèbre linéaire,
    dont nous redonnons la preuve ci-dessous.
    \begin{lemma}\label{span_dual_of_ker}
        Soient $k$ un corps, $E$ un $k$-espace vectoriel, $I$ un ensemble fini et $\psi : I\to E^\sharp$
        \footnote{On note $E^\sharp$ le dual algébrique de $E$.} une famille de formes linéaires sur $E$.
        Alors le sous-espace vectoriel de $E^\sharp$ engendré par l'image de $\psi$ est exactement l'ensemble 
        des formes linéaires $\varphi$ telles que $\ker\varphi\supseteq\bigcap_{i\in I}\ker\psi_i$.
    \end{lemma}

    Ce lemme fournit  $a : I\to\K$ telle que $\varphi = \sum_{i\in I} a_i\psi_i = \ket{\blank, \sum_{i\in I} a_i f_i}$,
    ce qui achève la preuve du lemme \ref{dual_of_weak}.
\end{proof}

\begin{proof}[Démonstration du lemme \ref{span_dual_of_ker}]
    Il est clair que toute combinaison linéaire des $\psi_i$ est nulle sur l'ensemble $\bigcap_{i\in I}\ker\psi_i$.
    Il s'agit donc de montrer que toute forme linéaire $\varphi$ vérifiant $\ker\varphi\supseteq\bigcap_{i\in I}\ker\psi_i$ est effectivement 
    combinaison linéaire des $\psi_i$. 
    
    Soit donc une telle forme $\varphi$. On considère l'application linéaire $\Psi : E\to k^I$ 
    dont les composantes sont les $\psi_i$, $F\subseteq k^I$ son image, et $\widetilde{\Psi}:E\to F$ l'application surjective induite.
    On a $\ker\widetilde{\Psi} = \ker\Psi = \bigcap_{i\in I}\ker\psi_i \subseteq \ker \varphi$, donc $\varphi$ se factorise 
    par $\widetilde{\Psi}$ en $\widetilde{\varphi}:F\to k$, que l'on prolonge en $\widehat{\varphi}:k^I\to k$ de manière arbitraire.
    En notant $e : I\to\C^I$ la base canonique et $e^*$ sa base duale, on a donc $\widehat{\varphi} = \sum_{i\in I} a_i e^*_i$
    pour une certaine famille $a : I\to\C$. Il vient enfin
    $\varphi = \widetilde{\varphi}\comp\widetilde{\Psi} = \widehat{\varphi}\comp\Psi = \sum_{i\in I} a_i (e^*_i \comp\Psi) = \sum_{i\in I} a_i\psi_i$,
    ce qui conclut.
\end{proof}

\begin{proof}[Démonstration du lemme \ref{positive_of_positive_form}]
    Supposons que l'ensemble $\mathbf{S} := \set{x\in X\tq\varphi(x)<0}$ n'est \emph{pas} $\mu$-localement-négligeable. Par définition, cela fournit 
    $\mathbf{T}\in\Bor(X)$ de mesure finie tel que $0<\mu(\mathbf{T}\cap\mathbf{S})<+\infty$, de sorte que $\indic_{\mathbf{T}\cap\mathbf{S}}$ est positive et intégrable. L'intégrale
    $\integral{}{}{\indic_{\mathbf{T}\cap\mathbf{S}}\varphi}{\mu}$ est donc positive par hypothèse, et négative comme intégrale d'une fonction négative,
    donc nulle. La fonction positive $-\indic_{\mathbf{T}\cap\mathbf{S}}\varphi$ est d'intégrale nulle, 
    l'inégalité de Markov \ref{markov_and_consequence} assure donc qu'elle est nulle presque partout, ce qui contredit le fait qu'elle est non-nulle 
    sur l'ensemble $\mathbf{T}\cap\mathbf{S}$ de mesure non-nulle.
\end{proof}

%,
%cette dernière condition pouvant se réécrire : 
%\begin{equation}\label{cara_left_uniform_continuous}
%    \forall\varepsilon>0,\set{g\in\Gamma\tq\forall x\in\Gamma, \abs{f(gx)-f(x)}<\varepsilon}\in\nhds_1
%\end{equation}
%De manière équivalente, $f\in\mathscr{L}^\infty(\Gamma)$ appartient à $UC_b(\Gamma)$ \ssi l'application $\fundef{\ev_f\comp\lambda}{\Gamma&\to&\mathrm{L}^\infty(\Gamma)}{g&\mapsto&\lambda(g)(f)}$ est continue en $1$
%\footnote{
%    En effet, le symmétrique d'un voisinage de $1$ étant voisinage de $1$, on a:
%    \begin{align*}
%        (\ref{cara_left_uniform_continuous})
%            &\iff \forall\varepsilon>0,\exists U\in\nhds_1^\Gamma,\forall x\in\Gamma,\abs{f(gx)-f(x)}\le\varepsilon \\
%            &\iff \forall\varepsilon>0,\exists U\in\nhds_1^\Gamma,\norm{\lambda(g\inv)(f) - f}_{\mathrm{L}^\infty}\le\varepsilon \\
%            &\iff \forall\varepsilon>0,\exists U\in\nhds_1^\Gamma,\norm{\lambda(g)(f) - f}_{\mathrm{L}^\infty}\le\varepsilon \\
%            &\iff \ev_f\comp\lambda \text{ continue en 1}
%    \end{align*}
%    De plus, si 
%}.

\subsection{Démonstration du théorème \ref{amenable_TFAE}} 

\begin{proof}
    Commençons par montrer \framebox{$(\ref{amenable_TFAE/amenable})\implies(\ref{amenable_TFAE/topological_mean})$}. Soit donc 
    $m$ une moyenne invariante sur $\Gamma$, et notons $\widetilde{m}$ sa restriction à $UC_b(\Gamma_d)$
    \footnote{On identifie $UC_b(\Gamma_d)$ à son image par l'application 
    $UC_b(\Gamma_d)\hookrightarrow\mathscr{L}^\infty(\Gamma)\twoheadrightarrow\mathrm{L}^\infty(\Gamma)$, qui est un plongement isométrique car 
    les normes $\norm{\cdot}_\infty$ et $\norm{\cdot}_{\mathrm{L}^\infty}$ coïncident sur les fonctions 
    continues.}. Pour $f\in UC_b(\Gamma_d)$ et $\varphi\in\mathscr{L}^1_{1, +}(\Gamma)$ \emph{à support compact}, étudions la fonction
    $\fundef{\varphi\cdot(\ev_f\comp\lambda)}{\Gamma&\to&UC_b(\Gamma_d)}{g&\mapsto&\varphi(g)\cdot\lambda(g)(f)}$. Le 
    lemme \ref{UCB_iff_translate} assure que $\ev_f\comp\lambda$ est continue, donc mesurable. 
    \TODO{Il nous faudrait de la mesurabilité forte !}
\end{proof}

\section{Contenance faible et C*-algèbres}

\newpage

\section{Annexes}

\subsection*{Résultats usuels de théorie de la mesure}

Cette section regroupe, avec ou sans démonstration, des résultats de théorie de la mesure utilisés au cours du document.
Nous nous cantonnons ici strictement aux mesures usuelles ($\sigma$-additives), le cas moins usuel des contenu et de 
l'intégration associée étant traitée dans le corps du texte. Enfin, nous supposons connus les définitions de tribu et de mesure,
la construction de l'intégrale associée à une mesure, ainsi que les théorèmes de convergence monotone et dominée.
Si $(X, \mathcal{A})$ est un espace mesurable, on note $\mathscr{S}(X, \mathcal{A})$ l'ensemble des \emph{fonctions simples}
sur $X$, c'est à dire des fonctions mesurables $f : (X, \mathcal{A})\to(\C, \Bor(\C))$ d'image finie. 

Avant toute chose, rappelons l'inégalité de Markov et une de ses conséquences élémentaires, qui nous servira à de nombreuses reprises.
\begin{proposition}\label{markov_and_consequence}
    Soit $(X, \mathcal{A}, \mu)$ un espace mesuré et $f:X\to\overline{\R_+}$ une fonction mesurable positive.
    On a $\forall a>0, a\cdot\mu\left(f\inv([a, +\infty])\right) \le \integral{}{}{f}{\mu}$. \\
    Par conséquent, si $\integral{}{}{f}{\mu} = 0$, on a $f =_\mu 0$, où $=_\mu$ désigne la relation 
    d'égalité $\mu$-presque-partout.
\end{proposition}

\begin{proof}
    L'inégalité de Markov est une conséquence directe de la croissance de l'intégrale appliquée à 
    l'inégalité $a\cdot\indic_{\set{x\tq f(x)\le a}} \le f$. Pour sa conséquence, notons que le petit théorème de convergence monotone donne :
    \begin{equation*}
        \mu(f\inv([2^{-n}, +\infty])) \xrightarrow[n\to+\infty]{} \mu\left(\bigcup_{i\in\N} f\inv([2^{-i}, +\infty])\right) = \mu(f\inv(\set{0}\compl))
    \end{equation*}
    Mais l'hypothèse $\integral{}{}{f}{\mu} = 0$ implique, \emph{via} l'inégalité de Markov, que pour tout $n\in\N$, $\mu(f\inv([2^{-n}, +\infty]))=0$,
    ce qui entraîne $f =_\mu 0$.
\end{proof}

Nous allons maintenant énoncer les résultats fondamentaux sur les espaces $\mathrm{L}^p$. Nous allons en fait considérer une variante subtile de 
la définition, qui est plus adaptée à l'analyse fonctionelle en ce que l'isométrie $\mathrm{L}^\infty \simeq (\mathrm{L}^1)^*$
sera valable \emph{sans hypothèse de $\sigma$-finitude}. Pour cela, on introduit la notion suivante.

\begin{definition}
    Soit $(X, \mathcal{A}, \mu)$ un espace mesuré, et soit $A\in\parts{X}$. On dit que $A$ est \emph{$\mu$-localement-négligeable} 
    si, pour tout $B\in\mathcal{A}$ \emph{de mesure finie}, $B\cap A$ est $\mu$-négligeable. \\
    Par analogie avec le cas des ensembles négligeables, on dit qu'un prédicat $P$ est valable 
    \emph{$\mu$-localement-presque-partout} si l'ensemble associé est $\mu$-localement-négligeable,
    et on note $f =_\mu^{loc} g$ si les fonctions $f$ et $g$ sont égales $\mu$-localement-presque-partout.
    Enfin, on définit le \emph{supremum $\mu$-localement-essentiel} d'une fonction mesurable $f:(X, \mathcal{A})\to(\overline{\R},\Bor(\overline{\R}))$ par 
    $\sup\ess_\mu^{loc} f := \inf\set{y\tq f\le y\text{ $\mu$-localement-presque-partout}}$.
\end{definition}

Remarquons directement que, si l'espace mesuré $(X, \mathcal{A}, \mu)$ est $\sigma$-fini, 
tout ensemble $\mu$-localement-négligeable est en fait $\mu$-négligeable. Cela garantit en particulier que, 
pour des fonctions intégrables, les relations $=_\mu$ et $=_\mu^{loc}$ coïncident, en vertu du lemme suivant.

\begin{lemma}
    Soit $f:X\to\C$ intégrable. L'ensemble $\set{x\tq f(x)\ne 0}$ est $\sigma$-fini\footnote{C'est à dire qu'il est $\sigma$-fini comme espace mesuré, 
    lorsqu'on le munit de la mesure restreinte}. 
\end{lemma}

\begin{proof}
    Chaque $\set{x\tq \abs{f(x)}\ge 2^{-n}}$ est de mesure finie par l'inégalité de Markov, et l'union de ses ensembles pour 
    $n\in\N$ recouvre $\set{x\tq f(x)\ne 0}$.
\end{proof}

\TODO{Définitions et propriétés des espaces $\mathrm{L}^p$ avec le cette notion de localement négligeable, 
sans preuve mais petite explication de ce qui change dans le cas $p=+\infty$.}

%Rappelons maintenant les définitions et propriétés usuelles des espaces $\mathrm{L}^p$, d'abord pour $p<+\infty$. Pour $(X, \mathcal{A}, \mu)$ espace mesuré et $p\in[1, +\infty[$, on 
%note $\mathscr{L}^p(X, \mu)$ l'espace vectoriel semi-normé des fonctions mesurables $f:(X, \mathcal{A})\to(\C, \Bor(\C))$ telles que $\abs{f}^p$ soit intégrable,
%muni de la seminorme $\norm{\cdot}_{\mathscr{L}^p} : f\mapsto\left(\integral{}{}{\abs{f}^p}{\mu}\right)^{\frac1p}$; on note alors $\mathrm{L}^p(X, \mu)$ l'espace 
%vectoriel normé associé, c'est à dire le quotient de $\mathscr{L}^p(X, \mu)$ par le sous-espace fermé $\set{f\tq\norm{f}_{\mathscr{L}^p}=0}$,
%muni de la norme induite notée $\norm{\cdot}_{\mathrm{L}^p}$. Comme il est usuel de le faire dans la littérature, on oublie l'application de 
%projection sur le quotient : si $f\in\mathscr{L}^p(X, \mu)$, on note aussi $f$ son image dans $\mathrm{L}^p(X, \mu)$.
%
%La proposition \ref{markov_and_consequence} assure que l'ensemble $\set{f\tq\norm{f}_{\mathscr{L}^p}=0}$ est exactement l'ensemble des fonctions 
%$\mu$-négligeables. En effet, la fonction $\abs{f}^p$ étant positive, son intégrale est nulle \ssi
%$\abs{f}^p =_\mu 0$, ce qui équivaut à $f =_\mu 0$. On peut donc voir $\mathrm{L}^p(X, \mu)$ comme le quotient de 
%$\mathscr{L}^p(X, \mu)$ par la relation d'égalité $\mu$-presque partout.
%
%Le théorème suivant rassemble les résultats dont nous aurons besoin sur les espaces $\mathrm{L}^p$.
%
%\begin{theorem}
%    Soit $(X, \mathcal{A}, \mu)$ un espace mesuré et $p\in[1, +\infty[$. Notons $q$ l'unique élément de $]1, +\infty]$
%    vérifiant $\frac1p + \frac1q = 1$.
%    \begin{enumerate}[(i)]
%        \item L'espace $\mathscr{S}(X, \mathcal{A})\cap\mathscr{L}^1(X, \mu)$ des fonctions simples intégrables 
%            est (inclu et) dense dans $\mathscr{L}^p(X, \mu)$. Son image dans $\mathrm{L}^p(X, \mu)$ est donc 
%            toujours dense.
%        \item L'espace $\mathrm{L}^p(X, \mu)$ est complet.
%        \item L'application bilinéaire $\fundef{B}{\mathcal{F}(X, \C)\times\mathcal{F}(X, \C)&\to&\mathcal{F}(X, \C)}{(f, g)&\mapsto&fg}$ 
%            se restreint et passe au quotient en une application bilinéaire continue $\widetilde{B} : \mathcal{}.
%    \end{enumerate}
%\end{theorem}

%\begin{definition}
%    Soit $(X, \mathcal{A})$ un espace mesurable. Une mesure $\mu$ sur cet espace est dite \emph{semi-finie} si tout ensemble
%    $A\in\mathcal{A}$ de mesure non-nulle contient un ensemble $B\in\mathcal{A}$ de mesure 
%    \emph{finie} non-nulle.
%\end{definition}

\begin{definition}
    Soit $X$ un espace topologique localement compact et séparé. Une \emph{mesure de Radon} sur $X$
    est une mesure $\mu$ sur l'espace mesuré $(X, \Bor(X))$ vérifiant les trois conditions suivantes :
    \begin{itemize}
        \item Pour tout $K\subseteq X$ compact, $\mu(X)<+\infty$.
        \item Régularité extérieure : 
        \begin{equation*}
            \forall A\in\Bor(X), \mu(A) = \inf \set{\mu(U) \tq U\text{ ouvert}, A\subseteq U}
        \end{equation*}
        \item Régularité intérieure pour les ouverts : 
        \begin{equation*}
            \forall U\text{ ouvert}, \mu(U) = \sup \set{\mu(K) \tq K\text{ compact}, K\subseteq U}
        \end{equation*}
    \end{itemize}
\end{definition}

La conjonction des deux conditions de régularité donne, pour tout borélien $A$ de mesure finie et tout $\varepsilon>0$,
un ouvert $U\supseteq A$ de mesure finie et un compact $K\subseteq U$ tels que $\mu(U)<\mu(A)+\varepsilon$ et 
$\mu(U)<\mu(K)+\varepsilon$. Bien sûr, on ne peut ici pas supposer $K\subseteq A$, mais cela permet 
tout de même d'obtenir des résultats intéressants.

On peut notamment appliquer le lemme d'Urysohn, pour obtenir $f:X\to[0,1]$ continue à support compact
vérifiant $\forall x\in K, f(x) = 1$ et $\Supp f\subseteq U$. On a alors, pour tout $p\in[1, +\infty[$ : 
\begin{align*}
    \integral{}{}{\abs{f-\indic_A}^p}{\mu} 
        &= \integral{A}{}{\abs{f-1}^p}{\mu} + \integral{U\setminus A}{}{\abs{f}^p}{\mu} \\
        &\le \integral{U}{}{(1-f)^p}{\mu} + \integral{U\setminus A}{}{f^p}{\mu} \\
        &= \mu(U\setminus K) + \mu(U\setminus A) \\
        &< 2\varepsilon
\end{align*}
On est donc capable d'approcher en norme $\mathscr{L}^p$ l'indicatrice de tout ensemble $A$ de mesure finie par des fonctions continues 
à support compact. Ces indicatrices engendrant un sous-espace dense de $\mathrm{L}^p(X, \mu)$ (\TODO{ref}), on vient de montrer le résultat suivant.

\begin{proposition}
    Si $\mu$ est de Radon, l'espace $C_c(X)$ est dense dans $\mathrm{L}^p(X, \mu)$ pour tout $p\in[1, +\infty[$.
\end{proposition}

\TODO{Radon implique semifinie}

\subsection*{Sur la convergence uniforme des translatées d'une fonction}

Soient $X$ un ensemble et $Y$ un espace uniforme. On note $\mathcal{F}_u(X, Y)$ l'ensemble des 
fonctions de $X$ dans $Y$ muni de la structure uniforme de la convergence uniforme, dont le filtre des entourages 
$\mathcal{U}(\mathcal{F}_u(X, Y))$ admet pour base les ensembles de la forme 
$\mathbf{S}_X(V) := \set{(f, g)\tq\forall x\in X, (f(x),g(x))\in V}$ lorsque $V$ parcourt (une base de) $\mathcal{U}Y$.

Le but de cette section est de démontrer le théorème suivante.
\begin{theorem}\label{uniform_continuous_iff_postcomp}
    Soient $G$ un groupe topologique, $X$ un espace uniforme, et $f:G\to X$. Les assertions suivantes sont équivalentes.
    \begin{enumerate}[(i)]
        \item $f:G_d\to X$ est uniformément continue \emph{pour la structure uniforme droite}.
        \item L'application $\fundef{\ev_f\comp\lambda}{G_s&\to&\mathcal{F}_u(G,X)}{g&\mapsto&\lambda(g)(f)}$ est uniformément continue
            \emph{pour la structure uniforme gauche}.
        \item L'application $\ev_f\comp\lambda:G\to\mathcal{F}_u(G,X)$ est continue.
    \end{enumerate}
\end{theorem}

\begin{lemma}
    Soit $G$ un groupe topologique. L'application $\ell : G_d\mapsto\mathcal{F}_u(G, G_d)$
    \footnote{On écrit $\mathcal{F}_u(G, G_d)$ au lieu de $\mathcal{F}_u(G_d, G_d)$ car l'espace
    ne dépend pas de la structure uniforme du premier facteur (en fait, on devrait même écrire l'ensemble $G$ 
    à la place du groupe topologique $G$).} est un plongement d'espaces uniformes.
\end{lemma}

\begin{proof}
    Il est évident que $\ell$ est injective, puisque $\ell(g)(1)=g$. Il s'agit
    donc de montrer que $\mathcal{U}G_d = (\ell\times\ell)^*\mathcal{U}(\mathcal{F}_u(G, G_d))$,
    et on va pour cela remarquer que la famille $\mathcal{B}:=\set{\divop_d\inv(V)}_{V\in\nhds_1^G}$ est une base commune pour ces deux filtres.

    L'égalité $\mathcal{U}G_d=(\divop_d)^*\nhds_1^G$ garantit que $\mathcal{B}$ est effectivement une base de $\mathcal{U}G_d$,
    et donc que $\mathcal{C}:=\set{(\ell\times\ell)\inv\left(\mathbf{S}_G(\divop_d\inv(V))\right)}_{V\in\nhds_1^G}$ est une base de $(\ell\times\ell)^*\mathcal{U}(\mathcal{F}_u(G, G_d))$.
    Or, pour $V\in\nhds_1^G$ et $g, h\in G$, on a :
    \begin{align*}
        (g, h)\in(\ell\times\ell)\inv\left(\mathbf{S}_G(\divop_d\inv(V))\right) 
            &\iff (\ell_g,\ell_h)\in \mathbf{S}_G(\divop_d\inv(V)) \\
            &\iff \forall x\in G, (\ell_g(x),\ell_h(x))\in\divop_d\inv(V) \\
            &\iff \forall x\in G, g x x\inv h\inv \in V \\
            &\iff g h\inv \in V \\
            &\iff (g, h) \in\divop_d\inv(V)
    \end{align*}
    Ceci montre que $\mathcal{B}=\mathcal{C}$ est bien une base de $(\ell\times\ell)^*\mathcal{U}(\mathcal{F}_u(G, G_d))$, ce qui conclut.
\end{proof}

Remarquons que l'application $\ev_f\comp\lambda$ peut encore s'écrire comme la composition suivante :
\begin{center}
    \begin{tikzcd}
        G_s \arrow[r, , "\invop", "\sim"' inner sep=.4mm] &G_d \arrow[r, hook, "\ell"] &\mathcal{F}_u(G, G_d) \arrow[r, , "(f\comp\blank)"] &\mathcal{F}_u(G, X)
    \end{tikzcd}
\end{center}
Or, $\invop$ et $\ell$ étant respectivement un isomorphisme et un plongement d'espaces uniformes (et donc d'espaces topologiques), 
la continuité (resp. la continuité uniforme) de $\ev_f\comp\lambda$ équivaut à la continuité (resp. la continuité uniforme) de 
$(f\comp\blank):\mathcal{F}_u(G, G_d)\to\mathcal{F}_u(G, X)$. Le théorème \ref{uniform_continuous_iff_postcomp} découle donc 
du lemme suivant. 

\begin{lemma}
    Soient $X$, $Y$ deux espaces uniformes et $f:X\to Y$. Les propositions suivantes sont équivalentes.
    \begin{enumerate}[(i)]
        \item $f$ est uniformément continue
        \item $(f\comp\blank) : \mathcal{F}_u(X, X)\to\mathcal{F}_u(X, Y)$ est uniformément continue
        \item $(f\comp\blank)$ est continue
        \item $(f\comp\blank)$ est continue en $\id_X$
    \end{enumerate}
\end{lemma}

\begin{proof}
    Montrons \framebox{$(i)\implies(ii)$}. On suppose donc $f$ uniformément continue, et on se donne $V\in\mathcal{U}Y$. On a alors :
    \begin{align*}
        ((f\comp\blank)\times(f\comp\blank))\inv(\mathbf{S}_X(V)) 
            &= \set{(\varphi_1, \varphi_2)\tq\forall x, (f(\varphi_1(x)), f(\varphi_2(x)))\in V} \\
            &= \mathbf{S}_X((f\times f)\inv(V))
    \end{align*}
    L'uniforme continuité de $f$ garantit que $(f\times f)\inv(V)\in\mathcal{U}X$, ce qui entraîne :
    \begin{equation*}
        \mathbf{S}_X((f\times f)\inv(V))\in\mathcal{U}(\mathcal{F}_u(X, X))
    \end{equation*}
    Ceci montre que $(f\comp\blank)$ est uniformément continue.

    Les implications \framebox{$(ii)\implies(iii)$} et \framebox{$(iii)\implies(iv)$} étant claires, montrons
    \framebox{$(iv)\implies(i)$}. On suppose donc $(f\comp\blank)$ continue en $\id_X$, et on se donne 
    $V\in\mathcal{U}Y$. La définition de la convergence uniforme assure que 
    $\mathbf{T}:=\set{h : X\to Y\tq \forall x, (h(x), f(x))\in V}$ est un voisinage de $f = (f\comp\blank)(\id_X)$ dans 
    $\mathcal{F}_u(X, Y)$. Par continuité, $(f\comp\blank)\inv(\mathbf{T})$ est donc un voisinage de $\id_X$ dans $\mathcal{F}_u(X, X)$,
    ce qui signifie qu'il contient un ensemble de la forme $\mathbf{U}:=\set{\varphi : X\to X\tq \forall x, (\varphi(x), x)\in W}$ pour 
    un $W\in\mathcal{U}X$. 
    
    Or, pour $(x_1, x_2)\in W$, il est clair qu'il existe une fonction $\varphi\in\mathbf{U}$ telle que $\varphi(x_2) = x_1$
    \footnote{On peut par exemple poser $\varphi(x_2)=x_1$ et $\varphi(x) = x$ pour $x\ne x_2$.}.
    Mais alors $f\comp\varphi\in\mathbf{T}$, d'où, en évaluant en $x_2$, $(f(x_1),f(x_2))\in V$.
    On a donc montré que $W\subseteq (f\times f)\inv(W)$, ce qui garantit l'uniforme continuité de $f$.

    Ceci conclut la preuve du lemme, ainsi que celle du théorème \ref{uniform_continuous_iff_postcomp}.
\end{proof}

\end{document}