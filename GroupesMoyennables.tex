\documentclass[a4paper,12pt]{article}
\usepackage[utf8]{inputenc}
\usepackage[french]{babel}
\usepackage{amssymb}
\usepackage{amsmath}
\usepackage{bbm}
\usepackage{amsthm}
\usepackage{a4wide}
\usepackage{mathrsfs}
\usepackage{stmaryrd}
\usepackage{mathtools}
\usepackage{graphicx}
\usepackage{hyperref}
\usepackage{faktor}
\usepackage{enumerate}
\usepackage[T1]{fontenc}

\newtheorem{theorem}{Théorème}[section]
\newtheorem{proposition}[theorem]{Proposition}
\newtheorem{definition}[theorem]{Définition}
\newtheorem{corollary}[theorem]{Corollaire}
\newtheorem{lemma}[theorem]{Lemme}
\newtheorem{remark}[theorem]{Remarque}

\newcommand{\R}{\mathbb{R}}
\newcommand{\N}{\mathbb{N}}
\newcommand{\Q}{\mathbb{Q}}
\newcommand{\Z}{\mathbb{Z}}
\newcommand{\C}{\mathbb{C}}
\newcommand{\K}{\mathbb{K}}
\newcommand{\F}{\mathcal{F}}
\newcommand{\G}{\mathcal{G}}
\newcommand{\U}{\mathcal{U}}
\newcommand{\Bor}{\mathcal{B}}
\newcommand{\norm}[1]{\left\Vert #1\right\Vert}
\newcommand{\abs}[1]{\left\vert#1\right\vert}
\newcommand{\card}[1]{\abs{#1}}
\newcommand{\ket}[1]{\left\langle #1 \right\rangle}
\newcommand{\halfilon}{{\frac\varepsilon2}}
\newcommand{\set}[1]{\left\{ #1 \right\}}
\newcommand{\indic}{\mathbbm{1}}
\newcommand{\integral}[2]{\int #1~\mathrm{d}#2}
\newcommand\fundef[3]{#1: \left\{\begin{array}{ccc}#2\\#3\end{array}\right.}
\newcommand\funlam[2]{\left\{\begin{array}{ccc}#1\\#2\end{array}\right.}
\newcommand{\tq}{\;\middle|\;}
\newcommand{\ssi}{si et seulement si }
\newcommand{\interior}[1]{\mathring{#1}}
\newcommand{\closure}[1]{\overline{#1}}
\newcommand{\transpose}[1]{\prescript{t}{}{#1}{}{}}
\newcommand{\inv}{^{-1}}
\newcommand{\infi}{\bigwedge}
\newcommand{\supr}{\bigvee}
\newcommand{\comp}{\circ}
\newcommand{\nhds}{\mathcal{N}}
\renewcommand{\implies}{\Rightarrow}
\renewcommand{\iff}{\Leftrightarrow}
\newcommand{\blank}{{-}}
\newcommand{\invop}{\mathrm{inv}}

\DeclareMathOperator{\sgn}{sgn}
\DeclareMathOperator{\Id}{id}
\DeclareMathOperator{\id}{id}
\DeclareMathOperator{\Mat}{Mat}
\DeclareMathOperator{\Vect}{Vect}
\DeclareMathOperator{\Ima}{Im}
\DeclareMathOperator{\solset}{Sol}
\DeclareMathOperator{\Sp}{Sp}

\begin{document}

\begin{titlepage}
\title{Groupes Moyennables}
\author{Anatole \textsc{Dedecker}}
\maketitle
\thispagestyle{empty}
\end{titlepage}

\tableofcontents
\thispagestyle{empty}

\clearpage

\pagenumbering{arabic}

\section*{Conventions et remarques préliminaires}

Dans ce mémoire, nous utiliserons le terme de \textit{mesure} pour désigner une mesure \textbf{finiment}-additive sur un 
\textit{espace mesurable}, c'est à dire un couple $(X, \mathcal{A})$ où $\mathcal{A}$ est une $\sigma$-algèbre. 
Lorsqu'une telle mesure est de plus $\sigma$-additive (ce qui est souvent inclus dans la définition de \og{}mesure\fg{}),
nous dirons qu'il s'agit d'une \textit{$\sigma$-mesure}. 

Nous admettrons qu'il est possible de définir une théorie de l'intégration pour toute mesure, les théorèmes de convergence
monotone et dominée n'étant bien sûr valables que dans le cas des $\sigma$-mesures. 

Si $\varphi:(X,\mathcal{A})\to(Y,\mathcal{B})$ est une application mesurable et $m$ est une mesure sur $(X, \mathcal{A})$, 
nous noterons $\varphi_*m$ la mesure image sur $(Y, \mathcal{B})$. C'est une $\sigma$-mesure si $m$ en est une, et intégrer 
une fonction $f$ selon $\varphi_*m$ revient exactement à intégrer $f\comp\varphi$ selon $m$. Enfin, c'est une construction
fonctorielle, au sens où l'on a $(\psi\comp\varphi)_*m = \psi_*\varphi_*m$ pour toute application mesurable 
$\psi : (Y, \mathcal{B})\to(Z, \mathcal{C})$. 

TODO mesure à densité

\paragraph{}

Conformément à la convention dans le monde anglophone, et pour éviter toute confusion, nous dirons qu'un espace topologique
$X$ est \emph{compact} s'il vérifie la propriété de Borel-Lebesgue, sans hypothèse de séparation. 
Nous dirons que $X$ est \emph{localement compact} si, pour tout $x\in X$, le filtre $\nhds_x$ des voisinages de $x$ admet
une \textbf{base} formée d'ensembles compacts. Si $X$ est séparé, on retrouve que $X$ est localement compact 
\ssi tout point admet \textbf{un} voisinage compact.

\paragraph{}
Si $G$ est un groupe et $g\in G$, nous noterons $\invop$ l'inversion et $\ell_g$, $r_g$ les applications de tranlations:
\begin{equation*}
    \fundef{\invop = (\blank)\inv}{G&\to& G}{x&\mapsto& x\inv}\text{;}\quad\fundef{\ell_g=(g\blank)}{G&\to& G}{x&\mapsto& gx}\text{;}\quad\fundef{r_g=(\blank g)}{G&\to& G}{x&\mapsto& xg}
\end{equation*}

Si le groupe $G$ est muni d'une $\sigma$-algèbre stable par les tranlations droite et gauche, ce qui sera notamment le cas
si $G$ est un groupe topologique muni de sa tribu borélienne $\Bor(G)$, nous dirons qu'une mesure $m$ sur $G$ est \textit{invariante
(par translations) à gauche (resp. à droite)} si $\forall g\in G, {\ell_g}_*m = m$ (resp. ${r_g}_*m = m$). Si les deux conditions
sont vérifiées, nous parlerons simplement de mesure \textit{invariante par translations}.

Une \textit{mesure de Haar à gauche (resp. à droite)} sur un groupe topologique $G$ séparé et localement compact est une mesure de
Radon sur $G$ invariante par tranlations à gauche (resp. à droite).

TODO rappeler théorème

\section{Premières définitions}

\subsection{Mesure et moyenne}

Fixons $\Gamma$ un groupe topologique. La notion qui va nous intéresser dans ces notes est la suivante.

\begin{definition}
    On dit qu'un groupe topologique $\Gamma$ est \emph{moyennable} s'il admet une 
    mesure de probabilité borélienne invariante par translations.
\end{definition}

Notons tout de suite que l'invariance bilatère n'impose pas de restriction suplémentaire. 
\begin{proposition}\label{bilateral_of_left}
    Supposons qu'il existe une mesure de probabilité borélienne $m$ sur $\Gamma$, invariante par translations à gauche.
    Alors $\Gamma$ est un groupe moyennable.
\end{proposition} 

\begin{proof}
    Il s'agit donc de construire une \textbf{autre} mesure $n$ sur $\Gamma$ qui soit cette-fois invariante des deux côtés.
    Posons d'abord, pour $A\in\Bor(G)$, $\fundef{f_A}{\Gamma&\to& \Gamma}{g&\mapsto& m(Ag\inv)}$. Chaque $f_A$ est bornée par $1 = m(\Gamma)$, 
    donc intégrable pour la mesure de probabilité $\invop_*m$. Posons alors $n(A) := \integral{f_A}{(\invop_*m)}$. \\
    Notons que, pour $A$ et $B$ boréliens disjoints, ainsi que $g, x\in\Gamma$, on a :
    \begin{gather*}
        f_\Gamma = \indic \\
        f_{A\cup B} = f_A + f_B \\
        f_{g\inv A}(x) = m(g\inv Ax\inv) = ({\ell_g}_*m)(Ax\inv) = m(Ax\inv) = f_A(x) \\
        f_{Ag\inv}(x) = m(Ag\inv x\inv) = f_A(xg) = (f_A \comp r_g)(x)
    \end{gather*}
    En intégrant ces relations, on obtient bien :
    \begin{gather*}
        n(\Gamma) = (\invop_*m)(\Gamma) = 1 \\
        n(A\cup B) = n(A) + n(B) \\
        ({\ell_g}_*n)(A) = n(g\inv A) = n(A) \\
        ({r_g}_*n)(A) = \integral{f_A\comp r_g\comp\invop}{m} = \integral{f_A\comp\invop\comp \ell_g}{m} = n(A)
    \end{gather*}
    Ce qui conlut.
\end{proof}

Les premiers exemples de groupes moyennables sont les groupes compacts séparés (et en particulier les groupes finis discrets).
En effet, la mesure de Haar normalisée d'un tel groupe est une mesure de probabilité invariante par tranlation. Les phénomènes
plus intéressants vont donc se produire dans le cas non-compact, et il convient de noter que ces groupes ne peuvent pas avoir
de $\sigma$-mesure de probabilité invariante par translations (TODO ref). C'est donc l'affaibilissement de la 
$\sigma$-additivité en additivité finie qui engendre la complexité. \\

Comme constaté dans la preuve ci-dessus, il va sans surprise être très souvent utile d'étudier la mesure $m$ à travers la théorie de l'intégration.
On dispose pour cela du résultat suivant.
\begin{proposition}\label{repr}
    Soit $(X, \mathcal{A})$ un espace mesurable. Pour $m$ mesure de probabilité, [TODO]
\end{proposition}

\begin{proof}
    TODO
\end{proof}

Cela nous amène à donner la définition suivante.

\begin{definition}
    Soit $\Gamma$ un groupe localement compact et séparé. Une \emph{moyenne à gauche (resp. à droite)} sur $\Gamma$ est une forme 
    linéaire positive $T : \mathrm{L}^\infty(\Gamma)\to\C$ vérifiant $T(\indic) = 1$ et $\forall g\in\Gamma, T\comp\lambda_g = T$
    (resp. $T\comp\rho_g = T$). Une moyenne bilatère sera simplement appelée \emph{moyenne}.
\end{definition}

\begin{remark}
    Les moyennes à gauche sur $\Gamma$ sont exactement les morphismes de la représentation $(\mathrm{L}^\infty(\Gamma), \lambda)$ de 
    $\Gamma$ vers la représentation triviale, avec la condition suppplémentaire que $T(\indic) = 1$.
\end{remark}

Remarquons que la bijection $T$ de la proposition \ref{repr} fait correspondre les mesures de probabilité 
invariantes aux moyennes sur $\Gamma$. En effet, si $m$ est invariante à gauche, alors pour tous 
$f\in\mathscr{L}^\infty(\Gamma)$ et $g\in\Gamma$, on a 
$T_m(\lambda_g(f)) = \integral{f\comp\ell_{g\inv}}{m} = \integral{f}{m} = T_m(f)$. Réciproquement, si $T$ est une 
moyenne à gauche, alors la mesure $m$ associe vérifie, pour tous $A\in\Bor(\Gamma)$ et $g\in\Gamma$, 
$m(g\inv A) = T(\indic_{g\inv A}) = T(\indic_A \comp\ell_g) = T(\lambda_{g\inv}(\indic_A)) = T(\indic_A) = m(A)$.
En prenant en compte la proposition \ref{bilateral_of_left}, on vient donc de montrer le résultat suivant.
\begin{proposition}
    Un groupe localement compact séparé $\Gamma$ est moyennable \ssi il admet une moyenne (resp. une moyenne à gauche, resp. une manière à droite).
\end{proposition}

Dans la suite, nous supposons toujours que le groupe topologique $\Gamma$ est séparé et localement compact.
\paragraph{}

Pour étudier les moyennes avec les outils de l'analyse fonctionnelle, il sera utile de transformer la condition de positivité en une condition
de norme. C'est ce que fournit le lemme suivant.

\begin{lemma}
    Soit $X$ un espace topologique séparé et localement compact muni d'une mesure de Radon et $\varphi:\mathrm{L}^\infty(X)\to\C$ une forme linéaire. 
    Si $\varphi(\indic) = 1$, on a :
    \begin{equation*}
        \norm{\varphi} = 1 \iff \forall a \ge 0, \varphi(a) \geq 0
    \end{equation*}
\end{lemma}

\begin{proof}
    Supposons d'abord $\norm{\varphi} = 1$, et soit $a\ge 0$ de norme 1. On a donc, pour $x\in X$, $a(x), 1-a(x)\in[0,1]$,
    d'où enfin $\norm{\indic-a}\le 1$. Mais on a $1 = \varphi(a) + \varphi(\indic-a) \le \varphi(a) + \abs{\varphi(\indic-a)} \le \varphi(a) + \norm{\indic - a}$, d'où
    $\varphi(a)\ge 1 - \norm{\indic-a}\ge0$. \\
    Supposons maintenant $\varphi$ positif. Notons déjà que l'égalité $\varphi(\indic) = 1$ entraîne $\norm{\varphi}\ge1$. Soit donc $a\in\mathscr{L}^\infty(X)$ quelconque,
    et notons que $-\norm{a}\indic\le a\le\norm{a}\indic$, de sorte que $-\norm{a}\le\varphi(a)\le\norm{a}$, ce qui conclut. \\
\end{proof}

L'intérêt de la condition $\norm\varphi = 1$ dans ce lemme est qu'elle est conservée lors du prolongement de l'application $\varphi$ par le théorème
de Hahn-Banach, alors qu'il n'est pas clair du tout qu'on puisse prolonger une forme linéaire positive en préservant la positivité. On va donc pouvoir se restreindre
à étudier les moyennes sur des sous-espaces plus concrets de $\mathrm{L}^\infty(\Gamma)$.

Plus précisément, toujours pour $\Gamma$ groupe localement compact et séparé, on introduit $L_0(\Gamma) := \sum_{g\in\Gamma} \Ima(\lambda(g) - \id)$ le sous-espace de 
$\mathrm{L}^\infty(\Gamma)$ engendré par les (classes de) fonctions de la forme $x\mapsto f(g\inv x) - f(x)$ pour $f\in\mathscr{L}^\infty(\Gamma)$. 
Il est alors clair qu'une forme linéaire sur $\mathrm{L}^\infty(\Gamma)$ est invariante à gauche \ssi sa restriction à $L_0(\Gamma)$ est nulle. 

\begin{lemma}
    $\indic\notin L_0(G)$
\end{lemma}

\begin{proof}
    Supposons d'abord que $\indic$ soit de la forme $\lambda(\gamma)(f) - f$ pour certains $\gamma\in\Gamma$ et 
    $f\in\mathrm{L}^\infty(\Gamma)$. On a alors $\forall n\in\N, f(\gamma^n) = f(1) + n$, donc $\norm{f(\gamma^n)}\xrightarrow[n\to+\infty]{}+\infty$
    ce qui contredit le fait que $f$ est borné. \\
    Pour le cas général, on suppose cette fois $\indic = \sum_{i\in I} (\lambda(\gamma_i)(f_i) - f_i)$ pour $I$ fini, $\gamma : I \to\Gamma$ et
    $f : I\to\mathrm{L}^\infty(\Gamma)$. On considère alors le groupe séparé et localement compact $\Gamma^I$ (dont $\gamma : i\mapsto \gamma_i$ est
    un élément), et la fonction : 
    \begin{equation*}
        \fundef{F}{\Gamma^I&\to&\C}{x&\mapsto&\frac{1}{\card{I}}\sum_{i\in I}f_i(x_i)}
    \end{equation*}
    $F$ est mesurable, et la majoration
    $\forall x\in\Gamma^I, \abs{f(x)}\le\frac1{\card{I}}\sum_i\norm{f_i}_\infty$ assure que $F\in\mathscr{L}^\infty(\Gamma^I)$. 
    Enfin, $\indic_{\Gamma^I} = \lambda(\gamma)(F) - F$, ce qui nous ramène au cas déjà traité. \\
    TODO classes de fonctions ?
\end{proof}

On peut alors considérer l'espace $L(G) := \C\cdot\indic \oplus L_0(G)$, qui est intéressant en ce qu'il permet de caractériser entièrement
les moyennes à gauche.

\paragraph{}


On peut maintenant donner notre premier exemple de groupe non-compact moyennable.

\begin{theorem}\label{Z_amenable}
    $\Z$ est moyennable.
\end{theorem}

On va pour cela utiliser la variation suivante sur le théorème de Hahn-Banach.

TODO

\begin{proof}[Démonstration du théorème \ref{Z_amenable}]
    Considérons le sous-espace vectoriel $C$ de $\mathrm{L}^\infty(\Z) = \ell^\infty(\Z)$ formé 
    des suites $u:\Z\to\R$ convergentes en $+\infty$. Posons alors 
    $\fundef{\varphi}{C&\to&\R}{u&\mapsto&\lim_{n\to+\infty}u_n}$.

    Il est clair que $\varphi$ est linéaire. De plus, pour $u\in C$, on a $\forall n, \abs{u_n}\le\norm{u_n}_\infty$
    donc en passant à la limite $\abs{\varphi(u)}\le\norm{u}_\infty$, donc $\varphi$ est continue. $\varphi$ est enfin clairement positive.
    Par le résultat ci-dessus (TODO ref) on obtient donc $\widetilde{\varphi}:$

    TODO
\end{proof}

\subsection{Condition de F\o{}lner}

\end{document}