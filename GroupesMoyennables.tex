\documentclass[a4paper,12pt]{article}
\usepackage[utf8]{inputenc}
\usepackage[french]{babel}
\usepackage{amssymb}
\usepackage{amsmath}
\usepackage{bbm}
\usepackage{amsthm}
\usepackage{a4wide}
\usepackage{mathrsfs}
\usepackage{stmaryrd}
\usepackage{mathtools}
\usepackage{graphicx}
\usepackage{hyperref}
\usepackage{faktor}
\usepackage{enumerate}
\usepackage{xcolor}
\usepackage[T1]{fontenc}

\newtheorem{theorem}{Théorème}[section]
\newtheorem{proposition}[theorem]{Proposition}
\newtheorem{definition}[theorem]{Définition}
\newtheorem{corollary}[theorem]{Corollaire}
\newtheorem{lemma}[theorem]{Lemme}
\newtheorem{remark}[theorem]{Remarque}

\newcommand{\R}{\mathbb{R}}
\newcommand{\N}{\mathbb{N}}
\newcommand{\Q}{\mathbb{Q}}
\newcommand{\Z}{\mathbb{Z}}
\newcommand{\C}{\mathbb{C}}
\newcommand{\K}{\mathbb{K}}
\newcommand{\F}{\mathcal{F}}
\newcommand{\G}{\mathcal{G}}
\newcommand{\U}{\mathcal{U}}
\newcommand{\Bor}{\mathcal{B}}
\newcommand{\norm}[1]{\left\Vert #1\right\Vert}
\newcommand{\abs}[1]{\left\vert#1\right\vert}
\newcommand{\card}[1]{\abs{#1}}
\newcommand{\ket}[1]{\left\langle #1 \right\rangle}
\newcommand{\halfilon}{{\frac\varepsilon2}}
\newcommand{\set}[1]{\left\{ #1 \right\}}
\newcommand{\indic}{\mathbbm{1}}
\newcommand{\integral}[4]{\int_{#1}^{#2} #3~\mathrm{d}#4}
\newcommand\fundef[3]{#1: \left\{\begin{array}{ccc}#2\\#3\end{array}\right.}
\newcommand\funlam[2]{\left\{\begin{array}{ccc}#1\\#2\end{array}\right.}
\newcommand{\tq}{\;\middle|\;}
\newcommand{\ssi}{si et seulement si }
\newcommand{\interior}[1]{\mathring{#1}}
\newcommand{\closure}[1]{\overline{#1}}
\newcommand{\transpose}[1]{\prescript{t}{}{#1}{}{}}
\newcommand{\inv}{^{-1}}
\newcommand{\compl}{^c}
\newcommand{\infi}{\bigwedge}
\newcommand{\supr}{\bigvee}
\newcommand{\comp}{\circ}
\newcommand{\nhds}{\mathcal{N}}
\renewcommand{\implies}{\Rightarrow}
\renewcommand{\iff}{\Leftrightarrow}
\newcommand{\blank}{{-}}
\newcommand{\invop}{\mathrm{inv}}
\newcommand{\parts}{\mathfrak{P}}
\newcommand{\finparts}{\mathfrak{P}_{\mathrm{fin}}}
\newcommand{\TODO}[1]{{\color{red}TODO :} #1}

\DeclareMathOperator{\sgn}{sgn}
\DeclareMathOperator{\Id}{id}
\DeclareMathOperator{\id}{id}
\DeclareMathOperator{\Mat}{Mat}
\DeclareMathOperator{\Vect}{Vect}
\DeclareMathOperator{\Ima}{Im}
\DeclareMathOperator{\solset}{Sol}
\DeclareMathOperator{\Sp}{Sp}

\begin{document}

\begin{titlepage}
\title{Groupes Moyennables}
\author{Anatole \textsc{Dedecker}}
\maketitle
\thispagestyle{empty}
\end{titlepage}

\tableofcontents
\thispagestyle{empty}

\clearpage

\pagenumbering{arabic}

\section*{Conventions et remarques préliminaires}

Dans ce mémoire, nous utiliserons le terme de \textit{mesure} pour désigner une mesure \textbf{finiment}-additive sur un 
\textit{espace mesurable}, c'est à dire un couple $(X, \mathcal{A})$ où $\mathcal{A}$ est une $\sigma$-algèbre. 
Lorsqu'une telle mesure est de plus $\sigma$-additive (ce qui est souvent inclus dans la définition de \og{}mesure\fg{}),
nous dirons qu'il s'agit d'une \textit{$\sigma$-mesure}. 

Nous admettrons qu'il est possible de définir une théorie de l'intégration pour toute mesure, les théorèmes de convergence
monotone et dominée n'étant bien sûr valables que dans le cas des $\sigma$-mesures. 

Si $\varphi:(X,\mathcal{A})\to(Y,\mathcal{B})$ est une application mesurable et $m$ est une mesure sur $(X, \mathcal{A})$, 
nous noterons $\varphi_*m$ la mesure image sur $(Y, \mathcal{B})$. C'est une $\sigma$-mesure si $m$ en est une, et intégrer 
une fonction $f$ selon $\varphi_*m$ revient exactement à intégrer $f\comp\varphi$ selon $m$. Enfin, c'est une construction
fonctorielle, au sens où l'on a $(\psi\comp\varphi)_*m = \psi_*\varphi_*m$ pour toute application mesurable 
$\psi : (Y, \mathcal{B})\to(Z, \mathcal{C})$. 

\paragraph{}

Conformément à la convention dans le monde anglophone, et pour éviter toute confusion, nous dirons qu'un espace topologique
$X$ est \emph{compact} s'il vérifie la propriété de Borel-Lebesgue, sans hypothèse de séparation. 
Nous dirons que $X$ est \emph{localement compact} si, pour tout $x\in X$, le filtre $\nhds_x$ des voisinages de $x$ admet
une \textbf{base} formée d'ensembles compacts. Si $X$ est séparé, on retrouve que $X$ est localement compact 
\ssi tout point admet \textbf{un} voisinage compact.

\paragraph{}
Si $G$ est un groupe et $g\in G$, nous noterons $\invop$ l'inversion et $\ell_g$, $r_g$ les applications de tranlations:
\begin{equation*}
    \fundef{\invop = (\blank)\inv}{G&\to& G}{x&\mapsto& x\inv}\text{;}\quad\fundef{\ell_g=(g\blank)}{G&\to& G}{x&\mapsto& gx}\text{;}\quad\fundef{r_g=(\blank g)}{G&\to& G}{x&\mapsto& xg}
\end{equation*}

Si le groupe $G$ est muni d'une $\sigma$-algèbre stable par les tranlations droite et gauche, ce qui sera notamment le cas
si $G$ est un groupe topologique muni de sa tribu borélienne $\Bor(G)$, nous dirons qu'une mesure $m$ sur $G$ est \textit{invariante
(par translations) à gauche (resp. à droite)} si $\forall g\in G, {\ell_g}_*m = m$ (resp. ${r_g}_*m = m$). Si les deux conditions
sont vérifiées, nous parlerons simplement de mesure \textit{invariante par translations}.

Une \textit{mesure de Haar à gauche (resp. à droite)} sur un groupe topologique $G$ séparé et localement compact est une mesure de
Radon sur $G$ invariante par tranlations à gauche (resp. à droite). Rappelons le théorème fondamental concernant les mesures de Haar, que nous 
utiliserons à de nombreuses reprises.

\begin{theorem}\label{theorem_Haar}
    Tout groupe topologique $G$ séparé et localement compact admet une mesure de Haar à gauche, unique à multiplication par un réel strictement 
    positif près. 
    
    Si $\mu$ est une telle mesure, tout ouvert $U$ non-vide de $G$ est de mesure strictement positive,
    et $\mu(G)<+\infty$ \ssi G est compact.

    Enfin, si $G$ est compact, les mesures de Haar à gauche sont exactement les mesures de Haar à droite.
\end{theorem}

\TODO{Définir $\lambda$, $\rho$ les actions régulières de $G$ sur les espaces de fonctions}

\TODO{Notations $\mathscr{L}^p$, $\mathrm{L}^p$, $B$ (fonctions mesurables bornées (pas "presque partout"), sans quotient)}

\section{Premières définitions}

Fixons $\Gamma$ un groupe topologique. La notion qui va nous intéresser dans ce mémoire est la suivante.

\begin{definition}
    On dit qu'un groupe topologique $\Gamma$ est \emph{moyennable} s'il admet une 
    mesure de probabilité borélienne invariante par translations.
\end{definition}

Notons tout de suite que l'invariance bilatère n'impose pas de restriction suplémentaire. 
\begin{proposition}\label{bilateral_of_left}
    Supposons qu'il existe une mesure de probabilité borélienne $m$ sur $\Gamma$, invariante par translations \emph{à gauche}.
    Alors $\Gamma$ est un groupe moyennable.
\end{proposition} 

\begin{proof}
    Il s'agit donc de construire une \emph{autre} mesure $n$ sur $\Gamma$ qui soit cette-fois invariante des deux côtés.
    Posons d'abord, pour $A\in\Bor(G)$, $\fundef{f_A}{\Gamma&\to& \Gamma}{g&\mapsto& m(Ag\inv)}$. Chaque $f_A$ est bornée par $1 = m(\Gamma)$, 
    donc intégrable pour la mesure de probabilité $\invop_*m$. Posons alors $n(A) := \integral{}{}{f_A}{(\invop_*m)}$. \\
    Notons que, pour $A$ et $B$ boréliens disjoints, ainsi que $g, x\in\Gamma$, on a :
    \begin{gather*}
        f_\Gamma = \indic \\
        f_{A\cup B} = f_A + f_B \\
        f_{g\inv A}(x) = m(g\inv Ax\inv) = ({\ell_g}_*m)(Ax\inv) = m(Ax\inv) = f_A(x) \\
        f_{Ag\inv}(x) = m(Ag\inv x\inv) = f_A(xg) = (f_A \comp r_g)(x)
    \end{gather*}
    En intégrant ces relations, on obtient bien :
    \begin{gather*}
        n(\Gamma) = (\invop_*m)(\Gamma) = 1 \\
        n(A\cup B) = n(A) + n(B) \\
        ({\ell_g}_*n)(A) = n(g\inv A) = n(A) \\
        ({r_g}_*n)(A) = \integral{}{}{f_A\comp r_g\comp\invop}{m} = \integral{}{}{f_A\comp\invop\comp \ell_g}{m} = n(A)
    \end{gather*}
    Ce qui conlut.
\end{proof}

Les premiers exemples de groupes moyennables sont les groupes compacts séparés (et en particulier les groupes finis discrets).
En effet, la mesure de Haar normalisée d'un tel groupe est une mesure de probabilité invariante par tranlation. Les phénomènes
plus intéressants vont donc se produire pour les groupes non-compacts, et il convient de noter que ces groupes ne peuvent pas avoir
de $\sigma$-mesure de probabilité invariante par translations d'après le théorème \ref{theorem_Haar}. C'est donc l'affaibilissement de la 
$\sigma$-additivité en additivité finie qui engendre la complexité. 

\paragraph{}

Avant d'aller plus loin, donnons aussi un premier exemple de groupe \emph{non}-moyennable.

\begin{theorem}\label{not_amenable_F2}
    Le groupe libre en deux générateurs $F_2$ (muni de la topologie discrète) n'est pas moyennable.
\end{theorem}

\begin{proof}
    Notons $a, b$ les deux générateurs. Pour $m$ mot réduit en $\set{a, b, a\inv, b\inv}$, notons 
    $S(m)$ l'ensemble des $g\in F_2$ dont l'écriture (unique) comme mot réduit
    commence par $m$. On a par exemple $ab\in S(a)$ mais $a a\inv b \notin S(a)$ car le mot réduit associé 
    à $a a\inv b$ est $b$.

    Remarquons que $a\inv S(a) = S(a\inv)\compl$. En effet, si $m$ est un mot réduit ne commençant pas par $a\inv$, 
    $am$ est un mot réduit commençant par $a$. Réciproquement si $am$ est un mot réduit alors $m$ est réduit et ne commence 
    pas par $a\inv$. \TODO{est-ce que c'est assez clair ?}

    On montre de même que $b\inv S(b) = S(b\inv)\compl$. Supposons alors qu'il existe une mesure de probabilité $m$ sur 
    $F_2$ invariante par translation. On a alors :
    \begin{gather*}
        m(S(a)) + m(S(a\inv)) = m(a\inv S(a)\ \amalg\ S(a\inv)) = 1 \\
        m(S(b)) + m(S(b\inv)) = m(b\inv S(b)\ \amalg\ S(b\inv)) = 1
    \end{gather*}
    Mais les ensembles $S(a)$, $S(a\inv)$, $S(b)$ et $S(b\inv)$, donc :
    \begin{equation*}
        1 = m(F_2) \ge m(S(a)) + m(S(a\inv)) + m(S(b)) + m(S(b\inv)) = 2
    \end{equation*} 
    D'où contradiction.
\end{proof}

\TODO{Mentionner importance historique ?}

\paragraph{}

Comme constaté dès la preuve de la proposition \ref{bilateral_of_left}, il va sans surprise être très souvent utile d'étudier la mesure $m$ à travers la théorie de l'intégration.
On dispose pour cela du résultat suivant.
\begin{proposition}\label{repr}
    Soit $X$ un espace mesurable. Pour toute mesure de probabilité $m$ sur $X$, on définit :
    \begin{equation*}
        \fundef{I_m}{B(X)&\to&\C}{f&\mapsto&\integral{}{}{f}{m}}
    \end{equation*}
    L'application $I:m\mapsto I_m$ est alors une bijection de l'ensemble $\mathrm{Proba}(X)$ des mesures de
    probabilités sur $X$ sur l'ensemble des formes linéaires positives sur $B(X)$ de valeur $1$ en $\indic$, la 
    réciproque étant donnée par $T\mapsto(A\mapsto T(\indic_A))$.
\end{proposition}

Commençons par prouver le lemme suivant, qui sera utile en lui-même.

\begin{lemma}\label{positive_iff_norm}
    Soit $X$ un espace mesurable et $\varphi:B(X)\to\C$ une forme linéaire. 
    Si $\varphi(\indic) = 1$, on a l'équivalence :
    \begin{equation*}
        \norm{\varphi} = 1 \iff \forall a \ge 0, \varphi(a) \geq 0
    \end{equation*}
\end{lemma}

\begin{proof}
    Supposons d'abord $\norm{\varphi} = 1$, et soit $a\ge 0$ de norme 1. On a donc, pour $x\in X$, $a(x), 1-a(x)\in[0,1]$,
    d'où enfin $\norm{\indic-a}\le 1$. Mais on a $1 = \varphi(a) + \varphi(\indic-a) \le \varphi(a) + \abs{\varphi(\indic-a)} \le \varphi(a) + \norm{\indic - a}$, d'où
    $\varphi(a)\ge 1 - \norm{\indic-a}\ge0$. \\
    Supposons maintenant $\varphi$ positif. Notons déjà que l'égalité $\varphi(\indic) = 1$ entraîne $\norm{\varphi}\ge1$. Soit donc $a\in B(X)$ quelconque,
    et notons que $-\norm{a}\indic\le a\le\norm{a}\indic$, de sorte que $-\norm{a}\le\varphi(a)\le\norm{a}$, ce qui conclut.
\end{proof}

\begin{proof}[Démonstration du théorème \ref{repr}]
    Toute fonction mesurable bornée étant intégrable par rapport à toute mesure de probabilité, il est clair que $I$ est bien définie. 
    Les propriétés élémentaires de l'intégration de Lebesgue, qui restent valables dans le cas de mesures finiment-additives,
    donnent immédiatement que chaque $I_m$ est bien une forme linéaire positive avec $I_m(\indic) = 1$, et qu'on a
    $I_m(\indic_A) = m(A)$ pour tout $A\subseteq X$ mesurable.

    Soit maintenant $T$ une forme linéaire positive sur $B(X)$, vérifiant $T(\indic) = 1$.
    Il faut alors montrer que $m : A\mapsto T(\indic_A)$ est bien une mesure de probabilité sur $X$,
    puis que $T = I_m$ pour cette mesure $m$. $m$ est bien à valeurs positive par positivité de $T$, et 
    de plus $m(X) = T(\indic) = 1$ et $m(\varnothing) = T(0) = 0$. Soient enfin $A, B\subseteq X$ deux ensembles
    mesurables disjoints, de sorte que $\indic_{A\cup B} = \indic_A + \indic_B$. On a alors bien 
    $m(A\cup B) = m(A) + m(B)$ ce qui assure l'additivité finie. 

    Pour finir, considérons le sous-espace vectoriel $S(X, m)$ de $B(X)$ des fonctions $m$-simples, c'est à dire des 
    fonctions $f:X\to\C$ mesurables, d'image finie, et vérifiant $\forall c\in\C^*, m(f\inv(x)) < +\infty$ (cette dernière 
    condition est automatiquement vérifiée dans notre cas d'une mesure de probabilité, mais nécessaire en général).
    Montrons que $S(X, m)$ est dense dans $B(X)$. Soit donc $f\in B(X)$, que nous supposons d'abord positive, et construisons 
    une suite $g:\N\to S(X, m)$ de la manière suivante :
    \begin{gather*}
        g_0 := 0 \\
        g_{n+1} := g_n + \frac12\norm{f - g_n}\indic_{\set{x\tq f(x) - g_n(x) \ge \frac12\norm{f - g_n}}}
    \end{gather*}
    Il est clair que chaque $g_n$ est une fonction simple positive et inférieure à $f$, et que la suite $g$ est croissante.
    De plus, pour tout $n\in\N$ et $x\in X$, on est dans l'un des cas suivants : 
    \begin{gather*}
        f(x)-g_{n+1}(x) = f(x)-g_n(x) \le \frac12\norm{f - g_n} \\
        f(x)-g_{n+1}(x) = f(x)-g_n(x)-\frac12\norm{f-g_n} \le \frac12\norm{f-g_n}
    \end{gather*}
    Il vient $\norm{f - g_{n+1}}\le\frac12\norm{f - g_n}$, d'où $\norm{f - g_n}\xrightarrow[n\to+\infty]{} 0$.
    Dans le cas général, il suffit alors de décomposer $f$ en combinaison linéaire de fonctions positives et d'appliquer 
    le résultat à chacune de ces fonctions. Cela montre donc la densité souhaitée. Or les formes linéaires $T$ et $I_m$
    coïncident sur les indicatrices, donc sur $S(X, m)$, et elles sont continues par le lemme \ref{positive_iff_norm}. 
    Par densité, elles sont donc égales, ce qui termine la preuve.
\end{proof}

\TODO{Commentaire sur lien avec construction de l'intégrale de Bochner ?}

Cela nous amène à donner la définition suivante.

\begin{definition}
    Soit $\Gamma$ un groupe localement compact et séparé. Une \emph{moyenne à gauche} (resp. \emph{à droite}) sur $\Gamma$ est une forme 
    linéaire positive $T : B(\Gamma)\to\C$ vérifiant $T(\indic) = 1$ et $\forall g\in\Gamma, T\comp\lambda_g = T$
    (resp. $T\comp\rho_g = T$). Une moyenne bilatère sera simplement appelée \emph{moyenne}.
\end{definition}

\begin{remark}
    Les moyennes à gauche sur $\Gamma$ sont exactement les morphismes de la représentation $(B(\Gamma), \lambda)$ de 
    $\Gamma$ vers la représentation triviale, avec la condition suppplémentaire que $T(\indic) = 1$.
\end{remark}

Remarquons que la bijection $I$ de la proposition \ref{repr} fait correspondre les mesures de probabilité 
invariantes aux moyennes sur $\Gamma$. En effet, si $m$ est invariante à gauche, alors pour tous 
$f\in B(\Gamma)$ et $g\in\Gamma$, on a 
$I_m(\lambda_g(f)) = \integral{}{}{f\comp\ell_{g\inv}}{m} = \integral{}{}{f}{m} = I_m(f)$. 

Réciproquement, si $T$ est une 
moyenne à gauche, alors la mesure $m$ associe vérifie, pour tous $A\in\Bor(\Gamma)$ et $g\in\Gamma$, 
$m(g\inv A) = T(\indic_{g\inv A}) = T(\indic_A \comp\ell_g) = T(\lambda_{g\inv}(\indic_A)) = T(\indic_A) = m(A)$.
En prenant en compte la proposition \ref{bilateral_of_left}, on vient donc de montrer le résultat suivant.
\begin{proposition}
    Un groupe localement compact séparé $\Gamma$ est moyennable \ssi il admet une moyenne (resp. une moyenne à gauche, resp. une manière à droite).
\end{proposition}

Dans la suite, nous supposons toujours que le groupe topologique $\Gamma$ est séparé et localement compact.
\paragraph{}

Sans grande surprise, nous allons maintenant vouloir appliquer différents outils de l'analyse fonctionelle à l'étude des moyennes.
Pour cela, nous allons utiliser à nouveau le lemme \ref{positive_iff_norm}, de manière beaucoup plus fine que dans la preuve 
du théorème \ref{repr} (où nous avions seulement besoin de la continuité des formes linéaires positives).

En effet, la condition $\norm\varphi = 1$ qui apparaît dans ce lemme a le bon goût d'être conservée lors du prolongement de l'application 
$\varphi$ par le théorème de Hahn-Banach, alors qu'il n'est pas clair du tout qu'on puisse prolonger une forme linéaire positive en préservant 
la positivité. On va donc pouvoir se restreindre à étudier les moyennes sur des sous-espaces plus concrets de $B(\Gamma)$.

Plus précisément, toujours pour $\Gamma$ groupe localement compact et séparé, on introduit $L_0(\Gamma) := \sum_{g\in\Gamma} \Ima(\lambda(g) - \id)$ le sous-espace de 
$B(\Gamma)$ engendré par les (classes de) fonctions de la forme $x\mapsto f(g\inv x) - f(x)$ pour $f\in B(\Gamma)$. 
Il est alors clair qu'une forme linéaire sur $B(\Gamma)$ est invariante à gauche \ssi sa restriction à $L_0(\Gamma)$ est nulle. 

\begin{lemma}
    $\indic\notin L_0(\Gamma)$
\end{lemma}

\begin{proof}
    Supposons d'abord que $\indic$ soit de la forme $\lambda(\gamma)(f) - f$ pour certains $\gamma\in\Gamma$ et 
    $f\in B(\Gamma)$. On a alors $\forall n\in\N, f(\gamma^n) = f(1) + n$, donc $\norm{f(\gamma^n)}\xrightarrow[n\to+\infty]{}+\infty$
    ce qui contredit le fait que $f$ est bornée. 
    %\TODO{Cet argument ne marche pas si l'égalité $\indic = \lambda(\gamma)(f) - f$ n'est 
    %vraie que presque partout !}

    Pour le cas général, on suppose cette fois $\indic = \sum_{i\in I} (\lambda(\gamma_i)(f_i) - f_i)$ pour $I$ fini, $\gamma : I \to\Gamma$ et
    $f : I\to B(\Gamma)$. On considère alors le groupe séparé et localement compact $\Gamma^I$ (dont $\gamma : i\mapsto \gamma_i$ est
    un élément), et la fonction : 
    \begin{equation*}
        \fundef{F}{\Gamma^I&\to&\C}{x&\mapsto&\frac{1}{\card{I}}\sum_{i\in I}f_i(x_i)}
    \end{equation*}
    $F$ est mesurable, et la majoration
    $\forall x\in\Gamma^I, \abs{f(x)}\le\frac1{\card{I}}\sum_i\norm{f_i}_\infty$ assure que $F\in B(\Gamma^I)$. 
    Enfin, $\indic_{\Gamma^I} = \lambda(\gamma)(F) - F$, ce qui nous ramène au cas déjà traité. 
\end{proof}

On peut alors considérer l'espace $L(\Gamma) := \C\cdot\indic \oplus L_0(\Gamma)$, qui est intéressant en ce qu'il permet de caractériser entièrement
les moyennes à gauche sur $\Gamma$, au sens du théorème suivant.

\begin{theorem}\label{left_mean_iff}
    Considérons l'application linéaire $\widetilde{T} : L(\Gamma)\to\C$ définie par $\widetilde{T}(\indic) = 1$
    et $\widetilde{T}_{|L_0(\Gamma)} = 0$. 
    
    Le groupe $\Gamma$ est moyennable \ssi l'application $\widetilde{T}$ est continue et de norme $1$.
    Si c'est le cas, les moyennes à gauche sur $\Gamma$ sont exactement les prolongements de $\widetilde{T}$ à 
    $\mathrm{L}^\infty(\Gamma)$ qui préservent la norme.
\end{theorem}

En remarquant que $\forall c\in\C, \forall f\in L_0(\Gamma), \widetilde{T}(c + f) = c$, et qu'on a toujours 
$\norm{\widetilde{T}}\ge1$ par $\widetilde{T}(\indic) = 1$, on obtient le critère de moyennabilité suivant:

\begin{corollary}\label{amenable_iff_L0}
    $\Gamma$ est moyennable \ssi $\forall c\in\C, \forall f\in L_0(\Gamma), \abs{c}\le\norm{c + f}_\infty$.
\end{corollary}

\begin{proof}[Démonstration du théorème \ref{left_mean_iff}]
    Supposons d'abord $\Gamma$ moyennable, et soit $T$ une moyenne à gauche sur $\Gamma$.
    On sait déjà que $T$ prolonge $\widetilde{T}$, puisque $T$ est nulle sur $L_0(\Gamma)$ et 
    $T(\indic)=1$. On a donc $\norm{\widetilde{T}}\le\norm{T}=1$ par restriction, et en fait $\norm{\widetilde{T}} = 1$
    puique $\widetilde{T}(\indic)=1$. 

    Supposons maintenant $\norm{\widetilde{T}}=1$. Par le théorème de Hahn-Banach, $\widetilde{T}$ se prolonge 
    en $T:B(\Gamma)\to\C$ linéaire continue de même norme. Comme de plus $T(\indic)=1$ et 
    $T_{|L_0(\Gamma)} = 0$, $T$ est une moyenne à gauche sur $\Gamma$. En fait, cet argument montre même 
    que \emph{tout} prolongement de norme $1$ de $\widetilde{T}$ \footnote{et pas seulement le prolongement fourni par Hahn-Banach} 
    est une moyenne à gauche, ce qui termine la preuve.
\end{proof}

Illustrons ce critère sur le cas du groupe libre $F_2$. Dans ce cas, on peut montrer que $\norm{\widetilde{T}}\ge3$.
Reprenons pour cela les notations de la preuve du théorème \ref{not_amenable_F2}, et posons $f := (\lambda(a\inv)-\id)(\indic_{S(a)})$ et 
$g:=(\lambda(b\inv) - \id)(\indic_{S(b)})$, qui sont deux éléments de $L_0(F_2)$. L'égalité $a\inv S(a) = S(a\inv)\compl$ donne que 
$f = \indic_{a\inv S(a)} - \indic_{S(a)} = \indic_{\set{1}\cup S(b)\cup S(b\inv)}$, et de même 
$g = \indic_{\set{1}\cup S(a)\cup S(a\inv)}$, de sorte que $f+g=\indic + \delta_1$. Mais alors 
$-\frac23(\indic+\delta_1)\in L_0(F_2)$, donc $\widetilde{T}(\indic - \frac23(\indic+\delta_1)) = 1$, 
et d'autre part $\norm{\indic - \frac23(\indic+\delta_1)}_\infty=\frac13$. 
On a donc bien $\norm{\widetilde{T}}\geq3$.

\paragraph{}

Donnons maintenant, toujours à l'aide du critère \ref{amenable_iff_L0}, notre premier exemple de groupe moyennable non-compact.

\begin{theorem}\label{Z_amenable}
    $\Z$ est moyennable.
\end{theorem}

\begin{proof}
    Commençons par simplifier un peu notre description de $L_0(\Z)$. Un simple argument de somme
    télescopique montre que, pour $n>0$ et $u\in B(\Z)=\ell^\infty(\Z)$:
    \begin{equation*}
        (\lambda(n)-\id)(u) = \left(\sum_{1\le i\le n} \lambda(i+1) - \lambda(i)\right)(u) = (\lambda(1)-\id)\left(\sum_{1\le i\le n} \lambda(i)(u)\right)
    \end{equation*}

    Comme de plus $\lambda(0)-\id = 0$ et $\lambda(-n)-\id = -(\lambda(n)-\id)\comp\lambda(-n)$,
    on en déduit que $\Ima(\lambda(n)-\id)\subseteq\Ima(\lambda(1)-\id)$ pour tout $n\in\Z$, et donc $L_0(\Z) = \Ima(\lambda(1)-\id)$. \\

    Soient maintenant $c\in\C$ et $v\in L_0(\Z)$. On peut donc écrire $v = \lambda(1)(u)-u$ pour un certain $u\in\ell^\infty(\Z)$. 
    On veut montrer $\abs{c}\le\norm{c + \lambda(1)(u) - u}_\infty =: M$. 

    Par définition, on a $\forall n\in\Z, -M\le c+u_{n-1}-u_n\le M$. En moyennant ces inégalités pour 
    $n\in\rrbracket N-k, N\rrbracket$, on obtient, encore par un argument de somme télescopique :
    \begin{equation}\label{Z_amenable_eq1}
        \forall N\in\Z, \forall k\in\N,\quad -M\le c-\frac{u_N - u_{N-k}}{k} \le M
    \end{equation}

    Mais $u$ est bornée, donc il existe une suite strictement croissante $\varphi:\N\to\N$ telle que $u\comp\varphi$
    soit convergente. En particulier $\frac{\abs{u_{\varphi(n+1)}-u_{\varphi(n)}}}{\varphi(n+1)-\varphi(n)}\le\abs{u_{\varphi(n+1)}-u_{\varphi(n)}}\xrightarrow[n\to+\infty]{}0$, 
    donc $c-\frac{u_{\varphi(n+1)}-u_{\varphi(n)}}{\varphi(n+1)-\varphi(n)}\xrightarrow[n\to+\infty]{}c$. En passant à la limite dans l'inégalité (\ref{Z_amenable_eq1}),
    on obtient donc $-M\le c\le M$, ce qui conclut.
\end{proof}

Présentée ainsi, la preuve précédente peut sembler très spécifique à $\Z$ et peu généralisable. Pourtant, l'idée clé 
est simplement de considérer une \og{}moyenne\fg{} d'inégalités de la forme $-M\le c + u(\gamma\inv x) - u(x)\le M$, 
ce qui peut tout à fait se généraliser à un groupe discret quelconque. 

Les obstacles sont donc la forme spécifique de $L_0(\Z)$ et le recours à une suite extraite pour forcer la convergence des $u_{n+1}-u_n$. Nous allons voir 
qu'il est possible de les contourner.

\begin{proof}[Deuxième démonstration du théorème \ref{Z_amenable}]
    Soient $c\in\C$ et $v\in L_0(\Z)$, que l'on peut bien sûr décomposer en $v = \sum_{i\in I} (\lambda(n_i)(f_i) - f_i)$ pour certains $I$ fini, $n : I \to\Z$ et
    $f : I\to\ell^\infty(\Z)$. Posons toujours $M := \norm{c + v}_\infty$.

    Par définition, on a $\forall k\in\Z$:
    \begin{equation*}
        -M \le c + \sum_i (f_i(k-n_i) - f_i(k)) \le M
    \end{equation*}
    En moyennant ces inégalités pour $k\in F_N := \llbracket-N,N\rrbracket$, on obient $\forall N\in\N$:
    \begin{equation}\label{Z_amenable_eq2}
        -M \le c + \sum_i \frac1{2N+1} \sum_{k=-N}^N (f_i(k-n_i) - f_i(k)) \le M
    \end{equation}

    Étudions donc, pour $i$ fixé, les moyennes de la forme $\frac1{2N+1} \sum_{k=-N}^N (f_i(k-n_i) - f_i(k))$. On a :
    \begin{align*}
        \abs{\sum_{k=-N}^N (f_i(k-n_i) - f_i(k))} &= \abs{\sum_{k\in F_N - n_i} f_i(k) - \sum_{k\in F_N}f_i(k)} \\
            &= \abs{\sum_{\substack{k\in F_N - n_i \\ k\notin F_N}} f_i(k) -
            \sum_{\substack{k\in F_N \\ k\notin F_N-n_i}} f_i(k)} \\
            &\le \sum_{k\in (F_N-n_i)\triangle F_N} \norm{f_i}_\infty \\
            &=\norm{f_i}_\infty \cdot \card{(F_N-n_i)\triangle F_N}
    \end{align*}
    Or, pour $N>n_i$, l'ensemble $(F_N-n_i)\triangle F_N = \llbracket -N-n_i, -N\llbracket\ \amalg\ \rrbracket N-n_i, N\rrbracket$ est de cardinal $2n_i$ constant.
    On a donc:
    \begin{gather*}
        \frac{\card{(F_N-n_i)\triangle F_N}}{2N+1} \xrightarrow[N\to+\infty]{} 0 \\
        \frac1{2N+1} \sum_{k=-N}^N (f_i(k-n_i) - f_i(k)) \xrightarrow[N\to+\infty]{} 0
    \end{gather*}

    En passant à la limite dans l'inégalité (\ref{Z_amenable_eq2}), on obtient donc $-M\le c\le M$, ce qui conclut.
\end{proof}

Notons que, dans cette deuxième démonstration, l'hypothèse $\Gamma=\Z$ n'a été utilisée que pour établir l'existence d'une suite $F:\N\to\finparts^*(\Z)$ 
    \footnote{On note $\parts(X)$ (resp. $\parts^*(X)$) l'ensemble des parties (resp. parties non-vides) de $X$, et $\finparts(X)$ (resp $\finparts^*(X)$) 
    l'ensemble des parties finies (resp. parties finies non-vides) de $X$.}
de parties finies non-vides de $\Z$,
telle que $\forall n\in\Z, \frac{\card{(F_N-n)\triangle F_N}}{\card{F_N}} \xrightarrow[N\to+\infty]{} 0$. 

Dans le cas général, la mesure de Haar $\mu$ va remplacer le cardinal, et on va donc s'intéresser aux réels $\frac{\mu(\gamma F \triangle F)}{\mu(F)}$ 
pour $F$ partie compacte de $\Gamma$ et $\gamma\in\Gamma$, et à leur comportement asymptotique
lorsque $F$ est grand. C'est l'objet de la partie suivante.

\section{Condition de F\o{}lner}

Dans cette section, $\Gamma$ est un groupe topologique séparé et localement compact, et on fixe 
$\mu$ une mesure de Haar à gauche sur $\Gamma$.

On s'intéresse à l'\emph{application de F\o{}lner} $\mathfrak{f} : \mathcal{K}_+(\Gamma) \to \mathcal{F}(\Gamma, \R_+)$, définie sur l'ensemble 
$\mathcal{K}_+(\Gamma)$
\footnote{On note $\mathcal{K}(X)$ (resp. $\mathcal{K}^*(X)$, resp. $\mathcal{K}_+(X)$) l'ensemble des parties compactes 
(resp. compactes non-vides, resp. compactes d'intérieur non-vide) d'un espace topologique $X$. }
des parties compactes de $\Gamma$ d'intérieur non vide par $\mathfrak{f}(K)(\gamma) = \frac{\mu(\gamma K \triangle K)}{\mu(K)}$.
C'est une application bien définie car tout élément de $\mathcal{K}_+(\Gamma)$ est de mesure finie non-nulle,
et il est clair qu'elle ne dépend pas de la normalisation choisie pour la mesure de Haar.

Un filtre $\mathscr{F}$ sur $\mathcal{K}_+(\Gamma)$ est dit \emph{de F\o{}lner} si le filtre $\mathfrak{f}_*\mathscr{F}$ image directe 
de $\mathscr{F}$ par l'application de F\o{}lner converge vers $0$ pour la topologie de la convergence 
%compacte 
simple
sur $\mathcal{F}(\Gamma, \R_+)$.
Une suite $K:\N\to\mathcal{K}_+(\Gamma)$ est \emph{de F\o{}lner} si le filtre $K_*\nhds_{+\infty}^\N$, image directe par $K$ du filtre des
voisinages de l'infini dans $\N$, est de F\o{}lner. Comme $\mathfrak{f}_*K_*\nhds_{+\infty}^\N = (\mathfrak{f}\comp K)_*\nhds_{+\infty}^\N$,
une suite $K:\N\to\mathcal{K}_+(\Gamma)$ est donc de F\o{}lner \ssi la suite de fonctions
$n\mapsto\left(\gamma\mapsto\frac{\mu(\gamma K_n\triangle K_n)}{\mu(K_n)}\right)$ converge vers $0$
%uniformément sur tout compact. 
simplement.

Notons que si $\Gamma$ est discret, 
%la topologie de la convergence compacte coïncide avec celle de la convergence simple,
les parties compactes d'intérieur non vide sont exactement les parties finies non-vides,
et la mesure de Haar s'identifie (à un scalaire près) à la mesure de comptage. Dans ce cas, un filtre $\mathscr{F}$
est donc de F\o{}lner \ssi, pour tout $\gamma\in\Gamma$, la fonction $F\mapsto \frac{\card{\gamma F\triangle F}}{\card{F}}$ 
converge vers $0$ selon le filtre $\mathscr{F}$. De même, une suite $F:\N\to\finparts^*(\Gamma)$
est de F\o{}lner \ssi pour tout $\gamma\in\Gamma$, la suite $n\mapsto\frac{\card{\gamma F_n\triangle F_n}}{\card{F_n}}$ converge 
vers $0$.

Avant d'aller plus loin, donnons un critère plus simple pour l'existence d'un filtre de F\o{}lner sur $\Gamma$.

\begin{lemma}
    Le groupe séparé et localement compact $\Gamma$ admet un filtre de F\o{}lner \ssi il satisfait la 
    \emph{condition de F\o{}lner} :
    \begin{equation}\label{Folner_cond}
        \forall\varepsilon>0, \forall \gamma\in\Gamma, \exists K\in\mathcal{K}_+(\Gamma), 
        \frac{\mu(\gamma K\triangle K)}{\mu(K)}<\varepsilon
    \end{equation}
%    \begin{equation}\label{Folner_cond}
%        \forall\varepsilon>0, \forall A\in\mathcal{K}(\Gamma), \exists K\in\mathcal{K}_+(\Gamma), \forall\gamma\in A, 
%        \frac{\mu(\gamma K\triangle K)}{\mu(K)}<\varepsilon
%    \end{equation}

    Si de plus $\Gamma$ est dénombrable, cette condition est équivalent à l'existence d'une suite de F\o{}lner.
\end{lemma}

\begin{proof}
    \TODO{}
\end{proof}

\TODO{Quelle est la bonne topologie à mettre sur $\mathcal{F}(\Gamma, \R_+)$ pour la définition de filtre de F\o{}lner : convergence simple ou compacte ?
    En tout cas la convergence simple suffit à faire marcher le théorème suivant.}

Comme anoncé, on peut alors généraliser le théorème \ref{Z_amenable} à tout groupe muni d'un filtre de F\o{}lner,
en adaptant simplement la deuxième preuve de ce théorème.

\begin{theorem}\label{amenable_of_Folner}
    Si $\Gamma$ admet un filtre de F\o{}lner, alors $\Gamma$ est moyennable.
\end{theorem}

\begin{proof}
    On utilise toujours le critère \ref{amenable_iff_L0}. Soient donc $c\in\C$ et $v\in L_0(\Gamma)$, 
    que l'on écrit encore sous la forme $v = \sum_{i\in I} (\lambda(\gamma_i)(f_i) - f_i)$ pour certains $I$ fini, $\gamma : I \to\Gamma$ et
    $f : I\to B(\Gamma)$. Posons aussi $M := \norm{c + v}_\infty$.

    Par définition, on a $\forall x\in\Gamma$:
    \begin{equation*}
        -M \le c + \sum_i (f_i(\gamma_i\inv x) - f_i(x)) \le M
    \end{equation*}
    En moyennant ces inégalités sur un $K\in\mathcal{K}_+(\Gamma)$ quelconque, on obtient:
    \begin{equation}\label{amenable_of_Folner_eq1}
        -M \le c + \sum_i \frac1{\mu(K)} \integral{K}{}{(f_i(\gamma_i\inv x) - f_i(x))}{\mu(x)} \le M
    \end{equation}

    Or, pour $i$ fixé, l'invariance par translation de $\mu$ donne :
    \begin{align*}
        \abs{\integral{K}{}{(f_i(\gamma_i\inv x) - f_i(x))}{\mu(x)}} 
            &= \abs{\integral{\gamma_i\inv K}{}{f_i}{\mu} - \integral{K}{}{f_i}{\mu}} \\
            &= \abs{\integral{\gamma_i\inv K\setminus K}{}{f_i}{\mu} - \integral{K\setminus\gamma_i\inv K}{}{f_i}{\mu}} \\
            &\le \integral{\gamma_i\inv K\triangle K}{}{\norm{f_i}_\infty}{\mu} \\
            &=\norm{f_i}_\infty \cdot \mu(\gamma_i\inv K\triangle K)
    \end{align*}
    
    Soit finalement $\mathscr{F}$ un filtre de F\o{}lner pour $\Gamma$. L'estimation précédente assure alors que
    la fonction $K\mapsto\frac1{\mu(K)} \integral{K}{}{(f_i(\gamma_i\inv x) - f_i(x))}{\mu(x)}$ converge vers $0$
    selon $\mathscr{F}$, et ce pour chaque $i\in I$. Il suffit enfin de prendre la limite selon $\mathscr{F}$ des inégalités \ref{amenable_of_Folner_eq1}
    pour obtenir $-M\le c\le M$. 
\end{proof}

\TODO{Faire aussi la construction par ultrafiltre}

\TODO{Montrer la réciproque}

\end{document}