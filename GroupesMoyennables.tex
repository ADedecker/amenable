\documentclass[a4paper,12pt]{article}
\usepackage{amssymb}
\usepackage{amsmath}
\usepackage{amsthm}
\usepackage[utf8]{inputenc}
\usepackage{a4wide}
\usepackage{mathrsfs}
\usepackage{stmaryrd}
\usepackage{mathtools}
\usepackage[french]{babel}
\usepackage{graphicx}
\usepackage{hyperref}
\usepackage{faktor}
\usepackage{enumerate}
\usepackage[T1]{fontenc}

\newtheorem{thm}{Théorème}[subsection]
\newtheorem{prop}[thm]{Proposition}
\newtheorem{defn}[thm]{Définition}
\newtheorem{corr}[thm]{Corollaire}
\newtheorem{lem}[thm]{Lemme}
\addto\captionsenglish{\def\proofname{Preuve}}

\newcommand{\R}{\mathbb{R}}
\newcommand{\N}{\mathbb{N}}
\newcommand{\Q}{\mathbb{Q}}
\newcommand{\Z}{\mathbb{Z}}
\newcommand{\C}{\mathbb{C}}
\newcommand{\K}{\mathbb{K}}
\newcommand{\F}{\mathcal{F}}
\newcommand{\G}{\mathcal{G}}
\newcommand{\U}{\mathcal{U}}
\newcommand{\norm}[1]{\left\Vert #1\right\Vert}
\newcommand{\abs}[1]{\left\vert#1\right\vert}
\newcommand{\ket}[1]{\left\langle #1 \right\rangle}
\newcommand{\halfilon}{{\frac\varepsilon2}}
\newcommand{\set}[1]{\left\{ #1 \right\}}
\newcommand{\indic}{\mathds{1}}
\newcommand\fundef[3]{#1: \left\{\begin{array}{ccc}#2\\#3\end{array}\right.}
\newcommand\funlam[2]{\left\{\begin{array}{ccc}#1\\#2\end{array}\right.}
\newcommand{\tq}{\;\middle|\;}
\newcommand{\interior}[1]{\mathring{#1}}
\newcommand{\closure}[1]{\overline{#1}}
\newcommand{\transpose}[1]{\prescript{t}{}{#1}{}{}}
\newcommand{\inv}{^{-1}}
\newcommand{\infi}{\bigwedge}
\newcommand{\supr}{\bigvee}
\newcommand{\comp}{\circ}
\newcommand{\nhds}{\mathcal{N}}
\renewcommand{\implies}{\Rightarrow}
\renewcommand{\iff}{\Leftrightarrow}
\newcommand{\blank}{{-}}


\DeclareMathOperator{\sgn}{sgn}
\DeclareMathOperator{\Id}{Id}
\DeclareMathOperator{\Mat}{Mat}
\DeclareMathOperator{\Vect}{Vect}
\DeclareMathOperator{\Ima}{Im}
\DeclareMathOperator{\solset}{Sol}
\DeclareMathOperator{\Sp}{Sp}

\begin{document}

\selectlanguage{french}

\begin{titlepage}
\title{Groupes Moyennables}
\author{Anatole \textsc{Dedecker}}
\maketitle
\thispagestyle{empty}
\end{titlepage}

\tableofcontents
\thispagestyle{empty}
\selectlanguage{english}

\clearpage

\pagenumbering{arabic}

\section*{Conventions et remarques préliminaires}

Dans ce mémoire, nous utiliserons le terme de \textit{mesure} pour désigner une mesure \textbf{finiment}-additive sur un 
\textit{espace mesurable}, c'est à dire un couple $(X, \mathcal{A})$ où $\mathcal{A}$ est une $\sigma$-algèbre. 
Lorsqu'une telle mesure est de plus $\sigma$-additive (ce qui est souvent inclus dans la définition de \og{}mesure\fg{}),
nous dirons qu'il s'agit d'une \textit{$\sigma$-mesure}. 

Nous admettrons qu'il est possible de définir une théorie de l'intégration pour toute mesure, les théorèmes de convergence
monotone et dominée n'étant bien sûr valables que dans le cas des $\sigma$-mesures. 

Si $\varphi:(X,\mathcal{A})\to(Y,\mathcal{B})$ est une application mesurable et $m$ est une mesure sur $(X, \mathcal{A})$, 
nous noterons $\varphi_*m$ la mesure image sur $(Y, \mathcal{B})$. C'est une $\sigma$-mesure si $m$ en est une, et intégrer 
une fonction $f$ selon $\varphi_*m$ revient exactement à intégrer $f\comp\varphi$ selon $m$. Enfin, c'est une construction
fonctorielle, au sens où l'on a $(\psi\comp\varphi)_*m = \psi_*\varphi_*m$ pour toute application mesurable 
$\psi : (Y, \mathcal{B})\to(Z, \mathcal{C})$. 

\paragraph{}
Si $G$ est un groupe et $g\in G$, nous noterons $l_g$ et $r_g$ les applications de tranlations:
\begin{align*}
    \fundef{l_g=(g\blank)}{G\to G}{x\mapsto gx}\quad\text{ et }\quad\fundef{r_g=(\blank g)}{G\to G}{x\mapsto xg}
\end{align*}

Si le groupe $G$ est muni d'une $\sigma$-algèbre stable par les tranlations droite et gauche, ce qui sera notamment le cas
si $G$ est un groupe topologique muni de sa tribu borélienne, nous dirons qu'une mesure $m$ sur $G$ est \textit{invariante
(par translations) à gauche (resp. à droite)} si $\forall g\in G, (l_g)_*m = m$ (resp. $(r_g)_*m = m$). Si les deux conditions
sont vérifiées, nous parlerons simplement de mesure \textit{invariante par translations}.

Une \textit{mesure de Haar à gauche (resp. à droite)} sur un groupe topologique $G$ séparé et localement compact est une mesure de
Radon sur $G$ invariante par tranlations à gauche (resp. à droite).

TODO rappeler théorème

\section{Premières définitions}

\end{document}