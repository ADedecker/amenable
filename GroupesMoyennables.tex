\documentclass[a4paper,12pt]{article}
\usepackage[utf8]{inputenc}
\usepackage[french]{babel}
\usepackage{amssymb}
\usepackage{amsmath}
\usepackage{bbm}
\usepackage{amsthm}
\usepackage{a4wide}
\usepackage{mathrsfs}
\usepackage{stmaryrd}
\usepackage{mathtools}
\usepackage{graphicx}
\usepackage{hyperref}
\usepackage{faktor}
\usepackage{enumerate}
\usepackage{xcolor}
\usepackage{lipsum}
\usepackage{tikz, tkz-tab, tikz-cd}
\usepackage[T1]{fontenc}

\newtheorem{theorem}{Théorème}[section]
\newtheorem{proposition}[theorem]{Proposition}
\newtheorem{definition}[theorem]{Définition}
\newtheorem{corollary}[theorem]{Corollaire}
\newtheorem{lemma}[theorem]{Lemme}
\newtheorem{remark}[theorem]{Remarque}

\renewcommand{\i}{\mathrm{i}}
\newcommand{\R}{\mathbb{R}}
\newcommand{\N}{\mathbb{N}}
\newcommand{\Q}{\mathbb{Q}}
\newcommand{\Z}{\mathbb{Z}}
\newcommand{\C}{\mathbb{C}}
\newcommand{\K}{\mathbb{K}}
\newcommand{\F}{\mathcal{F}}
\newcommand{\G}{\mathcal{G}}
\newcommand{\U}{\mathcal{U}}
\newcommand{\ev}{\mathrm{ev}}
\newcommand{\Bor}{\mathcal{B}}
\newcommand{\norm}[1]{\left\Vert #1\right\Vert}
\newcommand{\abs}[1]{\left\vert#1\right\vert}
\newcommand{\card}[1]{\abs{#1}}
\newcommand{\ket}[1]{\left\langle #1 \right\rangle}
\newcommand{\floor}[1]{\left\lfloor #1 \right\rfloor}
\newcommand{\halfilon}{{\frac\varepsilon2}}
\newcommand{\set}[1]{\left\{ #1 \right\}}
\newcommand{\indic}{\mathbbm{1}}
\newcommand{\integral}[4]{\int_{#1}^{#2} #3~\mathrm{d}#4}
\newcommand\fundef[3]{#1: \left\{\begin{array}{ccc}#2\\#3\end{array}\right.}
\newcommand\funlam[2]{\left\{\begin{array}{ccc}#1\\#2\end{array}\right.}
\newcommand{\tq}{\;\middle|\;}
\newcommand{\ssi}{si et seulement si }
\newcommand{\interior}[1]{\mathring{#1}}
\newcommand{\closure}[1]{\overline{#1}}
\newcommand{\transpose}[1]{\prescript{t}{}{#1}{}{}}
\newcommand{\inv}{^{-1}}
\newcommand{\compl}{^c}
\newcommand{\infi}{\bigwedge}
\newcommand{\supr}{\bigvee}
\newcommand{\comp}{\circ}
\newcommand{\nhds}{\mathcal{N}}
\renewcommand{\implies}{\Rightarrow}
\renewcommand{\iff}{\Leftrightarrow}
\newcommand{\blank}{{-}}
\newcommand{\invop}{\mathrm{inv}}
\newcommand{\divop}{\mathrm{div}}
\newcommand{\parts}{\mathfrak{P}}
\newcommand{\finparts}{\mathfrak{P}_{\mathrm{fin}}}
\newcommand{\TODO}[1]{{\color{red}TODO :} #1}
\newcommand{\wle}{\preccurlyeq}

\DeclareMathOperator{\sgn}{sgn}
\DeclareMathOperator{\ess}{ess}
\DeclareMathOperator{\Id}{id}
\DeclareMathOperator{\Supp}{supp}
\DeclareMathOperator{\id}{id}
\DeclareMathOperator{\Mat}{Mat}
\DeclareMathOperator{\Vect}{Vect}
\DeclareMathOperator{\Ima}{im}
\DeclareMathOperator{\solset}{Sol}
\DeclareMathOperator{\Sp}{Sp}

\begin{document}

\begin{titlepage}
\title{Groupes Moyennables}
\author{Anatole \textsc{Dedecker}}
\maketitle
\thispagestyle{empty}
\end{titlepage}

\tableofcontents
\thispagestyle{empty}

\clearpage

\pagenumbering{arabic}

\section*{Conventions et remarques préliminaires}

\TODO{Réorganiser cette section, déplacer certaines choses en annexe}

Par convention, les produits scalaires sur le corps $\C$ sont supposés linéaires en le premier argument.

Conformément à la convention dans le monde anglophone, et pour éviter toute confusion, nous dirons qu'un espace topologique
$X$ est \emph{compact} s'il vérifie la propriété de Borel-Lebesgue, sans hypothèse de séparation. 
Nous dirons que $X$ est \emph{localement compact} si, pour tout $x\in X$, le filtre $\nhds_x$ des voisinages de $x$ admet
une \emph{base} formée d'ensembles compacts. Si $X$ est séparé, on retrouve que $X$ est localement compact 
\ssi tout point admet \emph{un} voisinage compact. 

Pour $G$ un groupe et $g\in G$, nous noterons $\invop$ l'inversion et $\ell_g$, $r_g$ les applications de translation :
\begin{equation*}
    \fundef{\invop = (\blank)\inv}{G&\to& G}{x&\mapsto& x\inv}\text{;}\quad\fundef{\ell_g=(g\blank)}{G&\to& G}{x&\mapsto& gx}\text{;}\quad\fundef{r_g=(\blank g)}{G&\to& G}{x&\mapsto& xg}
\end{equation*}

Si le groupe $G$ est muni d'une $\sigma$-algèbre stable par les tranlations droite et gauche, ce qui sera notamment le cas
si $G$ est un groupe topologique muni de sa tribu borélienne $\Bor(G)$, nous dirons qu'une mesure $m$ sur $G$ est \textit{invariante
(par translations) à gauche (resp. à droite)} si $\forall g\in G, {\ell_g}_*m = m$ (resp. ${r_g}_*m = m$). Si les deux conditions
sont vérifiées, nous parlerons simplement de mesure \textit{invariante par translations}.

Une \textit{mesure de Haar à gauche (resp. à droite)} sur un groupe topologique $G$ séparé et localement compact est une mesure de
Radon \emph{non nulle} sur $G$ invariante par tranlations à gauche (resp. à droite). Rappelons le théorème fondamental concernant les mesures de Haar, que nous 
utiliserons à de nombreuses reprises.

\TODO{Repréciser définition de mesure de Radon}
\TODO{Absolue continuité des mesures de Haar entre elles (via caractère modulaire)}
\TODO{Définir $\lambda$, $\rho$ les actions régulières de $G$ sur les espaces de fonctions}
\TODO{Notations $\mathscr{L}^p$, $\mathrm{L}^p$}

\section{Premières définitions}

Fixons $\Gamma$ un groupe topologique séparé et localement compact. Sauf mention explicite du contraire, $\mu$ désigne une mesure de Haar arbitraire
sur $\Gamma$.

La notion qui va nous intéresser dans ce mémoire est la suivante.

\begin{definition}\label{amenable_def}
    Le groupe séparé et localement compact $\Gamma$ est \emph{moyennable} s'il existe une forme linéaire positive $m : \mathrm{L}^\infty(G)\to\C$
    vérifiant $m(1) = 1$ et $\forall g\in\Gamma, m\comp\lambda_g = m = m\comp\rho_g$.
\end{definition}

Plus généralement, nous appellerons \emph{moyenne} sur un espace mesuré $(X,\mathcal{A},\mu)$ toute forme 
linéaire positive $m : \mathrm{L}^\infty(X, \mu)\to\C$ vérifiant $m(1) = 1$. Si de plus $X$ est un groupe 
et si $\mathcal{A}$ est stable par translations à gauche (resp. à droite), nous dirons qu'une moyenne est \emph{invariante à gauche}
(resp. \emph{à droite}) si $\forall x\in X, m\comp\lambda(x) = m$ (resp. $m\comp\rho(x) = m$). Si ces deux conditions 
sont vérifiées, nous dirons simplement que $T$ est \emph{invariante par translations} ou simplement \emph{invariante}.

Un groupe topologique séparé et localement compact $\Gamma$ est donc moyennable \ssi $(\Gamma, \Bor(\Gamma), \mu)$ admet une 
moyenne invariante, pour toute mesure de Haar $\mu$ sur $\Gamma$, le choix n'ayant aucune importance car toutes les mesures 
de Haar sont absolument continues les unes par rapport aux autres, et donc définissent le même espace $\mathrm{L}^\infty(\Gamma)$.

Notons tout de suite que l'invariance bilatère n'impose pas de restriction suplémentaire. 
\begin{proposition}\label{bilateral_of_left}
    Supposons qu'il existe une moyenne $m$ sur $(\Gamma, \Bor(\Gamma), \mu)$, invariante par translations \emph{à gauche}.
    Alors $\Gamma$ est un groupe moyennable.
\end{proposition} 

Commençons par prouver le lemme suivant, qui sera utile en lui-même.

\begin{lemma}\label{positive_iff_norm}
    Soit $(X, \mathcal{A}, \mu)$ un espace mesuré et $\varphi:\mathrm{L}^\infty(X,\mu)\to\C$ une forme linéaire \emph{non nécessairement continue}. 
    On a l'équivalence :
    \begin{equation*}
        \norm{\varphi} = \varphi(1) \iff \forall a \ge 0, \varphi(a) \geq 0
    \end{equation*}
\end{lemma}

\begin{proof}
    Supposons d'abord $\norm{\varphi} = \varphi(1)$, et soit $a\ge 0$ de norme 1. On a donc, pour localement presque tout $x\in X$, $\set{a(x), 1-a(x)}\subseteq[0,1]$,
    d'où enfin $\norm{1-a}_{\mathrm{L}^\infty}\le 1$. Mais par ailleurs $\norm{\varphi} = \varphi(a) + \varphi(1-a) \le \varphi(a) + \abs{\varphi(1-a)} \le \varphi(a) + \norm{\varphi}\norm{1 - a}_{\mathrm{L}^\infty}$, et finalement
    $\varphi(a)\ge \norm{\varphi}\left(1 - \norm{1-a}_{\mathrm{L}^\infty}\right) \ge0$. 

    Supposons maintenant $\varphi$ positive. Notons déjà qu'on a bien sûr $\norm{\varphi}\ge\abs{\varphi(1)}=\varphi(1)$. Soit donc $a\in\mathrm{L}^\infty(X,\mu)$ quelconque,
    et notons que $-\norm{a}_{\mathrm{L}^\infty}\cdot1\le a\le\norm{a}_{\mathrm{L}^\infty}\cdot1$, de sorte que $-\norm{a}_{\mathrm{L}^\infty}\varphi(1)\le\varphi(a)\le\norm{a}_{\mathrm{L}^\infty}\varphi(1)$, ce qui conclut.
\end{proof}

\begin{proof}[Démonstration de la proposition \ref{bilateral_of_left}]
    Il s'agit donc de construire une \emph{autre} moyenne $n$ sur $\Gamma$ qui soit cette-fois invariante des deux côtés.
    Posons d'abord, pour $f\in\mathrm{L}^\infty(\Gamma)$ quelconque, $\fundef{\widehat{f}}{\Gamma&\to&\C}{g&\mapsto&m(f\comp r_g)}$. Le lemme \ref{positive_iff_norm}
    montre que $m$ est continue de norme $m(1) = 1$. Cela entraîne
    $\forall g\in\Gamma, \abs{\widehat{f}(g)} \le \norm{f\comp r_g}_{\mathrm{L}^\infty} = \norm{f}_{\mathrm{L}^\infty}$, 
    donc $\widehat{f}\in B(\Gamma)\subseteq\mathscr{L}^\infty(\Gamma)$.
    Posons alors $n(f) := m\left(\widehat{f}\comp\invop\right)$, qui est bien défini car $\widehat{f}$ est bornée \emph{partout}.

    Notons que, pour $f, f_1, f_2\in\mathrm{L}^\infty(\Gamma)$ et $g, x\in\Gamma$, on a :
    \begin{gather*}
        \widehat{1} = 1 \\
        \widehat{f_1 + f_2} = \widehat{f_1} + \widehat{f_2} \\
        \widehat{f\comp\ell_g}(x) = m(f\comp\ell_g\comp r_x) = m(f\comp r_x\comp\ell_g) = m(f\comp r_x) = \widehat{f}(x) \\
        \widehat{f\comp r_g}(x) = m(f\comp r_g\comp r_x) = m(f\comp r_{xg}) = \widehat{f}(xg) = \left(\widehat{f}\comp r_g\right)(x)
    \end{gather*}
    En précomposant par $\invop$ et en appliquant $m$ à ces relations, on obtient :
    \begin{gather*}
        n(1) = m(1\comp\invop) = 1 \\
        n(f_1 + f_2) = m\left(\widehat{f_1}\comp\invop + \widehat{f_2}\comp\invop\right) = n(f_1) + n(f_2) \\
        (n\comp\lambda(g))(f) = m\left(\widehat{f\comp\ell_{g\inv}}\comp\invop\right) = m\left(\widehat{f}\comp\invop\right) = n(f) \\
        (n\comp\rho(g))(f) = m\left(\widehat{f\comp r_g}\comp\invop\right) = m\left(\widehat{f}\comp r_g\comp\invop\right) = m\left(\widehat{f}\comp \invop\comp\ell_{g\inv}\right) = n(f)
    \end{gather*}
    Ce qui conlut.
\end{proof}

Les premiers exemples de groupes moyennables sont les groupes compacts séparés (et en particulier les groupes finis discrets).
En effet, si l'on note $\mu$ la mesure de Haar normalisée d'un tel groupe, l'intégration selon $\mu$ fournit une moyenne invariante. 

Donnons tout de suite un contre-exemple, qui fut fondamental dans le développement historique de la théorie.

\begin{theorem}\label{not_amenable_F2}
    Le groupe libre en deux générateurs $F_2$ (muni de la topologie discrète) n'est pas moyennable.
\end{theorem}

\begin{proof}
    Notons $a, b$ les deux générateurs. Pour $m$ mot réduit en $\set{a, b, a\inv, b\inv}$, notons 
    $S(m)$ l'ensemble des $g\in F_2$ dont l'écriture (unique) comme mot réduit
    commence par $m$. On a par exemple $ab\in S(a)$ mais $a a\inv b \notin S(a)$ car le mot réduit associé 
    à $a a\inv b$ est $b$.

    Remarquons que $a\inv S(a) = S(a\inv)\compl$. En effet, si $m$ est un mot réduit ne commençant pas par $a\inv$, 
    $am$ est un mot réduit commençant par $a$. Réciproquement si $am$ est un mot réduit alors $m$ est réduit et ne commence 
    pas par $a\inv$.

    On montre de même que $b\inv S(b) = S(b\inv)\compl$. Supposons alors qu'il existe une moyenne $m$ sur 
    $F_2$ invariante par translations. On a alors :
    \begin{gather*}
        m\left(\indic_{S(a)}\right) + m\left(\indic_{S(a\inv)}\right) = m\left(\lambda(a\inv)(\indic_{S(a)}) + \indic_{S(a\inv)}\right) = m\left(\indic_{a\inv S(a)} + \indic_{S(a\inv)}\right) = 1 \\
        m\left(\indic_{S(b)}\right) + m\left(\indic_{S(b\inv)}\right) = m\left(\lambda(b\inv)(\indic_{S(b)}) + \indic_{S(b\inv)}\right) = m\left(\indic_{b\inv S(b)} + \indic_{S(b\inv)}\right) = 1 \\
    \end{gather*}
    Mais les ensembles $S(a)$, $S(a\inv)$, $S(b)$ et $S(b\inv)$ sont disjoints, donc :
    \begin{equation*}
        1 = m(1) \ge m\left(\indic_{S(a)} + \indic_{S(a\inv)} + \indic_{S(b)} + \indic_{S(b\inv)}\right) = 2
    \end{equation*} 
    D'où contradiction.
\end{proof}

\TODO{Mentionner importance historique ?}

Il est intéressant de noter que, dans cette preuve, nous n'avons besoin d'évaluer la moyenne que sur des fonctions indicatrices.
Cette remarque ainsi que le cas des groupes compacts suggèrent que l'on peut aussi voir une moyenne comme fonction 
définie sur des parties de $\Gamma$, de manière similaire à une mesure. On retrouverait alors le point de vue \og{}forme 
linéaire positive\fg{} par une forme d'intégration.

Une hypothèse optimiste mais naturelle serait que toute moyenne est donnée par l'intégration pour une mesure 
de Radon. En fait, si c'était le cas, la notion de moyennabilité ne serait pas très intéressante :
en effet, si $\mu$ est une mesure de Radon sur $\Gamma$ telle que $f\mapsto\integral{}{}{f}{\mu}$ soit une moyenne bien définie et invariante,
alors $\mu$ est automatiquement une mesure de Haar de masse $1$, et le théorème \ref{theorem_Haar} assure qu'une telle mesure n'existe que si
$\Gamma$ est compact. Les groupes moyennables seraient donc exactement les groupes compacts !

Évidemment ce n'est pas le cas, et le théorème \ref{Z_amenable} de moyennabilité de $\Z$ montrera que la classe des groupes moyennables est plus grande que 
celle des groupes compacts. Cependant, nous disposons bien d'un théorème de représentation des moyennes comme des
\og{}intégrales\fg{}, à condition d'affaiblir la condition de $\sigma$-additivité des mesures.

Un \emph{contenu} sur un espace mesurable $(X, \mathcal{A})$\footnote{On trouve dans la littérature des définitions de \emph{contenu} autorisant $\mathcal{A}$ à n'être qu'une algèbre d'ensembles,
mais nous supposerons toujours qu'il s'agit d'une $\sigma$-algèbre.} est une fonction $m:\mathcal{A}\to\closure{\R}_+$ vérifiant $m(\varnothing) = 0$
et $m(A_1\cup A_2) = m(A_1) + m(A_2)$ pour $A_1, A_2\in\mathcal{A}$ disjoints. Les mesures sur $(X, \mathcal{A})$
sont donc exactement les contenus $\sigma$-additifs. 

Nous allons à présent développer la théorie de l'intégration des fonctions \emph{bornées} selon un contenu \emph{fini}. Il convient de noter que,
même si nous utiliserons le symbole $\int$, cette théorie ne jouit pas des propriétés agréables de l'intégration de Lebesgue,
le défaut de $\sigma$-additivité se traduisant par l'absence des théorèmes de convergence monotone et dominée.
%Notons tout de suite que les contenus ne donnent pas lieu à 
%une théorie de l'intégration intéressante : en effet, en l'absence du théorème de convergence monotone, il n'est pas clair que l'intégrale 
%des fonctions positives \footnote{On rappelle que l'intégrale d'une fonction positive $f$ est définie comme la borne supérieure des 
%intégrales des fonctions simples positives inférieures à $f$} soit additive ! \TODO{Est-ce que c'est vrai ?} 
%On a tout de même un résultat positif si l'on se restreint
%intégrer des fonctions bornées selon des contenus finis, comme énoncé dans la proposition \ref{content_integration_and_repr} ci-dessous.

\begin{definition}
    Soit $(X, \mathcal{A})$ un espace mesurable, $m$ et $n$ deux contenus sur $(X, \mathcal{A})$.
    On dit que $m$ est \emph{absolument continu} par rapport à $n$, que l'on note $m \ll n$, si tout 
    ensemble $n$-localement négligeable est $m$-localement négligeable, la définition de négligeabilité 
    locale pour les contenus étant identique à celle donnée en \ref{loc_negligible} pour les mesures. 

    Dans le cas où $m$ est fini (i.e $m(X)<+\infty$), cela revient à demander que tout $A\in\mathcal{A}$
    $n$-localement négligeable vérifie $m(A) = 0$.
\end{definition}

\begin{proposition}\label{content_integration_and_repr}
    Soit $(X, \mathcal{A}, \mu)$ un espace mesuré et $m$ un contenu fini sur $(X, \mathcal{A})$ \emph{absolument continu par rapport
    à} $\mu$. Il existe alors une unique forme linéaire positive $\integral{}{}{\blank}{m} : \mathrm{L}^\infty(X, \mu)\to\C$ vérifiant $\integral{}{}{\indic_A}{m} = m(A)$ pour tout
    $A\in\mathcal{A}$. 

    De plus, l'application $I : m\mapsto \integral{}{}{\blank}{m}$ est une bijection de l'ensemble des contenus finis absoluments continus par rapport à $\mu$
    sur l'ensemble des formes linéaires positives sur $\mathrm{L}^\infty(X, \mu)$.
\end{proposition}

\begin{proof}
    On considère le sous-espace vectoriel $\mathscr{S}(X)$ de $\mathscr{L}^\infty(X, \mu)$ formé des fonctions simples, c'est à dire des
    fonctions $f:X\to\C$ mesurables d'image finie, et $\mathrm{S}(X, \mu)$ son image dans $\mathrm{L}^\infty(X, \mu)$. Il est clair que
    $\mathrm{S}(X, \mu)$ est l'espace vectoriel engendré par les classes des indicatrices des éléments de $\mathcal{A}$. Autrement dit, l'application
    $\fundef{\Theta}{\C^{(\mathcal{A})}&\to&\mathrm{S}(X, \mu)}{\delta_A&\mapsto&\indic_A}$ est surjective.
    
    Montrons tout d'abord que $\mathrm{S}(X, \mu)$ est dense dans $\mathrm{L}^\infty(X, \mu)$. Soit donc $f\in\mathscr{L}^\infty(X, \mu)$, 
    que nous supposons d'abord positive, et construisons une suite $g:\N\to \mathscr{S}(X, m)$ de la manière suivante :
    \begin{gather*}
        g_0 := 0 \\
        g_{n+1} := g_n + \frac12\norm{f - g_n}\indic_{\set{x\tq f(x) - g_n(x) \ge \frac12\norm{f - g_n}}}
    \end{gather*}
    Il est clair que chaque $g_n$ est une fonction simple positive et inférieure à $f$, et que la suite $g$ est croissante.
    De plus, pour tout $n\in\N$, et pour presque tout $x\in X$, on est dans l'un des cas suivants : 
    \begin{gather*}
        f(x)-g_{n+1}(x) = f(x)-g_n(x) \le \frac12\norm{f - g_n} \\
        f(x)-g_{n+1}(x) = f(x)-g_n(x)-\frac12\norm{f-g_n} \le \frac12\norm{f-g_n}
    \end{gather*}
    Il vient $\norm{f - g_{n+1}}\le\frac12\norm{f - g_n}$, d'où $\norm{f - g_n}\xrightarrow[n\to+\infty]{} 0$ dans $\mathrm{L}^\infty(X, \mu)$.
    Dans le cas général, il suffit alors de décomposer $f$ en combinaison linéaire de fonctions positives et d'appliquer 
    le résultat à chacune de ces fonctions. 

    Pour $m$ contenu fini avec $m\ll\mu$, posons $\fundef{\widehat{I}_m}{\C^{(\mathcal{A})}&\to&\C}{\delta_A&\mapsto&m(A)}$.
    Soit $\alpha\in\ker\Theta$, i.e tel que $f := \sum_{A\in\mathcal{A}} \alpha_A\indic_A =_\mu^{loc} 0$. Considérons alors l'ensemble 
    fini $\mathcal{S}:=\alpha\inv\left(\set{0}\compl\right)\subseteq\mathcal{A}$ des $A$ tels que $\alpha_A\ne 0$, et la fonction 
    $\epsilon:X\to 2^\mathcal{S}$\footnote{On identifie 2 à l'ensemble $\set{0, 1}$} dont la composante selon $A\in\mathcal{S}$
    est l'indicatrice de $A$. Notons que chaque $A\in\mathcal{S}$ est l'union disjointe des $\epsilon\inv(b)$ pour $b\in2^\mathcal{S}$, $b(A) = 1$.
    On a donc :
    \begin{align*}
        \widehat{I}_m(\alpha) 
            &= \sum_{A\in\mathcal{S}} \alpha_A m(A) \\
            &= \sum_{A\in\mathcal{S}} \sum_{\substack{b\in2^\mathcal{S} \\ b(A) = 1}} \alpha_A m(\epsilon\inv(b)) \\
            &= \sum_{b\in2^\mathcal{S}} m(\epsilon\inv(b)) \sum_{\substack{A\in\mathcal{S} \\ b(A) = 1}} \alpha_A \\
            &= \sum_{b\in2^\mathcal{S}} m(\epsilon\inv(b)) \sum_{A\in\mathcal{S}} \alpha_A b(A)
    \end{align*}
    Or, pour chaque $b\in2^\mathcal{S}$, la fonction $f$ est constante de valeur $\sum_{A\in\mathcal{S}} \alpha_A b(A)$ sur l'ensemble 
    $\epsilon\inv(b)$. Comme $f$ est $\mu$-localement presque nulle, cela implique qu'on a ou bien $\sum_{A\in\mathcal{S}} \alpha_A b(A)=0$ ou bien $\epsilon\inv(b)$ est $\mu$-localement-négligeable et donc $m(\epsilon\inv(b)) = 0$.
    Cela assure finalement que $\widehat{I}_m(\alpha) = 0$.

    On a donc montré que $\ker\Theta\subseteq\ker\widehat{I}_m$. $\widehat{I}_m$ se factorise donc à travers la surjection $\Theta$ en
    $\widetilde{I}_m:\mathrm{S}(X, \mu)\to\C$ linéaire vérifiant $\forall A\in\mathcal{A}, \widetilde{I}_m(\indic_A) = m(A)$. De plus, pour 
    $f\in\mathscr{S}(X)$, on a :
    \begin{equation*}
        \abs{\widetilde{I}_m(f)} = \abs{\sum_{z\in f(X)} zm(f\inv(z))} \le \sum_{z\in f(X)} \norm{f}_{\mathrm{L}^\infty} m(f\inv(z)) = \norm{f}_{\mathrm{L}^\infty} m(X)
    \end{equation*}
    Comme par ailleurs $\widetilde{I}_m(1)=m(X)$, on a donc $\widetilde{I}_m$ continue de norme exactement $m(X)$.
    Par le théorème de prolongement des applications uniformément continues, elle se prolonge donc de manière unique en $\integral{}{}{\blank}{m} : \mathrm{L}^\infty(X, \mu)\to\C$
    linéaire continue de même norme $m(X) = \integral{}{}{1}{m}$. Le lemme \ref{positive_iff_norm} assure alors que $\integral{}{}{\blank}{m}$ est positive, et convient donc. 
    L'unicité de $\integral{}{}{\blank}{m}$ est alors 
    immédiate, la formule $m(A) = \integral{}{}{\indic_A}{m}$ imposant la valeur sur $\mathrm{S}(X, \mu)$, puis sur $\mathrm{L}^\infty(X, \mu)$ par densité et continuité des formes linéaires 
    positives.

    Reste à montrer la bijectivité de $I:m\mapsto \integral{}{}{\blank}{m}$. L'injectivité est immédiate, la formule $m(A) = \integral{}{}{\indic_A}{m}$ déterminant
    entièrement $m$. Soit donc $T : \mathrm{L}^\infty(X, \mu)\to\C$ une forme linéaire positive quelconque, posons $\fundef{m}{\mathcal{A}&\to&\closure{\R}_+}{A&\mapsto&T(\indic_A)}$.
    Il est clair que $m$ est alors un contenu, et la définition assure que $m\ll\mu$ car l'indicatrice d'un borélien $\mu$-localement négligeable
    est nulle dans $\mathrm{L}^\infty(X,\mu)$. Comme $T$ est positive et $\forall A\in\mathcal{A}, m(A)=T(\indic_A)$, l'unicité 
    démontrée précédemment assure que $T = \integral{}{}{\blank}{m}$, ce qui conclut.
\end{proof}

%\TODO{Commentaire sur lien avec construction de l'intégrale de Bochner ?}

\begin{remark}
    Dans le cas où $m$ est une mesure finie, l'unicité assure que la forme $\integral{}{}{\blank}{m}$ construite ci-dessus coïncide bien avec l'intégrale usuelle
    par rapport à $m$, ou plus précisément sa précomposition
    par l'application naturelle $\mathrm{L}^\infty(X, \mu)\to\mathrm{L}^\infty(X, m)\hookrightarrow\mathrm{L}^1(X, m)$.
\end{remark}

Il est clair que la bijection $I$ de la proposition \ref{content_integration_and_repr} fait correspondre les moyennes 
sur $(X, \mathcal{A})$ aux contenus de masse 1 et absoluments continus par rapport à $\mu$. On voudrait maintenant pouvoir exprimer 
la condition d'invariance par translations, et il nous faut pour cela aborder l'aspect fonctioriel de l'intégration.

Soient $(X, \mathcal{A})$, $(Y, \mathcal{B})$ deux espaces mesurables, $m$ un contenu sur $(X, \mathcal{A})$ et $\varphi : X\to Y$ mesurable.
On définit alors le \emph{contenu image} de $m$ par $\varphi$ par $\fundef{\varphi_* m}{\mathcal{B}&\to&\closure{\R_+}}{B&\mapsto&m(\varphi\inv(B))}$.
Il est clair qu'il s'agit encore d'un contenu. Si de plus $X$ et $Y$ sont munis de mesures $\mu$ et $\nu$, et si on a 
$m\ll\mu$ et $\varphi_*\mu\ll\nu$
\footnote{Autrement dit, la préimage par $\varphi$ de tout ensemble $\nu$-localement négligeable est $\mu$-localement négligeable. Cette condition assure que la précomposition 
par $\varphi$ est une opération bien définie de $\mathrm{L}^\infty(Y, \nu)$ vers $\mathrm{L}^\infty(X, \mu)$. }, alors :
\begin{equation*}
    \forall B\in\mathcal{B}, \integral{}{}{\indic_B}{\varphi_*m} = m(\varphi\inv(B)) = \integral{}{}{\indic_{\varphi\inv(B)}}{m} = \integral{}{}{\indic_B\comp\varphi}{m}
\end{equation*}
La forme linéaire $\funlam{\mathrm{L}^\infty(Y, \nu)&\to&\C}{f&\mapsto&\integral{}{}{f\comp\varphi}{m}}$ étant par ailleurs positive, l'unicité dans la proposition
\ref{content_integration_and_repr} assure qu'elle est égale à $\integral{}{}{\blank}{\varphi_*m}$, de sorte que :
\begin{equation*}
    \forall f\in\mathrm{L}^\infty(Y, \nu), \integral{}{}{f}{\varphi_*m} = \integral{}{}{f\comp\varphi}{m}
\end{equation*}

En revenant au cas d'un groupe séparé localement compact $\Gamma$ muni d'une mesure de Haar $\mu$, 
cette dernière égalité montre que l'intégration par rapport à $m$ est une moyenne invariante à gauche (resp. à droite) \ssi 
$m$ est de masse $1$ et $\forall g\in\Gamma, {\ell_g}_*m = m$ (resp. ${r_g}_*m = m$). On dira qu'un tel 
contenu est \emph{invariant à gauche} (resp. \emph{à droite}) s'il vérifie cette dernière condition, et 
\emph{invariant} s'il est invariant à gauche et à droite. En prenant en compte la proposition \ref{bilateral_of_left},
on vient donc de montrer le résultat suivant.
\begin{proposition}
    Un groupe localement compact séparé $\Gamma$ est moyennable \ssi il admet un contenu de masse $1$ et invariant 
    (resp. invariant à gauche, resp. invariant à droite).
\end{proposition}

%\paragraph{}
%
%Comme constaté dès la preuve de la proposition \ref{bilateral_of_left}, il va sans surprise être très souvent utile d'étudier la quasi-mesure $m$ à travers la théorie de l'intégration.
%On dispose pour cela du résultat suivant.
%\begin{proposition}\label{repr}
%    Soit $X$ un espace mesurable. Pour toute quasi-mesure de probabilité $m$ sur $X$, on définit :
%    \begin{equation*}
%        \fundef{I_m}{B(X)&\to&\C}{f&\mapsto&\integral{}{}{f}{m}}
%    \end{equation*}
%    L'application $I:m\mapsto I_m$ est alors une bijection de l'ensemble $\mathrm{Proba}(X)$ des mesures de
%    probabilités sur $X$ sur l'ensemble des formes linéaires positives sur $B(X)$ de valeur $1$ en $\indic$, la 
%    réciproque étant donnée par $T\mapsto(A\mapsto T(\indic_A))$.
%\end{proposition}
%
%On appellera \emph{moyenne} toute forme linéaire positive sur $B(X)$ de valeur $1$ en $\indic$. On va donc montrer que 
%les moyennes sur $X$ sont en bijections avec les mesures de probabilité sur $X$.
%
%Commençons par prouver le lemme suivant, qui sera utile en lui-même.
%
%\begin{lemma}\label{positive_iff_norm}
%    Soit $X$ un espace mesurable et $\varphi:B(X)\to\C$ une forme linéaire. 
%    Si $\varphi(\indic) = 1$, on a l'équivalence :
%    \begin{equation*}
%        \norm{\varphi} = 1 \iff \forall a \ge 0, \varphi(a) \geq 0
%    \end{equation*}
%\end{lemma}
%
%\begin{proof}
%    Supposons d'abord $\norm{\varphi} = 1$, et soit $a\ge 0$ de norme 1. On a donc, pour $x\in X$, $a(x), 1-a(x)\in[0,1]$,
%    d'où enfin $\norm{\indic-a}\le 1$. Mais on a $1 = \varphi(a) + \varphi(\indic-a) \le \varphi(a) + \abs{\varphi(\indic-a)} \le \varphi(a) + \norm{\indic - a}$, d'où
%    $\varphi(a)\ge 1 - \norm{\indic-a}\ge0$. \\
%    Supposons maintenant $\varphi$ positif. Notons déjà que l'égalité $\varphi(\indic) = 1$ entraîne $\norm{\varphi}\ge1$. Soit donc $a\in B(X)$ quelconque,
%    et notons que $-\norm{a}\indic\le a\le\norm{a}\indic$, de sorte que $-\norm{a}\le\varphi(a)\le\norm{a}$, ce qui conclut.
%\end{proof}
%
%\begin{proof}[Démonstration du théorème \ref{repr}]
%    Toute fonction mesurable bornée étant intégrable par rapport à toute mesure de probabilité, il est clair que $I$ est bien définie. 
%    Les propriétés élémentaires de l'intégration de Lebesgue, qui restent valables dans le cas de mesures finiment-additives,
%    donnent immédiatement que chaque $I_m$ est bien moyenne, et qu'on a
%    $I_m(\indic_A) = m(A)$ pour tout $A\subseteq X$ mesurable.
%
%    Soit maintenant $T$ une moyenne sur $X$.
%    Il faut alors montrer que $m : A\mapsto T(\indic_A)$ est bien une mesure de probabilité sur $X$,
%    puis que $T = I_m$ pour cette mesure $m$. $m$ est bien à valeurs positive par positivité de $T$, et 
%    de plus $m(X) = T(\indic) = 1$ et $m(\varnothing) = T(0) = 0$. Soient enfin $A, B\subseteq X$ deux ensembles
%    mesurables disjoints, de sorte que $\indic_{A\cup B} = \indic_A + \indic_B$. On a alors bien 
%    $m(A\cup B) = m(A) + m(B)$ ce qui assure l'additivité finie. 
%
%    Pour finir, considérons le sous-espace vectoriel $S(X, m)$ de $B(X)$ des fonctions $m$-simples, c'est à dire des 
%    fonctions $f:X\to\C$ mesurables, d'image finie, et vérifiant $\forall c\in\C^*, m(f\inv(x)) < +\infty$ (cette dernière 
%    condition est automatiquement vérifiée dans notre cas d'une mesure de probabilité, mais nécessaire en général).
%    Montrons que $S(X, m)$ est dense dans $B(X)$. Soit donc $f\in B(X)$, que nous supposons d'abord positive, et construisons 
%    une suite $g:\N\to S(X, m)$ de la manière suivante :
%    \begin{gather*}
%        g_0 := 0 \\
%        g_{n+1} := g_n + \frac12\norm{f - g_n}\indic_{\set{x\tq f(x) - g_n(x) \ge \frac12\norm{f - g_n}}}
%    \end{gather*}
%    Il est clair que chaque $g_n$ est une fonction simple positive et inférieure à $f$, et que la suite $g$ est croissante.
%    De plus, pour tout $n\in\N$ et $x\in X$, on est dans l'un des cas suivants : 
%    \begin{gather*}
%        f(x)-g_{n+1}(x) = f(x)-g_n(x) \le \frac12\norm{f - g_n} \\
%        f(x)-g_{n+1}(x) = f(x)-g_n(x)-\frac12\norm{f-g_n} \le \frac12\norm{f-g_n}
%    \end{gather*}
%    Il vient $\norm{f - g_{n+1}}\le\frac12\norm{f - g_n}$, d'où $\norm{f - g_n}\xrightarrow[n\to+\infty]{} 0$.
%    Dans le cas général, il suffit alors de décomposer $f$ en combinaison linéaire de fonctions positives et d'appliquer 
%    le résultat à chacune de ces fonctions. Cela montre donc la densité souhaitée. Or les formes linéaires $T$ et $I_m$
%    coïncident sur les indicatrices, donc sur $S(X, m)$, et elles sont continues par le lemme \ref{positive_iff_norm}. 
%    Par densité, elles sont donc égales, ce qui termine la preuve.
%\end{proof}
%
%\TODO{Commentaire sur lien avec construction de l'intégrale de Bochner ?}
%
%Cela nous amène à donner la définition suivante.
%
%\begin{definition}
%    Soit $\Gamma$ un groupe localement compact et séparé. Une moyenne $T : B(\Gamma)\to\C$ sur $\Gamma$ est dite 
%    \emph{invariante à gauche} (resp. \emph{à droite}) si $\forall g\in\Gamma, T\comp\lambda_g = T$
%    (resp. $T\comp\rho_g = T$). Une moyenne invariante à droite et à gauche sera simplement dite \emph{invariante}.
%\end{definition}
%
%\begin{remark}
%    Les moyennes invariantes à gauche sur $\Gamma$ sont exactement les morphismes de la représentation $(B(\Gamma), \lambda)$ de 
%    $\Gamma$ vers la représentation triviale, avec la condition suppplémentaire que $T(\indic) = 1$.
%\end{remark}
%
%Remarquons que la bijection $I$ de la proposition \ref{repr} fait correspondre les mesures de probabilité 
%invariantes à gauche aux moyennes invariantes à gauche sur $\Gamma$. En effet, si $m$ est invariante à gauche, alors pour tous 
%$f\in B(\Gamma)$ et $g\in\Gamma$, on a 
%$I_m(\lambda_g(f)) = \integral{}{}{f\comp\ell_{g\inv}}{m} = \integral{}{}{f}{m} = I_m(f)$. 
%
%Réciproquement, si $T$ est une 
%moyenne invariante à gauche, alors la mesure $m$ associe vérifie, pour tous $A\in\Bor(\Gamma)$ et $g\in\Gamma$, 
%$m(g\inv A) = T(\indic_{g\inv A}) = T(\indic_A \comp\ell_g) = T(\lambda_{g\inv}(\indic_A)) = T(\indic_A) = m(A)$.
%En prenant en compte la proposition \ref{bilateral_of_left}, on vient donc de montrer le résultat suivant.
%\begin{proposition}
%    Un groupe localement compact séparé $\Gamma$ est moyennable \ssi il admet une moyenne invariante (resp. à gauche, resp. à droite).
%\end{proposition}
%
%Dans la suite, nous supposons toujours que le groupe topologique $\Gamma$ est séparé et localement compact.
%\paragraph{}

\paragraph{}
Nous allons maintenant vouloir appliquer les outils de l'analyse fonctionelle à l'étude des moyennes invariantes.
Pour cela, le lemme \ref{positive_iff_norm} va de nouveau s'avérer crucial.

En effet, la condition $\norm\varphi = \varphi(1)$ qui apparaît dans ce lemme a le bon goût d'être conservée lors du prolongement de l'application 
$\varphi$ par le théorème de Hahn-Banach, alors qu'il n'est pas clair du tout qu'on puisse prolonger une forme linéaire positive en préservant 
la positivité. On va donc pouvoir se restreindre à étudier les moyennes sur des sous-espaces plus concrets de $\mathrm{L}^\infty(\Gamma)$.

Plus précisément, toujours pour $\Gamma$ groupe localement compact et séparé, on introduit $L_0(\Gamma) := \sum_{g\in\Gamma} \Ima(\lambda(g) - \id)$ le sous-espace de 
$\mathrm{L}^\infty(\Gamma)$ engendré par les classes de fonctions de la forme $\lambda(g)(f) - f$ pour $f\in \mathrm{L}^\infty(\Gamma)$. 
Il est alors clair qu'une moyenne sur $\Gamma$ est invariante à gauche \ssi sa restriction à $L_0(\Gamma)$ est nulle. 

\begin{lemma}\label{one_not_mem_L0}
    $1\notin L_0(\Gamma)$
\end{lemma}

\begin{proof}
    Traitons d'abord le cas $1 =_\mu^{loc} \lambda(\gamma)(f) - f$ pour certains $\gamma\in\Gamma$ et 
    $f\in \mathscr{L}^\infty(\Gamma)$. L'ensemble $S:=\set{x\tq f(\gamma\inv x)\ne 1+f(x)}\cup\set{x\tq\abs{f(x)}>\norm{f}_{\mathrm{L}^\infty}}$ est donc $\mu$-localement négligeable,
    et par conséquent $T := \bigcup_{n\in\N}\gamma^{-n}S$ l'est aussi. $T\compl$ est donc non-vide\footnote{On peut par exemple considérer un voisinage compact $K$
    de l'origine, qui est de mesure finie non-nulle. L'intersection $K\cap T$ étant $\mu$-négligeable, on en déduit que $K\setminus T$ est non-vide.}, choisissons alors $t\in T\compl$. On a alors 
    $\forall n\in\N, \norm{f}_{\mathrm{L}^\infty}\geq \abs{f(\gamma^{n}t)} = \abs{f(t) + n}$, ce qui est impossible puisque $\abs{f(t) + n}\xrightarrow[n\to+\infty]{}+\infty$.
    
    Plus généralement, supposons maintenant $1 =_\mu^{loc} \sum_{i\in I} (\lambda(\gamma_i)(f_i) - f_i)$ pour $I$ fini, $\gamma : I \to\Gamma$ et
    $f : I\to \mathscr{L}^\infty(\Gamma)$, et ajoutons l'hypothèse supplémentaire que $\Gamma$ est $\sigma$-compact. 
    En particulier la mesure $\mu$ est alors $\sigma$-finie, et on peut donc munir le groupe séparé et localement compact $\Gamma^I$ (dont $\gamma : i\mapsto \gamma_i$ est
    un élément) de la mesure produit $\nu:=\mu^{\otimes I}$. Il est clair que la mesure $\nu$ ainsi formée est encore 
    de Radon et invariante (à gauche si $\mu$ est invariante à gauche, à droite si $\mu$ est invariante à droite), et donc une mesure de Haar sur $\Gamma^I$. On considère alors la fonction :
    \begin{equation*}
        \fundef{F}{\Gamma^I&\to&\C}{x&\mapsto&\frac{1}{\card{I}}\sum_{i\in I}f_i(x_i)}
    \end{equation*}
    $F$ est clairement mesurable, vérifions maintenant que la majoration $\abs{F(x)}\le\frac1{\card{I}}\sum_i\norm{f_i}_{\mathrm{L}^\infty}$
    est valable pour $\nu$-presque tout $x\in\Gamma^I$\footnote{On travaille avec des espaces mesurés $\sigma$-finis, donc 
    les ensembles localement négligeables sont exactement les ensembles négligeables.}. 

    Posons $S_i:=\set{x\in\Gamma\tq \abs{f_i(x)}>\norm{f}_{\mathrm{L}^\infty}}$ et $T:=\set{x\in\Gamma^I\tq \abs{F(x)}>\frac1{\card{I}}\sum_i\norm{f_i}_{\mathrm{L}^\infty}}$. 
    On a clairement $T\subseteq\bigcup_{i\in I}\pi_i\inv(S_i)$, où $\pi_i:\Gamma^I\to\Gamma$ désigne la $i$-ème projection. 
    Or $\mu^{\otimes I}(\pi_i\inv(S_i)) = \mu^{\otimes I}(\Gamma\times\dots\times S_i\times\dots\times\Gamma)=\mu(\Gamma)\times\dots\times\mu(S_i)\times\dots\times\mu(\Gamma)=0$ puisque $\mu(S_i)=0$.
    Donc $T$ est bien $\mu^{\otimes I}$-négligeable.
    Cela assure que $F\in \mathscr{L}^\infty(\Gamma^I)$. 
    On montre similairement que $1 =_{\nu} \lambda(\gamma)(F) - F$, ce qui nous ramène au cas déjà traité. 

    Pour conclure, il reste à montrer qu'on peut toujours se ramener à $\Gamma$ $\sigma$-compact. 
    On utilise pour cela le lemme suivant :
    \begin{lemma}\label{exists_sigma_compact_subgroup}
        Soit $\Gamma$ un groupe topologique séparé et localement compact. Toute partie compacte de $\Gamma$ est contenue dans un sous-groupe ouvert $\Gamma'$ de $\Gamma$ qui est $\sigma$-compact.
        De plus, la restriction à un tel $\Gamma'$ de toute mesure de Haar $\mu$ sur $\Gamma$
        est une mesure de Haar sur $\Gamma'$.
    \end{lemma}
    Une fois ce lemme acquis, supposons encore $1 =_\mu^{loc} \sum_{i\in I} (\lambda(\gamma_i)(f_i) - f_i)$ pour $I$ fini, $\gamma : I \to\Gamma$ et
    $f : I\to \mathscr{L}^\infty(\Gamma)$. Donnons nous un sous-groupe ouvert et $\sigma$-compact $\Gamma'$ contenant tous les $\gamma_i$, $\mu'$
    la restriction de $\mu$ à $\Gamma'$, et $\fundef{\widetilde{f}}{I&\to&\mathscr{L}^\infty(\Gamma')}{i&\mapsto&{f_i}_{|\Gamma'}}$.
    Comme la trace sur $\Gamma'$ d'un ensemble $\mu$-localement négligeable est $\mu'$-localement négligeable, on a 
    que chaque $\widetilde{f}_i$ est bien un élément de $\mathscr{L}^\infty(\Gamma')$, ainsi que 
    l'égalité $1 =_{\mu'}^{loc} \sum_{i\in I} (\lambda(\gamma_i)(\widetilde{f}_i) - \widetilde{f}_i)$, ce
    qui nous ramène au cas précédemment traité. Cela conclut la preuve.
\end{proof}

\begin{proof}[Démonstration du lemme \ref{exists_sigma_compact_subgroup}]
    Soit $K$ une partie compacte quelconque de $\Gamma$. Choisissons $T$ un voisinage compact de l'origine, 
    et posons $S := (K\cup T) \cup (K\cup T)\inv$, qui est un voisinage symétrique compact de l'origine 
    contenant $K$. Vérifions que $\Gamma' := \bigcup_{n\in\N^*} S^n$ convient. Tout d'abord, la suite de compact $S^n$ 
    est croissante, car $S$ contient $1$, donc tout $x\in S^n$ vérifie aussi $x = x\cdot 1\in S^{n+1}$. Cela assure que 
    $\Gamma'$ est $\sigma$-compact. De plus, il est clair qu'il s'agit d'un sous-groupe de $\Gamma'$ : chaque $S^n$ est
    symmétrique ce qui assure la stabilité par inverse, et si $x\in S^p, y\in S^q$ on a $xy \in S^{p+q}$. L'inclusion $S\subseteq\Gamma'$ 
    montre que $\Gamma'$ est un voisinage de l'origine donc ouvert, et également que $T\subseteq\Gamma'$.

    Soit maintenant un sous-groupe ouvert $\Gamma'$ de $\Gamma$ quelconque. Notons tout d'abord que 
    $\Gamma'$ est borélien, donc la restriction de $\mu$ à $\Gamma'$ est bien définie, et $\Gamma$ est localement compact 
    (séparé) comme ouvert d'un espace localement compact. Notons d'abord que $\mu'$ est bien de Radon. La 
    finitude sur les compacts est acquise, ainsi que la régularité intérieure puisque les ouverts de 
    $\Gamma'$ sont ouverts dans $\Gamma$. Pour la régularité supérieure, il suffit de remarquer que si 
    $B\subseteq\Gamma'$ est borélien (dans $\Gamma'$ donc dans $\Gamma$) et si $U\supseteq A$ est un ouvert de 
    $\Gamma$ tel que $\mu(U)$ soit proche de $\mu(A)$, alors $U\cap\Gamma'\supseteq A$ est un ouvert de 
    $\Gamma'$ tel que $\mu'(U)$ est encore plus proche de $\mu'(A) = \mu(A)$. Enfin l'invariance par translation de 
    $\mu'$ (du même côté que $\mu$ bien sûr) est évidente, ce qui conclut.
\end{proof}

Le lemme \ref{one_not_mem_L0} nous permet de considérer l'espace $L(\Gamma) := \C\cdot 1 \oplus L_0(\Gamma)$, qui est intéressant en ce qu'il permet de caractériser entièrement
les moyennes à gauche sur $\Gamma$, au sens du théorème suivant.

\begin{theorem}\label{left_mean_iff}
    Considérons l'application linéaire $\widetilde{T} : L(\Gamma)\to\C$ définie par $\widetilde{T}(1) = 1$
    et $\widetilde{T}_{|L_0(\Gamma)} = 0$. 
    
    Le groupe $\Gamma$ est moyennable \ssi l'application $\widetilde{T}$ est continue et de norme $1$.
    Si c'est le cas, les moyennes à gauche sur $\Gamma$ sont exactement les prolongements de $\widetilde{T}$ à 
    $\mathrm{L}^\infty(\Gamma)$ qui préservent la norme.
\end{theorem}

En remarquant que $\forall c\in\C, \forall f\in L_0(\Gamma), \widetilde{T}(c + f) = c$, et qu'on a toujours 
$\norm{\widetilde{T}}\ge1$ par $\widetilde{T}(1) = 1$, on obtient le critère de moyennabilité suivant:

\begin{corollary}\label{amenable_iff_L0}
    $\Gamma$ est moyennable \ssi $\forall c\in\C, \forall f\in L_0(\Gamma), \abs{c}\le\norm{c + f}_{\mathrm{L}^\infty}$.
\end{corollary}

\begin{proof}[Démonstration du théorème \ref{left_mean_iff}]
    Supposons d'abord $\Gamma$ moyennable, et soit $T$ une moyenne à gauche sur $\Gamma$.
    On sait déjà que $T$ prolonge $\widetilde{T}$, puisque $T$ est nulle sur $L_0(\Gamma)$ et 
    $T(1)=1$. On a donc $\norm{\widetilde{T}}\le\norm{T}=1$ par restriction, et en fait $\norm{\widetilde{T}} = 1$
    puique $\widetilde{T}(1)=1$. 

    Supposons maintenant $\norm{\widetilde{T}}=1$, et soit $T$ un prolongement linéaire continu de $\widetilde{T}$ de norme $1$.
    On a alors $T(1)=1=\norm{T}$ et $T_{|L_0(\Gamma)} = 0$, donc $T$ est une moyenne à gauche sur $\Gamma$. Le théorème de Hahn-Banach
    garantissant l'existence d'un tel prolongement, cela termine la preuve.
\end{proof}

Illustrons ce critère sur le cas du groupe libre $F_2$. Dans ce cas, on peut montrer que $\norm{\widetilde{T}}\ge3$.
Reprenons pour cela les notations de la preuve du théorème \ref{not_amenable_F2}, et posons $f := (\lambda(a\inv)-\id)(\indic_{S(a)})$ et 
$g:=(\lambda(b\inv) - \id)(\indic_{S(b)})$, qui sont deux éléments de $L_0(F_2)$. L'égalité $a\inv S(a) = S(a\inv)\compl$ donne que 
$f = \indic_{a\inv S(a)} - \indic_{S(a)} = \indic_{\set{1}\cup S(b)\cup S(b\inv)}$, et de même 
$g = \indic_{\set{1}\cup S(a)\cup S(a\inv)}$, de sorte que $f+g=\indic + \delta_1$. Mais alors 
$-\frac23(\indic+\delta_1)\in L_0(F_2)$, donc $\widetilde{T}(\indic - \frac23(\indic+\delta_1)) = 1$, 
et d'autre part $\norm{\indic - \frac23(\indic+\delta_1)}_{\mathrm{L}^\infty}=\frac13$. 
On a donc bien $\norm{\widetilde{T}}\geq3$.

\paragraph{}

Donnons maintenant, toujours à l'aide du critère \ref{amenable_iff_L0}, notre premier exemple de groupe moyennable non-compact.

\begin{theorem}\label{Z_amenable}
    $\Z$ est moyennable.
\end{theorem}

\begin{proof}
    Commençons par simplifier un peu notre description de $L_0(\Z)$. Un simple argument de somme
    télescopique montre que, pour $n>0$ et $u\in \mathrm{L}^\infty(\Z)=\ell^\infty(\Z)$:
    \begin{equation*}
        (\lambda(n)-\id)(u) = \left(\sum_{1\le i\le n} \lambda(i+1) - \lambda(i)\right)(u) = (\lambda(1)-\id)\left(\sum_{1\le i\le n} \lambda(i)(u)\right)
    \end{equation*}

    Comme de plus $\lambda(0)-\id = 0$ et $\lambda(-n)-\id = -(\lambda(n)-\id)\comp\lambda(-n)$,
    on en déduit que $\Ima(\lambda(n)-\id)\subseteq\Ima(\lambda(1)-\id)$ pour tout $n\in\Z$, et donc $L_0(\Z) = \Ima(\lambda(1)-\id)$. \\

    Soient maintenant $c\in\C$ et $v\in L_0(\Z)$. On peut donc écrire $v = \lambda(1)(u)-u$ pour un certain $u\in\ell^\infty(\Z)$. 
    On veut montrer $\abs{c}\le\norm{c + \lambda(1)(u) - u}_\infty =: M$. 

    Par définition, on a $\forall n\in\Z, -M\le c+u_{n-1}-u_n\le M$. En moyennant ces inégalités pour 
    $n\in\rrbracket N-k, N\rrbracket$, on obtient, encore par un argument de somme télescopique :
    \begin{equation}\label{Z_amenable_eq1}
        \forall N\in\Z, \forall k\in\N,\quad -M\le c-\frac{u_N - u_{N-k}}{k} \le M
    \end{equation}

    Mais $u$ est bornée, donc il existe une suite strictement croissante $\varphi:\N\to\N$ telle que $u\comp\varphi$
    soit convergente. En particulier $\frac{\abs{u_{\varphi(n+1)}-u_{\varphi(n)}}}{\varphi(n+1)-\varphi(n)}\le\abs{u_{\varphi(n+1)}-u_{\varphi(n)}}\xrightarrow[n\to+\infty]{}0$, 
    donc $c-\frac{u_{\varphi(n+1)}-u_{\varphi(n)}}{\varphi(n+1)-\varphi(n)}\xrightarrow[n\to+\infty]{}c$. En passant à la limite dans l'inégalité (\ref{Z_amenable_eq1}),
    on obtient donc $-M\le c\le M$, ce qui conclut.
\end{proof}

Présentée ainsi, la preuve précédente peut sembler très spécifique à $\Z$ et peu généralisable. Pourtant, l'idée clé 
est simplement de considérer une \og{}moyenne\fg{} d'inégalités de la forme $-M\le c + u(\gamma\inv x) - u(x)\le M$, 
ce qui peut tout à fait se généraliser, \emph{a minima} pour un groupe discret quelconque. 

Les obstacles sont donc la forme spécifique de $L_0(\Z)$ et le recours à une suite extraite pour forcer la convergence des $u_{n+1}-u_n$. Nous allons voir 
qu'il est possible de les contourner.

\begin{proof}[Deuxième démonstration du théorème \ref{Z_amenable}]
    Soient $c\in\C$ et $v\in L_0(\Z)$, que l'on peut bien sûr décomposer en $v = \sum_{i\in I} (\lambda(n_i)(f_i) - f_i)$ pour certains $I$ fini, $n : I \to\Z$ et
    $f : I\to\ell^\infty(\Z)$. Posons toujours $M := \norm{c + v}_\infty$.

    Par définition, on a $\forall k\in\Z$:
    \begin{equation*}
        -M \le c + \sum_i (f_i(k-n_i) - f_i(k)) \le M
    \end{equation*}
    En moyennant ces inégalités pour $k\in F_N := \llbracket-N,N\rrbracket$, on obient $\forall N\in\N$:
    \begin{equation}\label{Z_amenable_eq2}
        -M \le c + \sum_i \frac1{2N+1} \sum_{k=-N}^N (f_i(k-n_i) - f_i(k)) \le M
    \end{equation}

    Étudions donc, pour $i$ fixé, les moyennes de la forme $\frac1{2N+1} \sum_{k=-N}^N (f_i(k-n_i) - f_i(k))$. On a :
    \begin{align*}
        \abs{\sum_{k=-N}^N (f_i(k-n_i) - f_i(k))} &= \abs{\sum_{k\in F_N - n_i} f_i(k) - \sum_{k\in F_N}f_i(k)} \\
            &= \abs{\sum_{\substack{k\in F_N - n_i \\ k\notin F_N}} f_i(k) -
            \sum_{\substack{k\in F_N \\ k\notin F_N-n_i}} f_i(k)} \\
            &\le \sum_{k\in (F_N-n_i)\triangle F_N} \norm{f_i}_\infty \\
            &=\norm{f_i}_\infty \cdot \card{(F_N-n_i)\triangle F_N}
    \end{align*}
    Or, pour $N>n_i$, l'ensemble $(F_N-n_i)\triangle F_N = \llbracket -N-n_i, -N\llbracket\ \amalg\ \rrbracket N-n_i, N\rrbracket$ est de cardinal $2n_i$ constant.
    On a donc:
    \begin{gather*}
        \frac{\card{(F_N-n_i)\triangle F_N}}{2N+1} \xrightarrow[N\to+\infty]{} 0 \\
        \frac1{2N+1} \sum_{k=-N}^N (f_i(k-n_i) - f_i(k)) \xrightarrow[N\to+\infty]{} 0
    \end{gather*}

    En passant à la limite dans l'inégalité (\ref{Z_amenable_eq2}), on obtient donc $-M\le c\le M$, ce qui conclut.
\end{proof}

Notons que, dans cette deuxième démonstration, l'hypothèse $\Gamma=\Z$ n'a été utilisée que pour établir l'existence d'une suite $F$ 
de parties finies non-vides de $\Z$
telle que $\frac{\card{(F_N-n)\triangle F_N}}{\card{F_N}} \xrightarrow[N\to+\infty]{} 0$ pour tout $n\in\Z$. 

Dans le cas général, la mesure de Haar $\mu$ va remplacer le cardinal, et on va donc s'intéresser aux réels $\frac{\mu(\gamma F \triangle F)}{\mu(F)}$ 
pour $F$ partie compacte de $\Gamma$ et $\gamma\in\Gamma$, et à leur comportement asymptotique
lorsque $F$ est grand. C'est l'objet de la partie suivante.

\section{Conditions de F\o{}lner et de Reiter}

Dans cette section, $\Gamma$ est un groupe topologique séparé et localement compact, et on fixe 
$\mu$ une mesure de Haar \emph{à gauche} sur $\Gamma$. On prendra garde à ce que, contrairement au chapitre précédent,
nous travaillerons à partir de maintenant avec des espaces et des objets dépendants du choix de la mesure de Haar,
notamment pour des questions de normalisation. Évidemment cela nous soulève pas de difficulté majeure et 
le lecteur pourra aisément deviner comment se comportent ces objets par changement de la mesure.

Pour $X$ un ensemble quelconque, notons $\parts(X)$ (resp. $\parts^*(X)$) l'ensemble des parties (resp. parties non-vides) de $X$, et $\finparts(X)$ (resp $\finparts^*(X)$) 
l'ensemble des parties finies (resp. parties finies non-vides) de $X$. Si maintenant $X$ est un espace topologique, 
on note aussi $\mathfrak{K}(X)$ (resp. $\mathfrak{K}^*(X)$, resp. $\mathfrak{K}_+(X)$) l'ensemble des parties compactes 
(resp. compactes non-vides, resp. compactes d'intérieur non-vide) de $X$. 

On s'intéresse à l'\emph{application de F\o{}lner} $\mathfrak{f} : \mathfrak{K}_+(\Gamma) \to \mathcal{F}(\Gamma, \R_+)$, définie sur l'ensemble 
$\mathfrak{K}_+(\Gamma)$ par $\mathfrak{f}(K)(\gamma) = \frac{\mu(\gamma K \triangle K)}{\mu(K)}$.
C'est une application bien définie car tout élément de $\mathfrak{K}_+(\Gamma)$ est de mesure finie non-nulle,
et il est clair qu'elle ne dépend pas de la normalisation choisie pour la mesure de Haar.

Un filtre $\mathscr{F}$ sur $\mathfrak{K}_+(\Gamma)$ est dit \emph{faiblement de F\o{}lner} s'il est non-trivial\footnote{Cela signifie que $\varnothing\notin\mathscr{F}$,
ou encore que $\mathscr{F}\ne\parts(X)$.} et si le filtre $\mathfrak{f}_*\mathscr{F}$ image directe 
de $\mathscr{F}$ par l'application de F\o{}lner converge vers $0$ pour la topologie de la convergence simple sur $\mathcal{F}(\Gamma, \R_+)$. 
Si la convergence est uniforme sur les compacts, on parle de filtre \emph{fortement de F\o{}lner}.
Une suite $K:\N\to\mathfrak{K}_+(\Gamma)$ est \emph{faiblement (resp. fortement) de F\o{}lner} si le filtre $K_*\nhds_{+\infty}^\N$, image directe par $K$ du filtre des
voisinages de l'infini dans $\N$, est faiblement (resp. fortement) de F\o{}lner. Comme $\mathfrak{f}_*K_*\nhds_{+\infty}^\N = (\mathfrak{f}\comp K)_*\nhds_{+\infty}^\N$,
une suite $K:\N\to\mathfrak{K}_+(\Gamma)$ est donc faiblement (resp. fortement) de F\o{}lner \ssi la suite de fonctions
$n\mapsto\left(\gamma\mapsto\frac{\mu(\gamma K_n\triangle K_n)}{\mu(K_n)}\right)$ converge vers $0$ simplement 
(resp. uniformément sur tout compact).

Notons que si $\Gamma$ est discret, 
la topologie de la convergence compacte coïncide avec celle de la convergence simple, de sorte que tout filtre 
faiblement de F\o{}lner est automatiquement fortement de F\o{}lner.
De plus, les parties compactes d'intérieur non vide sont exactement les parties finies non-vides,
et la mesure de Haar s'identifie (à un scalaire près) à la mesure de comptage. Dans ce cas, un filtre non-trivial $\mathscr{F}$
est donc de F\o{}lner si et seulement si, pour tout $\gamma\in\Gamma$, la fonction $F\mapsto \frac{\card{\gamma F\triangle F}}{\card{F}}$ 
converge vers $0$ selon le filtre $\mathscr{F}$. De même, une suite $F:\N\to\finparts^*(\Gamma)$
est de F\o{}lner \ssi pour tout $\gamma\in\Gamma$, la suite $n\mapsto\frac{\card{\gamma F_n\triangle F_n}}{\card{F_n}}$ converge 
vers $0$.

Avant d'aller plus loin, donnons un critère plus simple pour l'existence de filtres de F\o{}lner (faibles ou forts) sur $\Gamma$.

\begin{lemma}\label{Folner_filter_of_cond}
    $\Gamma$ admet un filtre faiblement de F\o{}lner \ssi il satisfait la 
    \emph{condition de F\o{}lner faible} :
    \begin{equation}\label{weak_Folner_cond}\tag{WF}
        \forall\varepsilon>0, \forall S\in\finparts(\Gamma), \exists K\in\mathfrak{K}_+(\Gamma), \forall \gamma\in S, 
        \frac{\mu(\gamma K\triangle K)}{\mu(K)}<\varepsilon
    \end{equation}
    Si de plus $\Gamma$ est dénombrable, cette condition est équivalente à l'existence d'une \emph{suite} faiblement de F\o{}lner.

    $\Gamma$ admet un filtre fortement de F\o{}lner \ssi il satisfait la \emph{condition de F\o{}lner forte} : 
    \begin{equation}\label{strong_Folner_cond}\tag{SF}
        \forall\varepsilon>0, \forall A\in\mathfrak{K}(\Gamma), \exists K\in\mathfrak{K}_+(\Gamma), \forall\gamma\in A, 
        \frac{\mu(\gamma K\triangle K)}{\mu(K)}<\varepsilon
    \end{equation}
    Si de plus $\Gamma$ est $\sigma$-compact, cette condition est équivalente à l'existence s'une \emph{suite} fortement de F\o{}lner.
\end{lemma}

\begin{proof}
    Les ensembles de la forme $\set{f \in \mathcal{F}(\Gamma, \R_+) \tq \forall\gamma\in S, f(\gamma) < \varepsilon}$, pour $S$ fini et $\varepsilon>0$, 
    forment une base de voisinage de la fonction nulle dans $\mathcal{F}(\Gamma, \R_+)$ muni de la topologie produit. \eqref{weak_Folner_cond} exprime donc 
    le fait que $0$ est adhérent à l'image de $\mathfrak{f}$ pour cette topologie, ce qui est équivalent avec 
    l'existence d'un filtre $\mathscr{F}$ non-trivial sur $\mathfrak{K}_+(\Gamma)$ tel que $\mathfrak{f}_*\mathscr{F}$ converge vers $0$
    simplement. C'est précisément la définition d'un filtre de F\o{}lner faible.
    De plus, si $\Gamma$ est dénombrable, la topologie produit sur $\mathcal{F}(\Gamma, \R_+)$ est à base dénombrable de voisinages,
    donc l'adhérence séquentielle de l'image de $\mathfrak{f}$ égale son adhérence, ce qui fournit la caractérisation séquentielle recherchée.

    La deuxième partie du théorème se prouve similairement, en obesrvant que les ensembles de la forme 
    $\set{f \in \mathcal{F}(\Gamma, \R_+) \tq \forall\gamma\in A, f(\gamma) < \varepsilon}$, pour $A$ compact et $\varepsilon>0$, 
    forment une base de voisinage de la fonction nulle pour la topologie de la convergence compacte, et en notant que cette 
    topologie est à base dénombrable de voisinages lorsque $\Gamma$ est $\sigma$-compact.
\end{proof}

\begin{remark}
    On verra au théorème \ref{amenable_TFAE} que les conditions \eqref{weak_Folner_cond} et \eqref{strong_Folner_cond} sont en fait équivalentes.
    Cela donne un critère plus intéressant pour l'existence d'une suite de F\o{}lner faible : si $\Gamma$ 
    est $\sigma$-compact et vérifie \eqref{weak_Folner_cond}, il vérifie aussi \eqref{strong_Folner_cond} et admet 
    donc une suite de F\o{}lner forte, qui est automatiquement une suite de F\o{}lner faible.
\end{remark}

Comme anoncé, on peut alors généraliser le théorème \ref{Z_amenable} à tout groupe muni d'un filtre faiblement de F\o{}lner,
en adaptant directement la deuxième preuve de ce théorème.

\begin{theorem}\label{amenable_of_Folner}
    Si $\Gamma$ admet un filtre faiblement de F\o{}lner, alors $\Gamma$ est moyennable.
\end{theorem}

\begin{proof}
    On utilise toujours le critère \ref{amenable_iff_L0}. Soient donc $c\in\C$ et $v\in L_0(\Gamma)$, 
    que l'on écrit encore sous la forme $v = \sum_{i\in I} (\lambda(\gamma_i)(f_i) - f_i)$ pour certains $I$ fini, $\gamma : I \to\Gamma$ et
    $f : I\to \mathscr{L}^\infty(\Gamma)$. Posons aussi $M := \norm{c + v}_{\mathrm{L}^\infty}$.

    Par définition, on a $\forall x\in\Gamma$:
    \begin{equation}\label{amenable_of_Folner_eq1}
        -M \le c + \sum_i (f_i(\gamma_i\inv x) - f_i(x)) \le M
    \end{equation}
    En moyennant ces inégalités sur un $K\in\mathfrak{K}_+(\Gamma)$ quelconque, on obtient:
    \begin{equation}\label{amenable_of_Folner_eq2}
        -M \le c + \sum_i \frac1{\mu(K)} \integral{K}{}{\left(f_i(\gamma_i\inv x) - f_i(x)\right)}{\mu(x)} \le M
    \end{equation}

    Or, pour $i$ fixé, l'invariance par translation de $\mu$ donne :
    \begin{align*}
        \abs{\integral{K}{}{(f_i(\gamma_i\inv x) - f_i(x))}{\mu(x)}} 
            &= \abs{\integral{\gamma_i\inv K}{}{f_i}{\mu} - \integral{K}{}{f_i}{\mu}} \\
            &= \abs{\integral{\gamma_i\inv K\setminus K}{}{f_i}{\mu} - \integral{K\setminus\gamma_i\inv K}{}{f_i}{\mu}} \\
            &\le \integral{\gamma_i\inv K\triangle K}{}{\norm{f_i}_{\mathrm{L}^\infty}}{\mu} \\
            &= \norm{f_i}_{\mathrm{L}^\infty} \cdot \mu(\gamma_i\inv K\triangle K)
    \end{align*}
    
    Soit finalement $\mathscr{F}$ un filtre faiblement de F\o{}lner pour $\Gamma$. L'estimation précédente assure alors que
    la fonction $K\mapsto\frac1{\mu(K)} \integral{K}{}{(f_i(\gamma_i\inv x) - f_i(x))}{\mu(x)}$ converge vers $0$
    selon $\mathscr{F}$, et ce pour chaque $i\in I$. Il suffit enfin de prendre la limite selon $\mathscr{F}$ des inégalités \ref{amenable_of_Folner_eq2}
    pour obtenir $-M\le c\le M$. 
\end{proof}

La condition de F\o{}lner est très intéressante pour exprimer la moyennabilité de groupes discrets en termes combinatoires. Elle permet ainsi 
de lier la moyennabilité à la \emph{croissance} d'un groupe finiment engendré. %\TODO{Dire quelques mots de plus ?} 

Cependant, pour établir la théorie, il sera utile de travailler avec une condition un peu plus flexible. Pour voir cela, reprenons une dernière fois la preuve
précédente. Plutôt que de prendre une moyenne uniforme des inégalités \ref{amenable_of_Folner_eq1} sur un compact, observons ce qui se passe 
lorsqu'on considère une moyenne pondérée par une fonction $\varphi\in\mathscr{L}^1(\Gamma, \mu)$, positive et de masse $1$ (le cas déjà traité correspondant à $\varphi=\frac1{\mu(K)}\indic_K$).
On obtient alors:
\begin{equation*}\label{intuition_Reiter_eq1}
    -M \le c + \sum_i \integral{}{}{\varphi(x)\left(f_i(\gamma_i\inv x) - f_i(x)\right)}{\mu(x)} \le M
\end{equation*}
Pour $i$ fixé, on a alors :
\begin{align*}
    \abs{\integral{}{}{\varphi(x)(f_i(\gamma_i\inv x) - f_i(x))}{\mu(x)}} 
        &= \abs{\integral{}{}{\varphi(\gamma_i x)f_i(x)}{\mu(x)} - \integral{}{}{\varphi(x) f_i(x)}{\mu(x)}} \\
        &= \abs{\integral{}{}{f_i\cdot\left(\lambda(\gamma_i\inv)(\varphi)-\varphi\right)}{\mu}} \\
        &\le \norm{f_i\cdot\left(\lambda(\gamma_i\inv)(\varphi)-\varphi\right)}_{\mathrm{L}^1} \\
        &\le \norm{f_i}_{\mathrm{L}^\infty} \norm{\lambda(\gamma_i\inv)(\varphi)-\varphi}_{\mathrm{L}^1}
\end{align*}
Pour pouvoir conclure, il faudrait donc cette fois pouvoir faire tendre $\norm{\lambda(\gamma_i\inv)(\varphi)-\varphi}_{\mathrm{L}^1}$ vers $0$, ce qui motive les définitions suivantes.\\

Notons $\mathscr{L}^1(\Gamma, \mu)_{1,+}$ l'ensemble convexe des $f\in\mathscr{L}^1(\Gamma)$ positives et de masse $1$, et $\mathrm{L}^1(\Gamma, \mu)_{1, +}$ son image dans 
$\mathrm{L}^1(\Gamma)$. On s'intéresse à l'\emph{application de Reiter} $\mathfrak{r} : \mathrm{L}^1(\Gamma, \mu)_{1, +}\to\mathcal{F}(\Gamma, \R_+)$, définie par
par $\mathfrak{r}(\varphi)(\gamma) = \norm{\lambda(\gamma\inv)(\varphi)-\varphi}_{\mathrm{L}^1}$.

Un filtre $\mathscr{F}$ sur $\mathrm{L}^1(\Gamma, \mu)_{1, +}$ est dit \emph{faiblement de Reiter} (resp. \emph{fortement de Reiter}) s'il est non-trivial et si le filtre $\mathfrak{r}_*\mathscr{F}$ converge vers $0$ pour la 
topologie de la convergence simple (resp. uniforme sur les compacts) sur $\mathcal{F}(\Gamma, \R_+)$. 
Une suite $\varphi:\N\to\mathrm{L}^1(\Gamma, \mu)_{1, +}$ est \emph{faiblement de Reiter} (resp \emph{fortement de Reiter}) si le filtre $\varphi_*\nhds_{+\infty}^\N$ est faiblement (resp. fortement) de Reiter. Comme $\mathfrak{r}_*\varphi_*\nhds_{+\infty}^\N = (\mathfrak{r}\comp \varphi)_*\nhds_{+\infty}^\N$,
une suite $\varphi:\N\to\mathrm{L}^1(\Gamma, \mu)_{1, +}$ est donc faiblement (resp. fortement) de Reiter \ssi la suite de fonctions
$n\mapsto\left(\gamma\mapsto\norm{\lambda(\gamma_i\inv)(\varphi)-\varphi}_{\mathrm{L}^1}\right)$ converge vers $0$ simplement 
(resp. uniformément sur tout compact).

On a aussi un analogue du lemme \ref{Folner_filter_of_cond}, qui se prouve de manière similaire.

\begin{lemma}\label{Reiter_filter_of_cond}
    $\Gamma$ admet un filtre faiblement de Reiter \ssi il satisfait la 
    \emph{condition de Reiter faible} :
    \begin{equation}\label{weak_Reiter_cond}\tag{WR}
        \forall\varepsilon>0, \forall S\in\finparts(\Gamma), \exists \varphi\in\mathscr{L}^1(\Gamma, \mu)_{1,+}, \forall \gamma\in S,
        \norm{\lambda(\gamma\inv)(\varphi)-\varphi}_{\mathrm{L}^1}<\varepsilon
    \end{equation}
    Il admet de un filtre fortement de Reiter \ssi il satisfait la \emph{condition de Reiter forte} : 
    \begin{equation}\label{strong_Reiter_cond}\tag{SR}
        \forall\varepsilon>0, \forall A\in\mathfrak{K}(\Gamma), \exists\varphi\in\mathscr{L}^1(\Gamma, \mu)_{1,+}, \forall\gamma\in A, 
        \norm{\lambda(\gamma\inv)(\varphi)-\varphi}_{\mathrm{L}^1}<\varepsilon
    \end{equation}

    %\TODO{Dans quel cas a-t-on des suites ?}
    %Si de plus $\Gamma$ est dénombrable, cette condition est équivalent à l'existence d'une suite de F\o{}lner.
\end{lemma}

Comme prévu, tout filtre de F\o{}lner fort (resp. faible) induit un filtre de Reiter fort (resp. faible). En effet, pour $K\in\mathfrak{K}_+(\Gamma)$, l'application 
$\chi_K:=\frac1{\mu(K)}\indic_K$ appartient à $\mathscr{L}^1(\Gamma, \mu)_{1,+}$, et on a $\mathfrak{r}(\chi_K) = \mathfrak{f}(K)$.
Si $\mathscr{F}$ est un filtre de F\o{}lner fort (resp. faible), $\chi_*\mathscr{F}$ est donc un filtre de Reiter fort (resp. faible).

\begin{remark}
    On verra au théorème \ref{amenable_TFAE} que les deux conditions de Reiter sont équivalentes entre elles et avec les deux condition de 
    F\o{}lner. Comme une suite de F\o{}lner induit une suite de Reiter, on pourra déduire du lemme \ref{Folner_filter_of_cond} que \eqref{weak_Reiter_cond} implique l'existence d'une suite de Reiter 
    forte (donc faible) à condition que $\Gamma$ soit $\sigma$-compact.
\end{remark}

Avant d'en arriver au résultat crucial de cette partie, introduisons une dernière notion. Une moyenne $m$ sur $\Gamma$ est dite 
\emph{topologiquement invariante (à gauche)} si :
\begin{equation*}
    \forall f\in\mathrm{L}^\infty(\Gamma), \forall\varphi\in\mathrm{L}^1(\Gamma, \mu)_{1, +}, m(\varphi*f) = m(f)
\end{equation*}
\TODO{Renvoi vers propriétés et def de la convolution}

Le théorème suivant assure que toutes les notions définies dans cette partie sont en fait équivalentes à la moyennabilité ! À travers la condition de Reiter, il nous 
permettra de caractériser la moyennabilité par la théorie des représentations dans la partie suivante.

\begin{theorem}\label{amenable_TFAE}
    Soit $\Gamma$ un groupe topologique séparé et localement compact. Les assertions suivantes sont équivalentes :
    \begin{enumerate}[(i)]
        \item $\Gamma$ est moyennable \label{amenable_TFAE/amenable}
        \item Il existe une moyenne topologiquement invariante sur $\Gamma$ \label{amenable_TFAE/topological_mean}
        \item $\Gamma$ satisfait la condition de Reiter forte \label{amenable_TFAE/strong_Reiter} \eqref{strong_Reiter_cond}
        \item $\Gamma$ satisfait la condition de F\o{}lner forte \label{amenable_TFAE/strong_Folner} \eqref{strong_Folner_cond}
        \item $\Gamma$ satisfait la condition de F\o{}lner faible \label{amenable_TFAE/weak_Folner} \eqref{weak_Folner_cond}
        \item $\Gamma$ satisfait la condition de Reiter faible \label{amenable_TFAE/weak_Reiter} \eqref{weak_Reiter_cond}
    \end{enumerate}
\end{theorem}

La preuve nécessite un certain nombre de résultats et définitions préliminaires, que nous détaillons maintenant. 

\TODO{Dans les sections suivantes, ajouter un peu de motivation}

\subsection{L'espace \texorpdfstring{$UC_b(\Gamma_d)$}{des fonctions uniformément continues bornées}}

Nous notons $UC_b(\Gamma_d)$ l'ensemble des fonctions $f:\Gamma_d\to\C$ qui sont bornées et uniformément 
continues, muni de la norme uniforme. Dans cette définition et dans toute la suite, $\Gamma_d$ désigne l'espace uniforme obtenu 
en munissant le groupe topologique $\Gamma$ de sa structure uniforme \emph{droite}\footnote{On noterait de même $\Gamma_s$
le groupe topologique $\Gamma$ muni de sa structure uniforme \emph{gauche}.}.
Rappelons que cette structure uniforme, dont nous noterons $\mathcal{U}\Gamma_d$ le filtre des entourages,
est définie par l'égalité $\mathcal{U}\Gamma_d=(\divop_d)^*\nhds_1^\Gamma$, pour 
$\fundef{\divop_d}{\Gamma\times\Gamma&\to&\Gamma}{(x, y)&\mapsto&x y\inv}$\footnote{Dans le cas de la structure 
uniforme gauche, on remplaçerait $\divop_d$ par $\fundef{\divop_s}{\Gamma\times\Gamma&\to&\Gamma}{(x, y)&\mapsto&x\inv y}$.}.
Rappelons aussi que la structure uniforme associée à l'espace métrique $(X, d)$ est définie par 
$\mathcal{U}X = d^*\nhds_0$. 
Pour $f:\Gamma\to\C$ bornée, on a donc : 
\begin{align}
    f\in UC_b(\Gamma_d)
        &\iff \mathcal{U}\Gamma_d \le\footnotemark\ (f\times f)^*\mathcal{U}\C \nonumber \\
        &\iff (\divop_d)^*\nhds_1^\Gamma \le (f\times f)^*d^*\nhds_0 \nonumber \\
        &\iff \forall W\in\nhds_0, \exists V\in\nhds_1^\Gamma, \set{(x, y)\tq x y\inv\in V} \subseteq \set{(x, y)\tq \abs{f(x)-f(y)}\in W} \nonumber \\
        &\iff \forall \varepsilon>0, \exists V\in\nhds_1^\Gamma, \forall y\in\Gamma, \forall g\in V, \abs{f(gy)-f(y)} < \varepsilon \nonumber \\
        &\iff \forall \varepsilon>0, \exists V\in\nhds_1^\Gamma, \forall g\in V, \norm{\lambda(g)(f) - f}_\infty < \varepsilon \label{UCB_explicit_cara}
\end{align}
Où on a pu remplacer $\lambda(g\inv)$ par $\lambda(g)$ en remplaçant $V$ par son symétrique. Cela permet de donner la caractérisation suivante de $UC_b(\Gamma_d)$.
\footnotetext{Si $\mathscr{F}, \mathscr{G}$ sont deux filtres sur un ensemble $X$,
on note $\mathscr{F}\le \mathscr{G}$ si $\mathscr{F}$ est \emph{plus fin que} $\mathscr{G}$,
c'est à dire si $\mathscr{G}\subseteq \mathscr{F}$. } 

\begin{lemma}\label{UCB_iff_translate}
    Soit $f\in\ell^\infty(\Gamma)$\footnote{i.e $f$ est bornée \emph{partout}}. $f$ appartient à $UC_b(\Gamma_d)$ \ssi l'application 
    $\fundef{\ev_f\comp\lambda}{\Gamma&\to&\ell^\infty(\Gamma)}{g&\mapsto&\lambda(g)(f)}$
    est continue.
\end{lemma}

On donne une démonstration directe en se basant sur la caractérisation $(\ref{UCB_explicit_cara})$, mais on 
pourrait aussi utiliser le théorème \ref{uniform_continuous_iff_postcomp}, plus général, démontré en annexe.

\begin{proof}
    Remarquons d'abord que $\ev_f\comp\lambda$ est continue \ssi elle est continue en $1$. Le sens direct est immédiat, supposons donc 
    la continuité en $1$, et montrons la continuité en $x\in\Gamma$ quelconque. Soit donc $\varepsilon>0$. La continuité en $1$ fournit 
    $V\in\nhds_1$ tel que $\forall g\in V, \norm{\lambda(g)(f) - f}_\infty<\varepsilon$. Notons alors que :
    \begin{equation*}
        \forall g\in V, \norm{\lambda(xg)(f) - \lambda(x)(f)}_\infty = \norm{\lambda(x)(\lambda(g)(f) - f)}_\infty = \norm{\lambda(g)(f) - f}_\infty < \varepsilon
    \end{equation*}
    Comme $xV\in\nhds_x$ cela conclut, puisqu'on a $\forall y\in xV, \norm{\lambda(y)(f) - \lambda(x)(f)}_\infty < \varepsilon$.

    Or, on remarque que la caractérisation \ref{UCB_explicit_cara} de $UC_b(\Gamma_d)$ exprime précisément la continuité en $1$ de l'application $\ev_f\comp\lambda$.
    Cela conclut la preuve du lemme.
\end{proof}

Maintenant qu'on dispose de plusieurs caractérisations agréables, vérifions que $UC_b(\Gamma_d)$ est bien un espace de Banach.

\begin{proposition}\label{UCB_complete}
    $UC_b(\Gamma_d)$ est fermé dans $\ell^\infty(\Gamma)$, donc complet.
\end{proposition}

Encore une fois, on donne une preuve concrète basée sur le lemme \ref{UCB_iff_translate}. 
%, et on renvoie au théorème \TODO{\ref{uniform_limit_of_uniform_cont}} en annexe pour le résultat général.

\begin{proof}
    Soit $F:\N\to UC_b(\Gamma_d)$ une suite convergente dans $\ell^\infty(\Gamma)$, de limite $f$.
    Il s'agit de montrer que $f$ est uniformément continue. Notons que, pour tout $g\in\Gamma$, on a $\norm{\lambda(g)(f) - \lambda(g)(F_n)}_\infty = \norm{f - F_n}_\infty$.
    Or $\norm{f - F_n}_\infty\xrightarrow[n\to+\infty]{}0$, donc on a montré que la suite de fonctions $\ev_{F_n}\comp\lambda$ converge 
    uniformément vers $\ev_f\comp\lambda$. Chaque $\ev_{F_n}\comp\lambda$ étant continue en vertu du lemme \ref{UCB_iff_translate},
    sa limite l'est également, et une deuxième application du lemme \ref{UCB_iff_translate} conclut.
\end{proof}

Terminons par le résultat suivant, qui justifie que l'on s'intéresse à cet espace.

\begin{proposition}\label{conv_UCB}
    La convolée $\varphi\ast f$ de $\varphi\in\mathscr{L}^1(\Gamma)$ et $f\in\mathscr{L}^\infty(\Gamma)$, qui existe et appartient à $\ell^\infty(\Gamma)$ par l'inégalité de Young (\TODO{ref}),
    appartient en fait à $UC_b(\Gamma_d)$.

    Par conséquent, l'opérateur de convolution $\ast:\mathrm{L}^1(\Gamma)\times\mathrm{L}^\infty(\Gamma)\to\mathrm{L}^\infty(\Gamma)$
    se factorise à travers $UC_b(\Gamma_d)\hookrightarrow\mathrm{L}^\infty(\Gamma)$ en 
    $\ast:\mathrm{L}^1(\Gamma)\times\mathrm{L}^\infty(\Gamma)\to UC_b(\Gamma_d)$ encore bilinéaire continue de norme inférieure à $1$.
\end{proposition}

Il est clair que le deuxième point découle du premier puisque les normes $\norm{\cdot}_\infty$ et $\norm{\cdot}_{\mathrm{L}^\infty}$ coïncident 
sur $UC_b(\Gamma_d)$, de sorte que la composée $UC_b(\Gamma_d)\hookrightarrow\mathscr{L}^\infty(\Gamma)\twoheadrightarrow\mathrm{L}^\infty(\Gamma)$
est encore un plongement isométrique. Dorénavant, on identifiera d'ailleurs $UC_b(\Gamma_d)$ à son image dans $\mathrm{L}^\infty(\Gamma)$.

Il s'agit donc de montrer le premier point.

\begin{proof}
    On utilise la caractérisation $(\ref{UCB_explicit_cara})$. Soit donc $\varepsilon>0$. La continuité de la représentation régulière $\lambda$ de $\Gamma$ sur $\mathrm{L}^1(\Gamma)$ %(\TODO{faut-il la remontrer ? Raisonnement par densité}) 
    assure que 
    $\fundef{\ev_\varphi\comp\lambda}{\Gamma&\to&\mathrm{L}^1(\Gamma)}{g&\mapsto&\lambda(g)(\varphi)}$ est continue, ce qui fournit 
    un voisinage $V\in\nhds_1^\Gamma$ tel que $\forall g\in V, \norm{\lambda(g)(\varphi) - \varphi}_1<\frac{\varepsilon}{\norm{f}_{\mathrm{L}^\infty}}$. Or, pour $g, x\in \Gamma$ quelconques, on a :
    \begin{align*}
        \lambda(g)(\varphi\ast f)(x)
            &= (\varphi\ast f)(g\inv x) \\
            &= \integral{}{}{\varphi(h)f(h\inv g\inv x)}{\mu(h)} \\
            &= \integral{}{}{\varphi(g\inv h)f(h\inv x)}{\mu(h)} \\
            &= (\lambda(g)(\varphi)\ast f)(x)
    \end{align*}
    Pour $g\in V$, on a finalement :
    \begin{align*}
        \norm{\lambda(g)(\varphi\ast f) - \varphi\ast f}_\infty 
            &= \norm{(\lambda(g)(\varphi) - \varphi)\ast f}_\infty \\
            &\le \norm{\lambda(g)(\varphi) - \varphi}_1\norm{f}_{\mathrm{L}^\infty} \\
            &< \varepsilon
    \end{align*}
    Cela conclut.
\end{proof}

%\begin{proof}
%    Soit $f\in\ell^\infty(\Gamma)$ adhérent à $UC_b(\Gamma_d)$. Il s'agit de montrer que $f$ satisfait $(\ref{UCB_explicit_cara})$.
%    Soit donc $\varepsilon>0$. Par hypothèse, il existe une fonction $\widetilde{f}\in UC_b(\Gamma_d)$ avec 
%    $\norm{f-\widetilde{f}}_\infty<\frac{\varepsilon}{3}$. Or $\widetilde{f}$ vérifie $(\ref{UCB_explicit_cara})$, ce qui fournit $V\in\nhds_1^\Gamma$
%    tel que $\forall g\in V, \norm{\lambda(g)(\widetilde{f}) - \widetilde{f}}<\frac{\varepsilon}{3}$. Pour $g\in V$, on a donc 
%    $\lambda$
%\end{proof}

\subsection{Moyennes provenant de fonctions intégrables}

Rappelons que, sauf mention explicite du contraire, tous les espaces $\mathrm{L}^p$ sont complexes, et
si $E$ est un espace vectoriel topologique, $E^*$ désigne son dual topologique \emph{sur le corps $\C$}. Si
$E$ est un espace vectoriel complexe, nous noterons $\floor{E}$ l'espace vectoriel réel sous-jacent.

Rappelons aussi que, grâce à la définition choisie de $\mathrm{L}^\infty$ (\TODO{ref}), l'isomorphisme $(\mathrm{L}^1)^* \simeq \mathrm{L}^\infty$
est valable sur \emph{tout} espace mesuré, et non plus seulement dans le cas $\sigma$-fini (\TODO{ref}).

Notons $\mathcal{M}(X,\mu)$ l'ensemble des moyennes sur un espaces mesuré $(X,\mathcal{A},\mu)$, telles que définies après la définition \ref{amenable_def}.
Dans le cas d'un groupe localement compact et séparé $\Gamma$ muni d'une mesure de Haar $\mu$, nous noterons
$\mathcal{M}(\Gamma):=\mathcal{M}(\Gamma, \mu)$\footnote{Rappelons que cet ensemble ne dépend pas du choix de $\mu$.}. Notons déjà que cet ensemble est
$\ast$-faiblement compact. En effet, le théorème de Banach-Alaoglu assure que la boule unité fermée $\mathbb{B}$
de $\left(\mathrm{L}^\infty(X, \mu)\right)^*$ est compacte pour la topologie faible-$*$ $\sigma({\mathrm{L}^\infty}^*, \mathrm{L}^\infty)$,
et $\mathcal{M}(X, \mu)$ n'est autre que l'ensemble des $m\in\mathbb{B}$ telles que $m(1) = 1$ (lemme \ref{positive_iff_norm}),
qui est $*$-faiblement fermé dans $\mathbb{B}$ et donc $*$-faiblement compact.

Considérons désormais $X$ un espace topologique séparé et localement compact muni d'une mesure de Radon $\mu$, et 
étudions désormais le sous-ensemble des moyennes provenant de fonctions intégrables \emph{via} le plongement isométrique usuel $\mathrm{L}^1(X,\mu)\hookrightarrow\mathrm{L}^\infty(X,\mu)^*$.
Notons déjà que ce plongement se restreint en $\mathrm{L}^1(X, \mu)_{1, +}\hookrightarrow \mathcal{M}(X, \mu)$.
En effet, il est clair que, pour $f\in\mathscr{L}^1_{1, +}(X, \mu)$, la forme linéaire $\funlam{\mathscr{L}^\infty(X,\mu)&\to&\C}{ \varphi &\mapsto&\integral{}{}{f\varphi}{\mu}}$ est une moyenne,
car elle est positive et $\integral{}{}{f\cdot 1}{\mu} = \integral{}{}{f}{\mu} = 1$.

Le résultat principal quant au plongement $\mathrm{L}^1(X, \mu)_{1, +}\hookrightarrow \mathcal{M}(X, \mu)$ est que son image est $*$-faiblement dense.

\begin{theorem}\label{dense_in_means}
    Soit $X$ espace topologique séparé et localement compact muni d'une mesure de Radon $\mu$.

    Munissons l'espace $\mathrm{L}^1(X,\mu)$ de la topologie faible $\sigma(\mathrm{L}^1, \mathrm{L}^\infty) = \sigma\left(\mathrm{L}^1, {\mathrm{L}^1}^*\right)$
    \footnote{Cette égalité provient de ce que ${\mathrm{L}^1}^*\simeq\mathrm{L}^\infty$}, 
    l'espace $\mathrm{L}^\infty(X,\mu)^*$ de la topologie faible-$*$ $\sigma({\mathrm{L}^\infty}^*, \mathrm{L}^\infty)$,
    et les parties $\mathrm{L}^1_{1, +}(X, \mu)$, $\mathcal{M}(X, \mu)$ des topologies induites respectives.

    L'application $\iota:\mathrm{L}^1(X,\mu)\hookrightarrow\mathrm{L}^\infty(X,\mu)^*$ est alors un plongement d'espaces 
    vectoriels topologiques, et le plongement topologique induit $\widetilde{\iota}:\mathrm{L}^1_{1, +}(X, \mu)\hookrightarrow \mathcal{M}(X, \mu)$
    est d'image dense. 
\end{theorem}

\begin{proof}
    Notons que, pour toutes $f\in\mathscr{L}^1(X, \mu)$, $\varphi\in\mathscr{L}^\infty(X, \mu)$, on a, pour les dualités usuelles 
    $({\mathrm{L}^\infty}^*, \mathrm{L}^\infty)$ et $(\mathrm{L}^1, \mathrm{L}^\infty)$ :
    \begin{equation*}
        \ket{\iota(f), \varphi} = \iota(f)(\varphi) = \integral{}{}{f\varphi}{\mu} = \ket{f, \varphi}
    \end{equation*}
    Cela assure que $\iota$ est effectivement un plongement topologique pour les topologies faibles associées à ces dualités,
    et donc un plongement d'espaces vectoriels topologiques par linéarité.

    La partie importante est donc la densité, qui va reposer crucialement sur la convexité de $\mathrm{L}^1_{1, +}(X, \mu)$ et sur le théorème de séparation de Hahn-Banach.
    Notre objectif est de montrer l'inclusion $\mathcal{M}(X, \mu)\subseteq\closure{\iota(\mathrm{L}^1_{1, +}(X, \mu))}$ (l'adhérence étant prise dans 
    ${\mathrm{L}^\infty(X,\mu)}^*$ pour la topologie faible-$*$). On suppose donc que ce n'est pas le cas, ce qui fournit
    $m\in\mathcal{M}(X, \mu)$ tel que $m\notin\closure{\iota(\mathrm{L}^1_{1, +}(X, \mu))}$. Appliquons alors le théorème de séparation de
    Hahn-Banach au compact (séparé) convexe $\set{m}$ et au fermé convexe $\closure{\iota(\mathrm{L}^1_{1, +}(X, \mu))}$
    de l'espace localement convexe \emph{réel} $\floor{\mathrm{L}^\infty(X,\mu)^*}$
    \footnote{L'espace vectoriel réel sous-jacent à un espace localement convexe complexe
    est par définition localement convexe. } 
    sous-jacent à l'espace localement convexe \emph{complexe} $\mathrm{L}^\infty(X,\mu)^*$
    \footnote{Toute topologie faible est localement convexe, comme topologie initiale 
    pour une famille d'applications linéaires à valeurs dans l'espace localement convexe $\K$, où $\K\in\set{\R, \C}$ 
    désigne le corps des scalaires. }. 
    Cela fournit une forme $\R$-linéaire continue $f : \floor{\mathrm{L}^\infty(X,\mu)^*} \to \R$ et un réel 
    $a$ tel que $f(m)<a$ et $\forall x\in \closure{\iota(\mathrm{L}^1_{1, +}(X, \mu))}, f(x)>a$. 
    %En particulier, par le théorème des 
    %valeurs intermédiaires, $a = f(\alpha)$ pour un certain $\alpha\in\mathrm{L}^\infty(X,\mu)^*$.

    Écrivons alors $f=\Re g$ où $\fundef{g}{\mathrm{L}^\infty(X,\mu)^* &\to& \C}{x&\mapsto&f(x)-\i f(\i x)}$ est $\C$-linéaire et continue, et 
    supposons que $g$ est de la forme $\ev_\varphi$ pour $\varphi\in\mathrm{L}^\infty(X, \mu)$. On a alors, pour $u\in\mathrm{L}^1_{1, +}(X, \mu)$ :
\begin{align*}
    \ket{u, \Re\varphi - a} 
        &= \Re\ket{u, \varphi - a}\footnotemark \\
        &= \Re\ket{\iota(u), \varphi} - \ket{u, a} \\
        &= \Re g(\iota(u)) - a\ket{u, 1} \\
        &= f(\iota(u)) - a \\
        &\ge 0
\end{align*}
\footnotetext{Cette égalité provient de ce que $u$ est à valeurs réelles.}Tout élément positif de $\mathrm{L}^1(X, \mu)$ étant nul ou multiple positif d'un élément de $\mathrm{L}^1_{1, +}(X, \mu)$
\footnote{Plus précisément, si $u\in\mathrm{L}^1(X,\mu)$ est positif et non nul, on a $\integral{}{}{u}{\mu}=\norm{u}_{\mathrm{L}^1}\ne 0$. On peut alors poser 
$\widetilde{u}:=\frac1{\integral{}{}{u}{\mu}}\cdot u\in\mathrm{L}^1_{1, +}(X, \mu)$, de sorte que
$u = \left(\integral{}{}{u}{\mu}\right)\cdot\widetilde{u}$.}, cette inégalité s'étend à tout $u\in\mathrm{L}^1(X, \mu)$ positif. 
Autrement dit, $\Re\varphi-a$ est positive \emph{lorsqu'elle est vue comme forme linéaire sur $\mathrm{L}^1(X, \mu)$}. Si l'on admet
que $\Re\varphi - a$ est en fait positive en tant qu'élément de $\mathrm{L}^\infty(X, \mu)$, on obtient directement une contradiction avec la positivité de $m$, puisque : 
\begin{align*}
    m(\Re\varphi - a) 
        &= m(\varphi - a) - \i m(\Im\varphi) \\
        &= \Re m(\varphi - a)\footnotemark \\
        &= \Re g(m) - a \\
        &= f(m) - a \\
        &< 0
\end{align*}
\footnotetext{Comme $\Im\varphi$ est à valeurs réelles, la positivité de $m$ assure que $\i m(\Im\varphi)\in \i\R$.}

Pour conclure, il s'agit de montrer que les deux hypothèses que nous avons faites au cours de la preuve sont vérifiées. C'est l'objet 
des deux lemmes suivants.

\begin{lemma}\label{dual_of_weak}
    Soit $E, F$ deux $\K$-espaces vectoriels ($\K\in\set{\R, \C}$) en dualité, et supposons $E$ muni de la topologie faible $\sigma(E, F)$ associée à cette dualité. 
    Alors toute forme linéaire continue sur $E$ est de la forme $x\mapsto\ket{x, y}$ pour un certain $y\in F$.
\end{lemma}

\begin{lemma}\label{positive_of_positive_form}
    Soit $\varphi\in\mathscr{L}^\infty(X, \mu)$. Si l'intégrale $\integral{}{}{f\varphi}{\mu}$ est positive 
    pour toute fonction $f\in\mathscr{L}^1(X, \mu)$ positive $\mu$-localement-presque-partout, alors $\varphi$ est 
    positive $\mu$-localement-presque-partout.
\end{lemma}

Le lemme \ref{dual_of_weak} assure ainsi que l'on peut toujours écrire $g=\ev_\varphi$,
tandis que le lemme \ref{positive_of_positive_form} assure la positivité de $\Re\varphi - a$, ce qui conclut la preuve.
\end{proof}

\begin{proof}[Démonstration du lemme \ref{dual_of_weak}]
    Soit $\varphi$ une forme linéaire continue sur $E$. La continuité assure que l'ensemble $\mathbf{U} := \varphi\inv(B(0, 1))$ est un voisinage
    de $0$ pour $\sigma(E, F)$, ce qui fournit\footnote{Car la topologie $\sigma(E, F)$ est initiale 
    pour la famille des applications $\ket{\blank, y}$, où $y$ parcourt $F$.} $\varepsilon>0$, $I$ fini et $f : I\to F$ tels que
    $\mathbf{V} := \bigcap_{i\in I} \set{x\in E\tq \abs{\ket{x, f_i}}<\varepsilon}\subseteq\mathbf{U}$. 
    
    Pour chaque $i\in I$, notons donc $\psi_i$ la forme linéaire $\ket{\blank, f_i}$. Si $x\in\bigcap_{i\in I}\ker\psi_i$, on a bien sûr 
    $x\in\mathbf{V}\subseteq\mathbf{U}$ donc $\abs{\varphi(x)}<1$. Mais on a aussi $\forall t\in\R_+, tx\in\bigcap_{i\in I}\ker\psi_i$, d'où 
    $\forall t\in\R_+^*, \abs{\varphi(x)}<\frac1t$, d'où $x\in\ker\varphi$ en faisant tendre $t$ vers $0$. 
    On a donc montré $\bigcap_{i\in I}\ker\psi_i\subseteq\ker\varphi$. Pour conclure, on va s'appuyer sur un lemme usuel d'algèbre linéaire, dont nous redonnons
    une preuve ci-dessous.
    \begin{lemma}\label{span_dual_of_ker}
        Soient $k$ un corps, $E$ un $k$-espace vectoriel, $I$ un ensemble fini et $\psi : I\to E^\sharp$
        \footnote{On note $E^\sharp$ le dual algébrique de $E$.} une famille de formes linéaires sur $E$.
        Alors le sous-espace vectoriel de $E^\sharp$ engendré par l'image de $\psi$ est exactement l'ensemble 
        des formes linéaires $\varphi$ telles que $\ker\varphi\supseteq\bigcap_{i\in I}\ker\psi_i$.
    \end{lemma}

    Ce lemme fournit  $a : I\to\K$ telle que $\varphi = \sum_{i\in I} a_i\psi_i = \ket{\blank, \sum_{i\in I} a_i f_i}$,
    ce qui achève la preuve du lemme \ref{dual_of_weak}.
\end{proof}

\begin{proof}
    Il est clair que toute combinaison linéaire des $\psi_i$ est nulle sur l'ensemble $\bigcap_{i\in I}\ker\psi_i$.
    Il s'agit donc de montrer que toute forme linéaire $\varphi$ vérifiant $\ker\varphi\supseteq\bigcap_{i\in I}\ker\psi_i$ est effectivement 
    combinaison linéaire des $\psi_i$. 
    
    Soit donc une telle forme $\varphi$. On considère l'application linéaire $\Psi : E\to k^I$ 
    dont les composantes sont les $\psi_i$, $F\subseteq k^I$ son image, et $\widetilde{\Psi}:E\to F$ l'application surjective induite.
    On a $\ker\widetilde{\Psi} = \ker\Psi = \bigcap_{i\in I}\ker\psi_i \subseteq \ker \varphi$, donc $\varphi$ se factorise 
    par $\widetilde{\Psi}$ en $\widetilde{\varphi}:F\to k$, que l'on prolonge en $\widehat{\varphi}:k^I\to k$ de manière arbitraire.
    En notant $e : I\to\C^I$ la base canonique et $e^*$ sa base duale, on a donc $\widehat{\varphi} = \sum_{i\in I} a_i e^*_i$
    pour une certaine famille $a : I\to\C$. Il vient enfin
    $\varphi = \widetilde{\varphi}\comp\widetilde{\Psi} = \widehat{\varphi}\comp\Psi = \sum_{i\in I} a_i (e^*_i \comp\Psi) = \sum_{i\in I} a_i\psi_i$,
    ce qui conclut.
\end{proof}

\begin{proof}[Démonstration du lemme \ref{positive_of_positive_form}]
    Supposons que l'ensemble $\mathbf{S} := \set{x\in X\tq\varphi(x)<0}$ n'est \emph{pas} $\mu$-localement-négligeable. Par définition, cela fournit 
    $\mathbf{T}\in\Bor(X)$ de mesure finie tel que $0<\mu(\mathbf{T}\cap\mathbf{S})<+\infty$, de sorte que $\indic_{\mathbf{T}\cap\mathbf{S}}$ est positive et intégrable. L'intégrale
    $\integral{}{}{\indic_{\mathbf{T}\cap\mathbf{S}}\varphi}{\mu}$ est donc positive par hypothèse, et négative comme intégrale d'une fonction négative,
    donc nulle. La fonction positive $-\indic_{\mathbf{T}\cap\mathbf{S}}\varphi$ est d'intégrale nulle, 
    l'inégalité de Markov \ref{markov_and_consequence} assure donc qu'elle est nulle presque partout, ce qui contredit le fait qu'elle est non-nulle 
    sur l'ensemble $\mathbf{T}\cap\mathbf{S}$ de mesure non-nulle.
\end{proof}

Pour conclure cette partie, nous allons donner une autre conséquence du lemme \ref{dual_of_weak} qui servira 
à montrer le théorème \ref{amenable_TFAE}.

\begin{proposition}\label{weak_convex_closed}
    Soit $(E, \mathcal{T})$ un $\K$-espace vectoriel topologique localement convexe, où $\K\in\set{\R, \C}$. Les ensembles convexes fermés pour la topologie 
    $\sigma(E, E^*)$ sont exactement les ensembles convexes fermés pour $\mathcal{T}$.
\end{proposition}

\begin{proof}
    On dit qu'une partie de $E$ est un \emph{demi-espace fermé} si elle s'écrit sous la forme $\set{x\in E\tq \Re f(x)\leq a}$ pour 
    $f\in E^*$ et $a\in\R$.
    Le lemme \ref{dual_of_weak} assure que $\sigma(E^*, E)$ et $\mathcal{T}$ définissent le même ensemble de formes linéaires continues,
    et donc les mêmes demi-espaces fermés. Ces deux topologies étant localement convexes, la proposition découle donc du lemme suivant.
\end{proof}

\begin{lemma}
    Soit $E$ un $\K$-espace vectoriel topologique localement convexe, où $\K\in\set{\R, \C}$. Toute partie 
    convexe fermée de $E$ est l'intersection des demi-espaces fermés qui le contiennent.
\end{lemma}

\begin{proof}
    Soit $F$ un convexe fermé, il est clair que $F$ est contenu dans l'intersection des demi-espaces fermés qui le contiennent.
    Soit donc $x\notin F$, et montrons qu'il existe au moins un demi-espace fermé contenant $F$ mais pas $x$. 
    Pour $\K=\R$ c'est une conséquence immédiate du théorème de séparation de Hahn-Banach. Pour $\K=\C$
    c'est le cas aussi, mais il faut un petit argument supplémentaire\footnote{Sauf si on utilise les versions complexes 
    du théorème de séparation, mais on préfère ici se cantonner au cas réel, plus standard.}. Le théorème 
    de séparation \emph{réel} fournit toujours une forme $\R$-linéaire continue $f:\floor{E}\to\R$ et un réel $a$ tel que $f(x)>a$
    et $\forall y\in F, f(y)<a$. Il suffit alors d'écrire $f = \Re g$ où $\fundef{g}{\mathrm{L}^\infty(X,\mu)^* &\to& \C}{x&\mapsto&f(x)-\i f(\i x)}$ est $\C$-linéaire et continue,
    et le demi-espace fermé associé à $g$ et $a$ contient bien $F$ mais pas $x$. 
\end{proof}

%,
%cette dernière condition pouvant se réécrire : 
%\begin{equation}\label{cara_left_uniform_continuous}
%    \forall\varepsilon>0,\set{g\in\Gamma\tq\forall x\in\Gamma, \abs{f(gx)-f(x)}<\varepsilon}\in\nhds_1
%\end{equation}
%De manière équivalente, $f\in\mathscr{L}^\infty(\Gamma)$ appartient à $UC_b(\Gamma)$ \ssi l'application $\fundef{\ev_f\comp\lambda}{\Gamma&\to&\mathrm{L}^\infty(\Gamma)}{g&\mapsto&\lambda(g)(f)}$ est continue en $1$
%\footnote{
%    En effet, le symmétrique d'un voisinage de $1$ étant voisinage de $1$, on a:
%    \begin{align*}
%        (\ref{cara_left_uniform_continuous})
%            &\iff \forall\varepsilon>0,\exists U\in\nhds_1^\Gamma,\forall x\in\Gamma,\abs{f(gx)-f(x)}\le\varepsilon \\
%            &\iff \forall\varepsilon>0,\exists U\in\nhds_1^\Gamma,\norm{\lambda(g\inv)(f) - f}_{\mathrm{L}^\infty}\le\varepsilon \\
%            &\iff \forall\varepsilon>0,\exists U\in\nhds_1^\Gamma,\norm{\lambda(g)(f) - f}_{\mathrm{L}^\infty}\le\varepsilon \\
%            &\iff \ev_f\comp\lambda \text{ continue en 1}
%    \end{align*}
%    De plus, si 
%}.

\subsection{Un lemme d'intégration}

Terminons cette présentation des prérequis du théorème \ref{amenable_TFAE} par un petit lemme d'intégration,
qui permettra de montrer l'implication $\eqref{strong_Reiter_cond}\implies\eqref{strong_Folner_cond}$. \TODO{}

\subsection{Démonstration du théorème \ref{amenable_TFAE}} 

\begin{proof}
    Commençons par montrer \framebox{$(\ref{amenable_TFAE/amenable})\implies(\ref{amenable_TFAE/topological_mean})$}. Soit donc 
    $m$ une moyenne invariante sur $\Gamma$. La première étape est de remarquer que, sans avoir à modifier la moyenne 
    $m$, on a déjà $m(\varphi\ast f) = m(f)$ pour $f\in UC_b(\Gamma_d)$ et $\varphi\in\mathrm{L}^1(\Gamma, \mu)_{1, +}$.
    Notons donc $\widetilde{m}$ la restriction de $m$ à $UC_b(\Gamma_d)$.

    Pour voir cela, fixons $f\in UC_b(\Gamma_d)$ et $\varphi\in\mathscr{L}^1(\Gamma)$, que nous ne supposons pas encore positive normalisée.
    Supposons dans un premier temps que $\varphi$ est continue à support compact. Le lemme \ref{UCB_iff_translate} assure 
    alors que la fonction $\ev_f\comp\lambda:\Gamma\to UC_b(\Gamma_d)\hookrightarrow\ell^\infty(\Gamma)$ est continue,
    de sorte que la fonction $\varphi\cdot(\ev_f\comp\lambda)$, à valeurs dans l'espace de Banach $UC_b(\Gamma_d)$, est continue à support compact.
    Le théorème \ref{strong_measurable_crit} et la remarque qui suit assurent donc que cette fonction est fortement mesurable et intégrable.
    Notons que les applications $\ev_x:UC_b(\Gamma_d)\to\C$ pour $x\in\Gamma$ sont bien sûr des formes linéaires continues, de sorte que $\forall x\in\Gamma$ :
    \begin{align*}
        \left(\integral{}{}{\varphi(g)\lambda(g)(f)}{\mu(g)}\right)(x) 
            &= \ev_x\left(\integral{}{}{\varphi(g)\lambda(g)(f)}{\mu(g)}\right) \\
            &= \integral{}{}{\ev_x(\varphi(g)\lambda(g)(f))}{\mu(g)} \\
            &= \integral{}{}{\varphi(g)f(g\inv x)}{\mu(g)} \\
            &= (\varphi\ast f)(x)
    \end{align*}
    De plus $\varphi\ast f\in UC_b(\Gamma_d)$, et $\widetilde{m}$ est une forme linéaire continue sur cet espace, d'où :
    \begin{align*}
        \widetilde{m}(\varphi\ast f) 
            &= \widetilde{m}\left(\integral{}{}{\varphi(g)\lambda(g)(f)}{\mu(g)}\right) \\
            &= \integral{}{}{\widetilde{m}(\varphi(g)\lambda(g)(f))}{\mu(g)} \\
            &= \integral{}{}{\varphi(g)\widetilde{m}(f)}{\mu(g)} \\
            &= \widetilde{m}(f)\integral{}{}{\varphi}{\mu}
    \end{align*}
    Cette égalité étant valable pour tout $\varphi\in C_c(\Gamma)$, le lemme \ref{cont_supp_compact_dense_Lp} entraîne, par continuité de la convolution, 
    %(\TODO{ref}) 
    de $\widetilde{m}$ et de l'intégration, que l'égalité $\widetilde{m}(\varphi\ast f) = \widetilde{m}(f)\integral{}{}{\varphi}{\mu}$ est valable 
    pour toute $\varphi\in\mathscr{L}^1(\Gamma)$. En particulier, pour $\varphi\in\mathscr{L}^1(\Gamma, \mu)_{1, +}$ (et toujours pour 
    $f\in UC_b(\Gamma_d)$), on a $\widetilde{m}(\varphi\ast f) = \widetilde{m}(f)$.
    
    Revenons maintenant au cas général, et fixons $\mathscr{F}$ un filtre d'approximation de l'unité 
    dans $\mathscr{L}^1(\Gamma, \mu)_{1, +}$, de sorte que pour tout $\varphi\in\mathscr{L}^1(\Gamma, \mu)_{1, +}$, la fonction 
    $\psi\mapsto\varphi\ast\psi$ converge selon $\mathscr{F}$ vers $\varphi$ en norme $\mathrm{L}^1$. 
    Si on se donne de plus $f\in\mathscr{L}^\infty(\Gamma)$, on a donc que la fonction $\psi\mapsto(\varphi\ast\psi)\ast f = \varphi\ast(\psi\ast f)$
    converge selon $\mathscr{F}$ vers $\varphi\ast f$ dans $\mathrm{L}^\infty(\Gamma)$. Mais pour tout $\psi\in\mathscr{L}^1(\Gamma, \mu)_{1, +}$, on
    a $\psi\ast f\in UC_b(\Gamma_d)$ (proposition \ref{conv_UCB}), donc $\widetilde{m}(\varphi\ast\psi\ast f) = \widetilde{m}(\psi\ast f)$,
    de sorte que le nombre $\widetilde{m}(\varphi\ast f)$ est limite selon $\mathscr{F}$ de $\psi\mapsto\widetilde{m}(\psi\ast f)$ et 
    ne dépend donc pas de $\varphi$. On a ainsi montré que $\forall\varphi, \varphi'\in\mathscr{L}^1(\Gamma, \mu)_{1, +}, \widetilde{m}(\varphi\ast f) = \widetilde{m}(\varphi'\ast f)$.

    Choisissons alors $\varphi_0\in\mathscr{L}^1(\Gamma, \mu)_{1, +}$, par exemple en posant $\varphi_0:=\chi_U=\frac{1}{\mu(U)}\indic_U$ pour 
    $U$ voisinage compact de $1$, et posons $\fundef{n}{\mathrm{L}^\infty(\Gamma)&\to&\C}{f&\mapsto&\widetilde{m}(\varphi_0\ast f)}$. Il s'agit bien d'une 
    forme linéaire continue et positive car la convolée de fonctions positives est positive, et on a $n(1)=\widetilde{m}(\varphi_0\ast 1) = \widetilde{m}\left(\left(\integral{}{}{\varphi_0}{\mu}\right)\cdot 1\right) = \widetilde{m}(1) = 1$.
    Reste à vérifier que la moyenne $n$ est bien topologiquement invariante. Soit donc $\varphi\in\mathscr{L}^1(\Gamma, \mu)_{1, +}$ quelconque, et calculons :
    \begin{align*}
        n(\varphi\ast f) 
            &= \widetilde{m}(\varphi_0\ast\varphi\ast f) \\
            &= \widetilde{m}(\varphi\ast f) \quad\text{car } \varphi\ast f\in UC_b(\Gamma_d)\\
            &= \widetilde{m}(\varphi_0\ast f) \quad\text{par l'indépendance en $\varphi$ de $\widetilde{m}(\varphi\ast f)$} \\
            &= n(f)
    \end{align*}
    Cela conclut cette première implication. 

    Montrons maintenant \framebox{$(\ref{amenable_TFAE/topological_mean})\implies(\ref{amenable_TFAE/strong_Reiter})$}. Soit donc $m$ une moyenne topologiquement invariante.
    Le théorème \ref{dense_in_means} fournit un filtre $\mathscr{F}$ non-trivial sur $\mathrm{L}^1(\Gamma, \mu)_{1, +}$
    tel que $\widetilde{\iota}_*\mathscr{F}$ converge vers $m$ pour la topologie faible-$*$ sur $\mathcal{M}(\Gamma)$.
    L'invariance topologique de $m$ assure donc que, pour toutes $\varphi\in\mathscr{L}^1(\Gamma)_{1, +}$ et 
    $f\in\mathscr{L}^\infty(\Gamma)$, la fonction $\ket{\blank, \varphi\ast f - f}$ converge vers $0$ selon $\mathscr{F}$.
    Or, pour toute $\psi\in\mathscr{L}^1(\Gamma)_{1, +}$, on a $\ket{\psi, \varphi\ast f - f} = \ket{\varphi^*\ast\psi - \psi, f}$ (lemme \TODO{}),
    donc :
    \begin{equation}\label{amenable_TFAE/eq1}
        \forall\varphi\in\mathscr{L}^1(\Gamma)_{1, +}, \forall f\in\mathscr{L}^\infty(\Gamma), \ket{\varphi\ast\psi - \psi, f}\xrightarrow[\psi\to\mathscr{F}]{}0
    \end{equation}
    Considérons maintenant l'espace vectoriel $E := \mathcal{F}(\mathrm{L}^1(\Gamma)_{1, +}, \mathrm{L}^1(\Gamma))$ muni de la topologie 
    produit des topologies normiques. $E$ est alors localement convexe, et la topologie faible $\sigma(E, E^*)$ sur $E$
    coïncide avec le produit des topologies faibles sur chaque $\mathrm{L}^1(\Gamma)$ (\TODO{explication ?}). Notons $\Sigma\subseteq E$ l'ensemble convexe formé
    des fonctions de la forme $\fundef{A_\psi}{\mathrm{L}^1(\Gamma)_{1, +}&\to&\mathrm{L}^1(\Gamma)}{\varphi&\mapsto&\varphi\ast\psi - \psi}$ pour $\psi\in\mathrm{L}^1(\Gamma)_{1, +}$.
    \eqref{amenable_TFAE/eq1} montre que $0$ est adhérent à $\Sigma$ pour le produit des topologies faibles, 
    donc pour la topologie faible sur $E$, et donc aussi pour la topologie originale de $E$ d'après la proposition \ref{weak_convex_closed}.
    Mais nous avons en plus que chaque $A_\psi$ est $1$-lipschitzienne par l'inégalité de Young, donc l'ensemble
    $\Sigma$ est équicontinu, donc son adhérence simple coïncide avec son adhérence pour la convergence compacte.
    $0$ est donc limite uniforme sur les compacts d'éléments de $\Sigma$, ce qui fournit un nouveau filtre $\mathscr{G}$ non-trivial sur $\mathrm{L}^1(\Gamma, \mu)_{1, +}$ 
    tel que $A:\psi\mapsto A_\psi$ converge vers $0$ selon $\mathscr{G}$ uniformément sur les compacts. Autrement dit :
    \begin{equation}\label{amenable_TFAE/eq2}
        \forall K\in\mathfrak{K}(\mathrm{L}^1(\Gamma, \mu)_{1, +}), \sup_{\varphi\in K}\norm{\varphi\ast\psi - \psi}_{\mathrm{L}^1} \xrightarrow[\psi\to\mathscr{G}]{} 0
    \end{equation}

    Nous avons désormais tous les outils nécessaires pour montrer $(\ref{amenable_TFAE/strong_Reiter})$. Soient $\varepsilon>0$ et $A\in\mathfrak{K}(\Gamma)$.
    Posons alors $Q := A\cup\set{1}$, qui est toujours compact, et choisissons $\varphi_0\in\mathscr{L}^1(\Gamma, \mu)_{1, +}$. La continuité de la représentation régulière $\lambda$ sur 
    $\mathrm{L}^1(\Gamma)$ assure que l'application $\fundef{\ev_{\varphi_0}\comp\lambda}{\Gamma&\to&\mathrm{L}^1(\Gamma, \mu)_{1, +}}{g&\mapsto&\lambda(g)(\varphi)}$ est continue. 
    L'ensemble $K := (\ev_{\varphi_0}\comp\lambda)(Q)$ est donc un compact de $\mathrm{L}^1(\Gamma, \mu)_{1, +}$. \eqref{amenable_TFAE/eq2} donne donc 
    que $\sup_{g\in Q}\norm{\lambda(g)(\varphi_0)\ast\psi - \psi}_{\mathrm{L}^1} \xrightarrow[\psi\to\mathscr{G}]{} 0$, ce qui fournit, par nontrivialité 
    de $\mathscr{G}$, un $\psi\in\mathscr{L}^1(\Gamma, \mu)_{1, +}$ tel que $\sup_{g\in Q}\norm{\lambda(g)(\varphi_0)\ast\psi - \psi}_{\mathrm{L}^1} < \halfilon$. Posons enfin $\varphi := \varphi_0\ast\psi$.
    On a alors $\lambda(g)(\varphi) = \lambda(g)(\varphi_0)\ast\psi$ pour tout $g\in\Gamma$ (\TODO{ref}). Il vient finalement, pour tout $g\in A\subseteq Q$ : 
    \begin{equation*}
        \norm{\lambda(g)(\varphi) - \varphi}_{\mathrm{L}^1} \leq    
            \norm{\lambda(g)(\varphi_0)\ast\psi - \psi}_{\mathrm{L}^1} + \norm{\lambda(1)(\varphi_0)\ast\psi - \psi}_{\mathrm{L}^1}
            < 2\halfilon = \varepsilon
    \end{equation*}
    Cela montre bien que $\Gamma$ satisfait la condition de Reiter forte \eqref{strong_Reiter_cond}.

    Montrons désormais \framebox{$(\ref{amenable_TFAE/strong_Reiter})\implies(\ref{amenable_TFAE/strong_Folner})$}. \TODO{}

    L'implication \framebox{$(\ref{amenable_TFAE/strong_Folner})\implies(\ref{amenable_TFAE/weak_Folner})$} est évidente.
    De plus, on a déjà vu que tout filtre de Reiter faible permet de définir un filtre de F\o{}lner faible. Au vu des lemmes 
    \ref{Folner_filter_of_cond} et \ref{Reiter_filter_of_cond}, cela entraîne l'implication 
    \framebox{$(\ref{amenable_TFAE/weak_Folner})\implies(\ref{amenable_TFAE/weak_Reiter})$}. 

    Il reste donc à montrer \framebox{$(\ref{amenable_TFAE/weak_Reiter})\implies(\ref{amenable_TFAE/amenable})$}. On a déjà vu comment 
    adapter la preuve du théorème \ref{amenable_of_Folner} pour montrer la moyennabilité d'un groupe muni d'un filtre de Reiter 
    faible, mais on va en profiter pour donner une autre construction. Supposons donc que $\Gamma$ satisfait la condition de Reiter faible,
    et soit $\mathscr{F}$ un filtre faiblement de Reiter sur $\mathrm{L}^1(\Gamma, \mu)_{1, +}$, dont l'existence est assurée par le lemme
    \ref{Reiter_filter_of_cond}. Comme $\mathscr{F}$ est non-trivial, il est plus grossier qu'un certain ultrafiltre $\mathscr{U}$. 

    Comme au théorème \ref{dense_in_means}, on munit les espaces $\mathrm{L}^1(\Gamma)$ et $\mathrm{L}^\infty(\Gamma)^*$ des topologies faible et faible-$*$,
    les parties $\mathrm{L}^1(\Gamma, \mu)_{1, +}$, $\mathcal{M}(\Gamma)$ des topologies induites respectives, et on note alors
    $\iota:\mathrm{L}^1(\Gamma)\hookrightarrow\mathrm{L}^\infty(\Gamma)^*$ et $\widetilde{\iota}:\mathrm{L}^1(\Gamma, \mu)_{1, +}\hookrightarrow \mathcal{M}(\Gamma)$
    les plongements topologiques usuels. L'espace topologique $\mathcal{M}(\Gamma)$ étant alors compact,
    l'ultrafiltre $\widetilde{\iota}_*\mathscr{U}$ sur $\mathcal{M}(\Gamma)$ converge vers une certaine moyenne $m$. Il s'agit de montrer que $m$ est invariante,
    soit donc $\gamma\in\Gamma$. Par définition d'un filtre de Reiter faible, la fonction $\ev_\gamma\comp\mathfrak{r}:\varphi\mapsto\norm{\lambda(\gamma\inv)(\varphi)-\varphi}_{\mathrm{L}^1}$
    converge vers $0$ selon $\mathscr{F}$, et donc selon $\mathscr{U}$. Autrement dit la 
    fonction $(\lambda(\gamma\inv)-\id)_{|\mathrm{L}^1(\Gamma, \mu)_{1, +}}$ converge vers $0$ selon $\mathscr{U}$
    en norme $\mathrm{L}^1$, donc pour la topologie faible, et par conséquent $(\iota\comp\lambda(\gamma\inv)-\iota)_{|\mathrm{L}^1(\Gamma, \mu)_{1, +}}$ converge 
    aussi vers $0$ selon $\mathscr{U}$ dans $\mathrm{L}^\infty(\Gamma)^*$. Or, pour tout
    $\varphi\in\mathrm{L}^1(\Gamma)$, on a $\iota(\lambda(\gamma\inv)(\varphi))=\iota(\varphi)\comp\lambda(\gamma)$, de sorte que 
    $\iota\comp\lambda(\gamma\inv) = \transpose{(\lambda(\gamma))}\comp\iota$.
    La transposée de l'isométrie $\lambda(\gamma):\mathrm{L}^\infty(\Gamma)\to\mathrm{L}^\infty(\Gamma)$ est continue 
    pour la topologie faible-$*$, donc l'application $(\iota\comp\lambda(\gamma\inv)-\iota)_{|\mathrm{L}^1(\Gamma, \mu)_{1, +}} = \transpose{(\lambda(\gamma))}\comp\widetilde{\iota} - \widetilde{\iota}$
    converge aussi vers $m\comp\lambda(\gamma)-m$ selon $\mathscr{U}$.
    $\mathscr{U}$ étant non-trivial et $\mathrm{L}^\infty(\Gamma)^*$ séparé, on a $m\comp\lambda(\gamma)-m = 0$ par unicité de la limite, ce qui conclut.
\end{proof}

\section{Contenance faible}

Dans cette section, on désigne par \og{}représentation de $\Gamma$\fg{} une représentation continue d'un
groupe topologique $\Gamma$ dans un espace vectoriel topologique complet sur $\C$, et on utilise le terme de \og{}morphisme 
de représentations\fg{} pour les morphismes associés, c'est à dire les morphismes de représentations abstraites
qui sont de plus linéaires et continus. Pour une représentation $\pi$, on note $V_\pi$ l'espace 
vectoriel topologique sous-jacent, et on note encore $\pi : \Gamma\to\mathcal{GL}(V_\pi)$ le morphisme 
de groupes\footnote{Rappelons qu'il ne s'agit \emph{pas} d'un morphisme de groupes topologiques en général.} associé.
Enfin, une \emph{représentation unitaire} de $\Gamma$ est la donnée d'une représentation $\pi$ et d'une structure 
d'espace hilbertien sur $V_\pi$, induisant bien-sûr la topologie originale sur $V_\pi$. Les morphismes de représentations 
unitaires sont les morphismes de représentations continues entre représentations unitaires.

On veut maintenant caractériser la moyennabilité d'un groupe localement compact (séparé) en termes
de la théorie des représentations unitaires de ce groupe. 

\TODO{Plan de la section}

Rappelons que, pour deux représentations continues $\pi_1$, $\pi_2$ de $\Gamma$, on note 
$\pi_1\le\pi_2$ s'il existe un morphisme de représentations de $\pi_1$ vers $\pi_2$ qui soit un plongement
d'espaces vectoriels topologiques, de sorte que $\pi_1$ s'identifie à une sous-représentation de $\pi_2$.
On note $f : \pi_1\hookrightarrow\pi_2$ pour un tel morphisme\footnote{Cette notation est justifiée par le fait que ces morphismes
sont exactement les monomorphismes de la catégorie des représentations continues de $\Gamma$.}.

Rappelons aussi que, si $\Gamma$ est localement compact et qu'on fixe une mesure de Haar à gauche $\mu$ sur 
$\Gamma$, on peut munir l'espace de Banach $\mathrm{L}^1(\Gamma, \mu)$ d'une structure d'algèbre de Banach \emph{via}
le produit de convolution, défini pour $f_1, f_2\in\mathscr{L}^1(\Gamma, \mu)$ par 
\begin{equation*}
    (f_1\ast f_2)(x) := \integral{}{}{f_1(g)f_2(g\inv x)}{\mu(g)}
\end{equation*}
L'inégalité de Young donne alors $\norm{f_1\ast f_2}_{\mathscr{L}^1} \le \norm{f_1}_{\mathscr{L}^1}\norm{f_2}_{\mathscr{L}^1}$.
Cela implique tout d'abord que, si $f_1$ ou $f_2$ est presque nulle, $f_2\ast f_2$ l'est aussi, et donc que
$\ast$ passe au quotient en $\ast : \mathrm{L}^1 \times\mathrm{L}^1 \to \mathrm{L}^1$, puis que l'opération 
bilinéaire obtenue est bien de norme inférieure à $1$, d'où finalement $\mathrm{L}^1(\Gamma, \mu)$ est bien une algèbre de Banach.
Vérifions enfin qu'il s'agit d'une algèbre de Banach involutive pour l'involution $f\mapsto(\Delta\overline{f})\comp\invop$, qui est bien définie car 
$\invop_*\mu\ll\mu$. Tout d'abord, pour $f\in\mathscr{L}^1(\Gamma, \mu)$, on a : 
\begin{align*}
    \norm{f^*}_{\mathrm{L}^1} 
        &= \integral{}{}{\abs{\Delta(g)\overline{f(g)}}}{(\invop_*\mu)(g)} \\
        &= \integral{}{}{\frac{\Delta(g)}{\Delta(g)}\abs{f(g)}}{(\mu)(g)} \quad\text{(proposition \ref{modular_character} \textit{(\ref{modular_character/inv_change_of_variable})})}\\
        &= \norm{f}_{\mathrm{L}^1}
\end{align*}
La semi-linéarité étant évidente, il reste à montrer, pour toutes $f_1, f_2\in\mathscr{L}^1(\Gamma, \mu)$, l'égalité $f_1^*\ast f_2^* = (f_2\ast f_1)^*$. Soit donc $x\in\Gamma$, et vérifions :
\begin{align*}
    (f_1^*\ast f_2^*)(x) 
        &= \integral{}{}{\Delta(g\inv)\Delta(x\inv g)\overline{f_1(g\inv) f_2(x\inv g)}}{\mu(g)} \\
        &= \Delta(x\inv)\overline{\integral{}{}{f_1(g\inv x\inv) f_2(g)}{\mu(g)}} \quad\text{(proposition \ref{modular_character} \textit{(\ref{modular_character/continuous_group_hom})})}\\
        &= \Delta(x\inv)\overline{(f_2\ast f_2)(x\inv)} \\
        &= (f_2\ast f_1)^*(x)
\end{align*}

On veut maintenant associer à toute représentation \emph{unitaire} $\pi : \Gamma\to\mathcal{U}(V_\pi)$ du \emph{groupe} $\Gamma$ une
représentation, toujours notée $\pi$, de l'\emph{algèbre involutive} $\mathrm{L}^1(\Gamma, \mu)$.

Pour $f\in\mathscr{L}^1(\Gamma, \mu)$ et $x\in V_\pi$, on définit une forme linéaire $\underline{\pi}(f)(x)$ sur $V_\pi^*$ par 
$\underline{\pi}(f)(x)(\varphi) := \integral{}{}{f(g)\cdot\varphi(\pi(g)(x))}{\mu(g)}$, cette intégrale étant finie car $f$ est
intégrable et $\forall g, \abs{\varphi(\pi(g)(x))}\le\norm{\varphi}\norm{x}$. Par l'inégalité de Hölder, on a donc 
\begin{equation*}
    \forall\varphi\in V_\pi^*, \norm{\underline{\pi}(f)(x)(\varphi)}\le\norm{f}_{\mathrm{L}^1}\norm{\varphi}\norm{x}
\end{equation*}
Cela assure que la forme linéaire $\underline{\pi}(f)(x)$ est continue, on a ainsi défini un élément de $V_\pi^{**}$. 
Comme $V_\pi$ est un espace de Hilbert, l'isométrie $\ev:V_\pi\to V_\pi^{**}$ est surjective, et on pose alors $\pi(f)(x) := \ev\inv(\underline{\pi}(f)(x))$. 
Autrement dit, en exploitant l'isomorphisme semi-linéaire $V_\pi\simeq V_\pi^*$ fourni par le théorème de représentation de Riesz, 
$\pi(f)(x)$ est l'unique élément de $V_\pi$ tel que :
\begin{equation*}
    \forall y\in V_\pi, \ket{\pi(f)(x), y} = \integral{}{}{f(g)\ket{\pi(g)(x), y}}{\mu(g)}
\end{equation*}
De plus, on a vu que $\norm{\pi(f)(x)} = \norm{\underline{\pi}(f)(x)} \le \norm{f}_{\mathrm{L}^1}\norm{x}$. 
On a donc $\pi(f)\in\mathcal{L}(V_\pi)$ et $\pi:\mathrm{L}^1(\Gamma, \mu)\to\mathcal{L}(V_\pi)$ est continue de norme inférieure à $1$.
Pour montrer qu'il s'agit d'un morphisme de représentations d'algèbres de Banach involutives, il s'agit donc de montrer...

\TODO{Continuer}

\begin{remark}\label{L1_repr_Cc}
    Pour $f\in C_c(\Gamma)$ et $x\in V_\pi$, la fonction $g\mapsto f(g)\pi(g)(x)\in V_\pi$ est continue à support compact,
    donc fortement intégrable en vertu du théorème \ref{strong_measurable_crit} et de la remarque qui l'accompagne. 
    Par continuité du produit scalaire, on a alors $\forall y\in V_\pi$ : 
    \begin{equation*}
        \ket{\integral{}{}{f(g)\pi(g)(x)}{\mu(g)}, y} = \integral{}{}{f(g)\ket{\pi(g)(x), y}}{\mu(g)} = \ket{\pi(f)(x), y}
    \end{equation*}
    Autrement dit, $\pi(f)(x) = \integral{}{}{f(g)\pi(g)(x)}{\mu(g)}$.
\end{remark}

\begin{theorem}\label{amenable_weak_contain}
    On note $\indic_\Gamma:\Gamma\to\mathbb{S}_1$ et $\lambda_\Gamma:\Gamma\to\mathcal{U}(\mathrm{L}^1(\Gamma))$
    les représentations triviales et régulières gauches de $\Gamma$. Les assertions suivantes sont équivalentes :
    \begin{enumerate}[(i)]
        \item\label{amenable_weak_contain/amenable} $\Gamma$ est moyennable
        \item\label{amenable_weak_contain/weak_almost_invariant} $\forall\varepsilon>0, \forall F\in\finparts(\Gamma), \exists\xi\in \mathrm{L}^2(\Gamma), \forall \gamma\in F, \norm{\xi}_{\mathrm{L}^2} = 1 \land \norm{\lambda(\gamma)(\xi) - \xi}_{\mathrm{L}^2}<\varepsilon$
        \item\label{amenable_weak_contain/strong_almost_invariant} $\forall\varepsilon>0, \forall K\in\mathfrak{K}(\Gamma), \exists\xi\in \mathrm{L}^2(\Gamma), \norm{\xi}_{\mathrm{L}^2} = 1 \land \forall \gamma\in K, \norm{\lambda(\gamma)(\xi) - \xi}_{\mathrm{L}^2}<\varepsilon$
        \item\label{amenable_weak_contain/weak_contain} $\indic_\Gamma\wle\lambda_\Gamma$
        \item\label{amenable_weak_contain/norm_eq_two} $\forall F\in\finparts(\Gamma), \norm{\sum_{\gamma\in F}\lambda_\Gamma(\gamma)+\lambda_\Gamma(\gamma\inv)} = 2\card{F}$
        %\item\label{amenable_weak_contain/one_mem_spectrum} $\forall\gamma\in\Gamma, 1\in\Sp(\lambda_\Gamma(\gamma))$
    \end{enumerate}
\end{theorem}

\TODO{commentaires}

\subsection{Quelques résultats de théorie spectrale}

Nous aurons besoin de la notion standard de \emph{spectre approché} d'un opérateur.

\begin{definition}
    Soit $E$ un espace de Banach complexe et $T\in\mathcal{L}(E)$. 
    On appelle \emph{spectre approché de $T$} l'ensemble $\sigma(T) := \set{\lambda\in\C\tq\forall\varepsilon>0, \exists\xi\in E, \norm{\xi}=1\land\norm{T(\xi)-\lambda\xi}<\varepsilon}$.
\end{definition}

Notons en particulier que l'assertion $\lambda\notin\sigma(T)$ s'écrit encore :
\begin{equation*}
    \exists\varepsilon>0, \forall\xi\in E, \norm{\xi}=1\implies \norm{T(\xi)-\lambda\xi}\geq\varepsilon
\end{equation*}
Or, cet énoncé traduit des propriétés analytiques fondamentales de l'opérateur $T - \lambda$, comme l'exprime le lemme suivant :

\begin{lemma}\label{EVT_emb_TFAE}
    Soient $E$ un espace de Banach, $F$ un espace vectoriel normé, et $T\in\mathcal{L}(E, F)$. Les assertions suivantes sont équivalentes :
    \begin{enumerate}[(i)]
        \item $T$ est un isomorphisme sur son image, i.e un plongement d'espaces vectoriels topologiques \label{EVT_emb_TFAE/emb}
        \item $T$ est injectif et d'image complète \label{EVT_emb_TFAE/closed_range_inj}
        \item $\exists\varepsilon>0, \forall x\in E, \norm{T(x)}\geq\varepsilon\norm{x}$ \label{EVT_emb_TFAE/lower_bound_norm}
        \item $\exists\varepsilon>0, \forall x\in E, \norm{x} = 1 \implies \norm{T(x)}\geq\varepsilon$ \label{EVT_emb_TFAE/lower_bound_norm'}
    \end{enumerate}
    De plus, ces conditions sont toujours vérifiées si $E=F$ est un espace de Hilbert et $T$ vérifie $\exists\varepsilon>0, \forall x\in E, \norm{x} = 1 \implies \abs{\ket{T(x), x}}\ge\varepsilon$.
\end{lemma}

\begin{proof} 
    L'implication $(\ref{EVT_emb_TFAE/emb}) \implies(\ref{EVT_emb_TFAE/closed_range_inj})$ découle de ce que les isomorphismes linéaires 
    continus entre espaces vectoriels normés sont bi-lipschitziens, et préservent donc la complétude\footnote{On peut aussi donner un argument plus général, 
    valable pour un isomorphisme linéaire $L$ entre des espaces vectoriels topologiques métrisables $A$ et $B$. En effet, cette hypothèse assure
    que les structures uniformes sur $A$ et $B$ provenant de leur structure métrique coïncident avec les 
    structures uniformes (gauche ou droite) associées aux groupes topologiques sous-jacents, et l'isomorphisme de groupes 
    topologiques $L$ est automatiquement un isomorphisme entre ces structures uniformes, qui préserve donc la complétude.}.
    L'implication réciproque $(\ref{EVT_emb_TFAE/closed_range_inj}) \implies(\ref{EVT_emb_TFAE/emb})$ est alors une conséquence immédiate du
    théorème d'isomorphisme de Banach, qui assure que la co-restriction de $T$ à son image est un isomorphisme 
    car bijective, linéaire et continue entre espaces de Banach.

    Il est clair que $(\ref{EVT_emb_TFAE/lower_bound_norm}) \iff(\ref{EVT_emb_TFAE/lower_bound_norm'})$, montrons donc $(\ref{EVT_emb_TFAE/emb}) \iff(\ref{EVT_emb_TFAE/lower_bound_norm})$. Notons que $(\ref{EVT_emb_TFAE/lower_bound_norm})$ implique
    immédiatement que $T$ est injectif, et donc une bijection sur son image. En notant $S:\Ima T\to E$ l'inverse (a priori seulement linéaire) de $T$ co-restreint à son image, 
    il s'agit donc de montrer que la continuité de $S$ équivaut à $(\ref{EVT_emb_TFAE/lower_bound_norm})$. Or :
    \begin{align*}
        S\text{ est continu }
            &\iff \exists C>0, \forall y\in\Ima T, \norm{S(y)} \le C\norm{y} \\
            &\iff \exists C>0, \forall x\in E, C\inv\norm{x} \le \norm{T(x)} \\
            &\iff \exists\varepsilon>0, \forall x\in E, \varepsilon\norm{x} \le \norm{T(x)}
    \end{align*}
        %\item[\framebox{$(\ref{EVT_emb_TFAE/emb}) \implies(\ref{EVT_emb_TFAE/lower_bound_norm})$}]
        %Notons $S\in\mathcal{L}(\Ima T, E)$ l'inverse de $T$ co-restreint à son image, et vérifions que $\varepsilon := \norm{S}\inv$ convient.
        %Soit donc $x\in E$. On a $S(T(x)) = x$ par définition, d'où $\norm{x} \leq \norm{S}\norm{T(x)}$, ce qui donne bien $\varepsilon\norm{x}\leq\norm{T(x)}$.
        %\item[\framebox{$(\ref{EVT_emb_TFAE/lower_bound_norm}) \implies(\ref{EVT_emb_TFAE/closed_range_inj})$}]
        %Soit $\varepsilon>0$ tel que fournit par l'hypothèse $(\ref{EVT_emb_TFAE/lower_bound_norm})$.
        %L'injectivité est immédaite, car si $x\in E$ est tel que $\norm{T(x)} = 0$ on a nécessairement $\norm{x} = 0$. Soit maintenant 
        %une suite de Cauchy $u:\N\to\Ima T$, que nous écrivons sous la forme $T\comp v$ pour une suite 
        %$v:\N\to E$. Notons que $v$ est encore de Cauchy. En effet, pour $\delta>0$ quelconque, l'hypothèse sur 
        %$u$ fournit $N\in\N$ tel que $\forall p, q \geq N, \norm{u_p - u_q} < \delta\varepsilon$.
        %On a donc 

    Enfin, supposons que $E=F$ est un espace de Hilbert, et que, pour un certain $\varepsilon>0$, on a $\forall x\in E, \norm{x} = 1 \implies \abs{\ket{T(x), x}}\ge\varepsilon$.
    Il suffit alors d'appliquer l'inégalité de Cauchy-Schwartz : pour tout $\xi$ unitaire, on a $\varepsilon\le\abs{\ket{T(x), x}}\le\norm{T(x)}$. La condition $(\ref{EVT_emb_TFAE/lower_bound_norm'})$
    est donc vérifiée.
\end{proof}

Le lemme \ref{EVT_emb_TFAE} permet de caractériser le spectre approché d'un opérateur $T$,
puisque son complémentaire $\sigma(T)\compl$ est précisément l'ensemble des $\lambda$ tels que $T-\lambda$ vérifie les 
conditions équivalentes de ce lemme. En particulier, cela nous permet de démontrer le résultat fondamental suivant :

\begin{lemma}\label{approx_spectrum_facts}
    Soient $H$ un espace de Hilbert complexe et $T\in\mathcal{L}(H)$. 
    \begin{enumerate}[(i)]
        \item Pour $T$ quelconque, on a $\sigma(T)\subseteq\Sp(T)$ \label{approx_spectrum_facts/subset}
        \item Si $T$ est unitaire, $\sigma(T)\cap\R = \Sp(T)\cap\R$ \label{approx_spectrum_facts/eq_of_unitary}
        \item Si $T$ est autoadjoint, $\sigma(T) = \Sp(T)$ \label{approx_spectrum_facts/eq_of_hermitian}
    \end{enumerate}
\end{lemma}

\begin{proof}
    Si $\lambda\in\sigma(T)$, le lemme \ref{EVT_emb_TFAE} assure que $T-\lambda$ n'est \emph{pas} un isomorphisme sur son image, et donc que 
    $T-\lambda$ n'est pas inversible. Ceci montre le point $(\ref{approx_spectrum_facts/subset})$.

    Pour montrer $(\ref{approx_spectrum_facts/eq_of_unitary})$ et $(\ref{approx_spectrum_facts/eq_of_hermitian})$, remarquons que dans les deux cas
    on a $\sigma(T)\cap\R = \sigma(T^*)\cap\R$. Si $T$ est autoadjoint c'est immédiat, vérifions le pour $T$ unitaire. Soient $\lambda\in\R$ que l'on
    suppose appartenir à $\sigma(T)$, et $\varepsilon>0$. Par hypothèse on a $\norm{T(\xi) - \lambda\xi}<\varepsilon$ pour un certain $\xi$ de norme $1$,
    ce qui entraîne $\norm{\xi - T^*(\lambda\xi)}<\varepsilon$, et donc $\norm{T^*(\xi) - \lambda\inv\xi}<\varepsilon\abs{\lambda}\inv$. Or $\lambda\in\sigma(T)\subseteq\Sp(T)\subseteq\mathbb{S}_1$
    car $T$ est unitaire, et $\lambda\in\R$ donc $\lambda\in\set{-1, 1}$. On a donc en fait $\norm{T^*(\xi) - \lambda\xi}<\varepsilon$, ce qui 
    assure que $\lambda\in\sigma(T^*)$. Cela montre $\sigma(T)\cap\R\subseteq\sigma(T^*)\cap\R$, et
    l'inclusion réciproque vient en appliquant ce résultat à l'opérateur unitaire $T^*$.

    Montrons alors, toujours pour $T$ unitaire ou autoadjoint, que $\sigma(T)\cap\R \supseteq \Sp(T)\cap\R$. Soit pour cela un réel $\lambda$
    n'appartenant \emph{pas} à $\sigma(T)$, de sorte que $T-\lambda$ soit injectif et d'image fermée (lemme \ref{EVT_emb_TFAE}). 
    Comme $\sigma(T)\cap\R = \sigma(T^*)\cap\R$, on a aussi $\lambda\notin\sigma(T^*)$, d'où le même résultat pour $T^* - \lambda = (T-\lambda)^*$. Or, 
    l'injectivité de $(T-\lambda)^*$ assure que l'image de $T-\lambda$ est dense. Cette image étant fermée, on a $T-\lambda$ surjectif et injectif,
    donc inversible par le théorème d'isomorphisme de Banach, d'où $\lambda\notin\Sp(T)$. 

    On a donc montré $\sigma(T)\cap\R = \Sp(T)\cap\R$. Cela conclut la preuve de $(\ref{approx_spectrum_facts/eq_of_unitary})$,
    et pour conclure celle de $(\ref{approx_spectrum_facts/eq_of_hermitian})$ il suffit de se rappeler que pour $T$ autoadjoint on a $\sigma(T) \subseteq \Sp(T) \subseteq\R$.
\end{proof}

%Nous avons désormais tous les outils nécessaires à la preuve du théorème \ref{amenable_weak_contain}.

\subsection{Preuve du théorème \ref{amenable_weak_contain}}

Commençons par un petit lemme qui sera crucial.

\begin{lemma}\label{abs_trick}
    Soit $\xi\in\mathscr{L}^2(\Gamma, \mu)$. La fonction $\abs{\xi} := \abs{\cdot}\comp\xi \in\mathscr{L}^2(\Gamma, \mu)$ est de même norme 
    que $\xi$, et on a $\forall\gamma\in\Gamma, \norm{\lambda(\gamma)(\abs{\xi}) - \abs{\xi}}_{\mathrm{L}^2} \le \norm{\lambda(\gamma)(\xi) - \xi}_{\mathrm{L}^2}$
\end{lemma}

\begin{proof}
    Il est clair que $\norm{\abs{\xi}}_{\mathrm{L}^2} = \norm{\xi}_{\mathrm{L}^2}$. Pour $x\in\Gamma$ quelconque, 
    la petite inégalité triangulaire donne $\abs{\abs{\xi}(\gamma\inv x) - \abs{\xi}(x)} = \abs{\abs{\xi(\gamma\inv x)} - \abs{- \xi(x)}} \leq \abs{\xi(\gamma\inv x) + \xi(x)}$. En intégrant le carré de ces inégalités,
    on obtient $\norm{\lambda(\gamma)(\abs{\xi}) - \abs{\xi}}_{\mathrm{L}^2} \leq \norm{\lambda(\gamma)(\xi) + \xi}_{\mathrm{L}^2}$.
\end{proof}

\begin{proof}[Démonstration du théorème \ref{amenable_weak_contain}]
    On va montrer $(\ref{amenable_weak_contain/weak_almost_invariant})
        \implies(\ref{amenable_weak_contain/amenable})
        \implies(\ref{amenable_weak_contain/strong_almost_invariant})
        \implies(\ref{amenable_weak_contain/weak_contain})
        \implies(\ref{amenable_weak_contain/norm_eq_two})
        \implies(\ref{amenable_weak_contain/weak_almost_invariant})$.
    
    Commençons donc par montrer \framebox{$(\ref{amenable_weak_contain/weak_almost_invariant})\implies(\ref{amenable_weak_contain/amenable})$} en s'appuyant sur le critère 
    de Reiter faible \eqref{weak_Reiter_cond}. 
    Soient donc $\varepsilon>0$ et $F\subseteq\Gamma$ fini, ainsi qu'un $\xi\in \mathscr{L}^2(\Gamma)$ unitaire vérifiant $\forall\gamma\in F, \norm{\pi(\gamma\inv)(\xi) - \xi}_{\mathrm{L}^2}<\varepsilon$ tel que fourni par $(\ref{amenable_weak_contain/weak_almost_invariant})$.
    Posons alors $f := \abs{\cdot}^2\comp\xi\in\mathscr{L}^1(\Gamma, \mu)_{1, +}$, qui vérifie \TODO{finir}

    On montre similairement \TODO{\framebox{$(\ref{amenable_weak_contain/amenable})\implies(\ref{amenable_weak_contain/strong_almost_invariant})$}} en s'appuyant sur le critère de Reiter fort \eqref{strong_Reiter_cond}.

    Montrons maintenant \framebox{$(\ref{amenable_weak_contain/strong_almost_invariant})\implies(\ref{amenable_weak_contain/weak_contain})$}. Soit donc $f\in\mathscr{L}^1(\Gamma, \mu)$,
    et montrons $\norm{\indic_\Gamma(f)}\le\norm{\lambda_\Gamma(f)}$. C'est trivial pour $f=_\mu^{loc} 0$, donc on suppose $\norm{f}_{\mathrm{L}^1}\ne0$. Par continuité des représentations $\indic_\Gamma$ et 
    $\lambda_\Gamma$ de $\mathrm{L}^1(\Gamma, \mu)$\footnote{Contrairement au cas des représentations de groupes topologiques, 
    les représentation continues d'algèbres de Banach involutives sont exactement les représentations algébriques telles que le 
    morphisme sous-jacent est continu}, il suffit en fait de montrer le résultat pour $f\in C_c(\Gamma)$, ce que nous supposons désormais.
    Nous allons montrer $\abs{\integral{}{}{f}{\mu}}^2\in\Sp\left(\lambda_\Gamma(f)^*\comp \lambda_\Gamma(f)\right)$, ce qui entraîne bien :
    \begin{equation*}
        \norm{\indic_\Gamma(f)}^2=\abs{\integral{}{}{f}{\mu}}^2\le\norm{\lambda_\Gamma(f)^*\comp \lambda_\Gamma(f)} = \norm{\lambda_\Gamma(f)}^2
    \end{equation*}
    Par l'absurde, on suppose donc $T := \abs{\integral{}{}{f}{\mu}}^2 - \lambda_\Gamma(f)^*\comp \lambda_\Gamma(f)$ inversible, ce qui fournit $\varepsilon>0$
    tel que $\forall\xi\in\mathrm{L}^2(\Gamma, \mu), \norm{T(\xi)}\ge\varepsilon\norm{\xi}$.
    Pour $\xi\in\mathrm{L}^2(\Gamma, \mu)$, la remarque \ref{L1_repr_Cc} donne :
    \begin{align*}
        \lambda_\Gamma(f)^*\left(\lambda_\Gamma(f)(\xi)\right) 
            &= \lambda_\Gamma(f^*)\left(\integral{}{}{f(g)\lambda_\Gamma(g)(\xi)}{\mu(g)}\right) \\
            &= \integral{}{}{f(g)\lambda_\Gamma(f^*)\Big(\lambda_\Gamma(g)(\xi)\Big)}{\mu(g)} \\
            &= \integral{}{}{f(g)\integral{}{}{f^*(h)\lambda_\Gamma(hg)(\xi)}{\mu(h)}}{\mu(g)} \\
            &= \integral{}{}{\integral{}{}{f(g)\overline{f(h)}\lambda_\Gamma(h\inv g)(\xi)}{\mu(h)}}{\mu(g)}\quad\text{\TODO{explications}}
    \end{align*} 
    D'autre part, on a :
    \begin{align*}
        \abs{\integral{}{}{f}{\mu}}^2 
            &= \left(\integral{}{}{f}{\mu}\right)\left(\integral{}{}{\overline{f}}{\mu}\right) \\
            &= \integral{}{}{\integral{}{}{f(g)\overline{f(h)}}{\mu(g)}}{\mu(h)}
    \end{align*}
    On a donc pour tout $\xi\in\mathrm{L}^2(\Gamma, \mu)$ unitaire :
    \begin{align*}
        \varepsilon 
            &\le \norm{\integral{}{}{\integral{}{}{f(g)\overline{f(h)}(\xi - \lambda_\Gamma(h\inv g)(\xi))}{\mu(h)}}{\mu(g)}}_{\mathrm{L}^2} \\
            &\le \integral{}{}{\integral{}{}{\abs{f(g)}\abs{f(h)}\norm{\xi - \lambda_\Gamma(h\inv g)(\xi)}_{\mathrm{L}^2}}{\mu(h)}}{\mu(g)}
    \end{align*}
    Mais, si l'on note $K$ le support de $f$ et que l'on applique $(\ref{amenable_weak_contain/strong_almost_invariant})$ au réel $\frac{\varepsilon}{\norm{f}_{\mathrm{L}^1}^2}$ et au compact $K\inv K$,
    on obtient $\xi\in\mathrm{L}^2(\Gamma, \mu)$ unitaire tel que $\sup_{\gamma\in K\inv K} \norm{\lambda_\Gamma(\gamma)(\xi) - \xi}_{\mathrm{L}^2}<\frac{\varepsilon}{\norm{f}_{\mathrm{L}^1}^2}$, d'où :
    \begin{equation*}
        \varepsilon
            \le \integral{}{}{\integral{}{}{\abs{f(g)}\abs{f(h)}\norm{\xi - \lambda_\Gamma(h\inv g)(\xi)}_{\mathrm{L}^2}}{\mu(h)}}{\mu(g)} 
            < \varepsilon
    \end{equation*}
    D'où la contradiction recherchée.

    Pour montrer l'implication \framebox{$(\ref{amenable_weak_contain/weak_contain})\implies(\ref{amenable_weak_contain/norm_eq_two})$}, donnons nous 
    $F\in\finparts(\Gamma)$ quelconque, et commençons par remarquer 
    que l'inégalité $\norm{\sum_{\gamma\in F}\lambda_\Gamma(\gamma)+\lambda_\Gamma(\gamma\inv)} \le 2\card{F}$ est automatiquement vérifée, il s'agit donc de montrer l'inégalité 
    inverse. Commençons par donner une preuve pour $\Gamma$ discret, qui nous éclairera pour la suite, et pour simplifier supposons
    que $\mu$ donne masse $1$ aux singletons. Il suffit alors d'appliquer $(\ref{amenable_weak_contain/weak_contain})$
    à la fonction $f := \sum_{\gamma\in F}\delta_\gamma + \delta_{\gamma\inv}\in\mathscr{L}^1(\Gamma)_{+}$ pour obtenir :
    \begin{equation*}
        2\card{F} = \abs{\integral{}{}{f}{\mu}} = \abs{\indic_\Gamma(f)} \le \norm{\lambda_\Gamma(f)} = \norm{\sum_{\gamma\in F}\lambda_\Gamma(\gamma)+\lambda_\Gamma(\gamma\inv)}\le2\card{F}
    \end{equation*}

    Revenons maintenant au cas général. On pourrait être tenté de se ramener au raisonnement précédent 
    en approchant la \emph{mesure} $\sum_{\gamma\in F}\delta_\gamma + \delta_{\gamma\inv}$ par des fonctions intégrables,
    une telle approximation s'obtenant aisément à partir d'une approximation de l'unité. Cependant,
    on se heurte alors au fait que la fonction $\lambda_\Gamma : \Gamma \to \mathcal{U}(\mathrm{L}^2(\Gamma, \mu))$
    n'est \emph{a priori} pas continue, ce qui rend inutilisable l'hypothèse de concentration de la masse : 
    même si $f\in\mathscr{L}^1(\Gamma, \mu)$ a sa masse concentrée autour des points de $F\cup F\inv$,
    rien ne dit que $\lambda_\Gamma(f)$ sera proche de $\sum_{\gamma\in F}\lambda_\Gamma(\gamma) + \lambda_\Gamma(\gamma\inv)$. 
    \footnote{Il est d'ailleurs intéressant que de remarquer que le morphisme $\lambda_\Gamma$ est bien continu
    si $\Gamma$ est discret, ce qui explique pourquoi la preuve fonctionne dans ce cas.}

    La solution que nous proposons est d'étudier ce qui se passe lorsqu'on itère notre opérateur $\sum_{\gamma\in F}\lambda_\Gamma(\gamma) + \lambda_\Gamma(\gamma\inv)$.
    Fixons pour cela une fonction $f_0\in\mathscr{L}^1(\Gamma, \mu)_{1, +}$\footnote{On a déjà vu qu'une telle fonction existe toujours, par exemple en posant $f_0:=\chi_U=\frac{1}{\mu(U)}\indic_U$ pour 
    un voisinage compact $U$ de $1$.} qui représente une distribution de masse initiale, et considérons les opérateurs $T := \sum_{\gamma\in F}\lambda_\Gamma(\gamma) + \lambda_\Gamma(\gamma\inv) \in\mathcal{L}(\mathrm{L}^2(\Gamma))$
    et $S := \sum_{\gamma\in F}\lambda(\gamma) + \lambda(\gamma\inv) \in \mathcal{L}(\mathrm{L}^1(\Gamma))$. Par \TODO{ref ou argument à base de convolution}, on a :
    \begin{equation*}
        \forall f\in\mathrm{L}^1(\Gamma), \lambda_\Gamma(S(f)) = T \comp \lambda_\Gamma(f)
    \end{equation*}
    Notons que, pour toute $f\in\mathscr{L}^1(\Gamma)$ positive, la fonction $S(f)$ est encore positive, et vérifie $\integral{}{}{S(f)}{\mu} = 2\card{F}\integral{}{}{f}{\mu}$.
    Par récurrence, on a donc $\forall n\in\N, S^n(f_0)\geq0 \land \integral{}{}{S^n(f_0)}{\mu} = (2\card{F})^n$. Or, $(\ref{amenable_weak_contain/weak_contain})$
    assure que l'on a, toujours pour $n\in\N$ quelconque :
    \begin{equation*}
        (2\card{F})^n = \integral{}{}{S^n(f_0)}{\mu} = \norm{\indic_\Gamma(S^n(f_0))} \leq \norm{\lambda_\Gamma(S^n(f_0))} = \norm{T^n \comp \lambda_\Gamma(f_0)} \le \norm{T}^n \norm{\lambda_\Gamma(f_0)}
    \end{equation*}
    La suite géométrique positive $n\mapsto\left(\frac{2\card{F}}{\norm{T}}\right)^n$ est donc bornée supérieurement par $\norm{\lambda_\Gamma(f_0)}$, ce qui assure que $\frac{2\card{F}}{\norm{T}} \leq 1$,
    et donc $2\card{F}\leq\norm{T}$ comme attendu.

    Montrons enfin \framebox{$(\ref{amenable_weak_contain/norm_eq_two})\implies(\ref{amenable_weak_contain/weak_almost_invariant})$}. Soient donc $\varepsilon>0, F\in\finparts(\Gamma)$, et 
    raisonnons par l'absurde en supposant :
    \begin{equation}\label{amenable_weak_contain/eq1}
        \forall\xi\in\mathrm{L}^2(\Gamma), \norm{\xi}_{\mathrm{L}^2} = 1\implies \exists\gamma\in F, \norm{\lambda_\Gamma(\gamma)(\xi) - \xi}_{\mathrm{L}^2}\geq\varepsilon
    \end{equation}
    Notons que cette hypothèse entraîne également :
    \begin{equation}\label{amenable_weak_contain/eq2}
        \forall\xi\in\mathrm{L}^2(\Gamma), \norm{\xi}_{\mathrm{L}^2} = 1\implies \exists\gamma\in F, \norm{\lambda_\Gamma(\gamma)(\xi) + \xi}_{\mathrm{L}^2}\geq\varepsilon
    \end{equation}
    En effet, pour $\xi\in\mathscr{L}^2(\Gamma)$ unitaire, on peut appliquer \eqref{amenable_weak_contain/eq1} à $\abs\xi$ pour obtenir un $\gamma\in F$ tel que $\norm{\lambda_\Gamma(\gamma)(\abs\xi) - \abs\xi}_{\mathrm{L}^2}\geq\varepsilon$.
    Mais on a $\norm{\lambda(\gamma)(\abs{\xi}) - \abs{\xi}}_{\mathrm{L}^2} \le \norm{\lambda(\gamma)(\xi) - \xi}_{\mathrm{L}^2}$ par le lemme \ref{abs_trick}, ce qui achève de montrer \eqref{amenable_weak_contain/eq2}.

    Pour $\rho\in\set{\pm1}$, considérons $U_\rho := \sum_{\gamma\in F}(\lambda_\Gamma(\gamma) - \rho\id)^*(\lambda_\Gamma(\gamma) - \rho\id) = 2\rho\card{F} - T$, 
    où l'on note toujours $T = \sum_{\gamma\in F}\lambda_\Gamma(\gamma) + \lambda_\Gamma(\gamma\inv)$. Notons que chaque $U_\rho$ est inversible. 
    En effet, \eqref{amenable_weak_contain/eq1} et \eqref{amenable_weak_contain/eq2} montrent que, quelle que soit la valeur de $\rho$, 
    on a pour tout $\xi\in\mathrm{L}^2(\Gamma)$ unitaire :
    \begin{equation*}
        \varepsilon^2\leq\sum_{\gamma\in F}\norm{\lambda_\Gamma(\gamma)(\xi) - \rho\xi}_{\mathrm{L}^2}^2 = \ket{U_\rho(\xi), \xi}
    \end{equation*}
    Le lemme \ref{EVT_emb_TFAE} assure alors que $U_\rho$ est injectif et d'image complète. Mais $U_\rho$ est autoadjoint, donc 
    son injectivité implique qu'il est d'image dense. Il vient que $U_\rho$ est bijectif, donc inversible par le théorème d'isomorphisme de Banach,
    et ce toujours pour $\rho\in\set{\pm1}$ quelconque.

    Or $U_\rho = 2\rho\card{F} - T$, donc par définition du spectre on a $\Sp(T)\cap\set{\pm2\card{F}}=\varnothing$. Mais $T$ est autoadjoint,
    et de norme $2\card{F}$ d'après $(\ref{amenable_weak_contain/norm_eq_two})$. Donc $\rho(T) = 2\card{F}$, d'où $\Sp(T)\cap\set{\pm2\card{F}}\neq\varnothing$
    puisque $\Sp(T)\subseteq\R$. Cela nous donne la contradiction recherchée.

    %On va maintenant montrer \framebox{$(\ref{amenable_weak_contain/norm_eq_two})\implies(\ref{amenable_weak_contain/one_mem_spectrum})$}, soit donc $\gamma\in\Gamma$ et notons 
    %$U := \lambda_\Gamma(\gamma)$. Conservons aussi la notation $T := U + U^*$. L'opérateur $T$ étant autoadjoint, son spectre est réel 
    %et sa norme est égale à son rayon spectral, de sorte que $\set{-2, 2}\cap\Sp(T)\ne\varnothing$. On va maintenant 
    %mettre en relation le spectre de $U$ et le spectre de $T$, à l'aide du calcul fonctionnel continu de l'opérateur unitaire (donc normal) $U$.
    %En effet, si l'on note $X \in C(\Sp U) = C(\Sp U, \C)$ l'application d'inclusion, on sait que $X(U) = U$ et $\overline{X}(U) = U^*$, d'où
    %$(2\Re(X))(U) = (X + \overline{X})(U) = U + U^* = T$ par linéarité du calcul fonctionnel. Le spectre de $T$ est donc l'image 
    %du spectre de $U$ par l'application $z\mapsto2\Re(z)$, ce qui fournit $z\in\Sp(U)$ tel que $2\Re(z)\in\set{-2, 2}$ ou encore $\Re(z)\in\set{-1, 1}$. 
    %Mais $z\in\Sp(U)\subseteq\overline{B}(0, 1)$, donc cela n'est possible que si $z\in\set{-1, 1}$.
%
    %Si $z=1$ on a fini. Supposons donc $-1\in\Sp(U)$, et montrons qu'on a aussi $1\in\Sp(U)$ dans ce cas.
    %Comme $U$ est unitaire, on peut en fait travailler avec le spectre approché en vertu du lemme \ref{approx_spectrum_facts}. Soit 
    %donc $\varepsilon>0$, et appliquons l'hypothèse $-1\in\sigma(U)$ pour obtenir un $\xi\in\mathscr{L}^2(\Gamma)$ tel que $\norm{\xi}_{\mathrm{L}^2} = 1$ et $\norm{\lambda(\gamma)(\xi) + \xi}_{\mathrm{L^2}} < \varepsilon$.
    %La fonction $\abs{\xi} = \abs{\cdot}\comp\xi$ est encore de carré intégrable avec $\norm{\abs{\xi}}_{\mathrm{L}^2} = \norm{\xi}_{\mathrm{L}^2} = 1$. Or, pour $x\in\Gamma$ quelconque, 
    %la petite inégalité triangulaire donne $\abs{\abs{\xi}(\gamma\inv x) - \abs{\xi}(x)} = \abs{\abs{\xi(\gamma\inv x)} - \abs{- \xi(x)}} \leq \abs{\xi(\gamma\inv x) + \xi(x)}$. En intégrant le carré de ces inégalités,
    %on obtient $\norm{\lambda_\Gamma(\gamma)(\abs{\xi}) - \abs{\xi}}_{\mathrm{L}^2} \leq \norm{\lambda_\Gamma(\gamma)(\xi) + \xi}_{\mathrm{L}^2} < \varepsilon$,
    %ce qui assure que $1\in\sigma(\lambda_\Gamma(\gamma))$.
%
    %Enfin, l'implication \framebox{$(\ref{amenable_weak_contain/one_mem_spectrum})\implies(\ref{amenable_weak_contain/weak_almost_invariant})$} découle directement du lemme \ref{approx_spectrum_facts}.
    %En effet les opérateurs $\lambda_\Gamma(\gamma)$ sont tous unitaires, donc $(\ref{amenable_weak_contain/one_mem_spectrum})$ assure que 
    %$1$ est dans le spectre approché de chacun des $\lambda_\Gamma(\gamma)$, ce qui signifie précisément que $(\ref{amenable_weak_contain/weak_almost_invariant})$ est vérifié.



    %Montrons finalement \framebox{$(\ref{amenable_weak_contain/weak_contain})\implies(\ref{amenable_weak_contain/weak_almost_invariant})$}. Soient donc $\varepsilon>0$ et $\gamma\in\Gamma$,
    %et raisonnons par l'absurde en supposant $\forall \xi\in\mathrm{L}^2(\Gamma, \mu), \norm{\xi}_{\mathrm{L}^2} = 1\implies \norm{\lambda(\gamma)(\xi) - \xi}_{\mathrm{L}^2}\geq\varepsilon$.
    %Autrement dit, si l'on note $S:=\lambda(\gamma) - \Id$, on a $\forall\xi\in\mathrm{L}^2(\Gamma, \mu), \norm{S(\xi)}_{\mathrm{L}^2}\ge\varepsilon\norm{\xi}_{\mathrm{L}^2}$.
    %Posons alors $T := S^*\comp S$, et montrons que $T$ est inversible. Tout d'abord, l'inégalité de Cauchy-Schwartz donne :
    %\begin{equation*}
    %    \norm{T(\xi)}\norm{\xi}\geq\ket{S^*(S(\xi)), \xi}=\norm{S(\xi)}^2\geq\varepsilon^2\norm{\xi}^2
    %\end{equation*}
    %D'où $\forall\xi, \norm{T(\xi)}\geq\varepsilon^2\norm{\xi}$. Cela assure que l'image de $T$ est complète (\TODO{ref}) donc fermée,
    %et également que $T$ est injective. Mais $T$ est autoadjoint, donc $\Ima T = \closure{\Ima T} = (\ker T)^\perp = \mathrm{L}^2(\Gamma, \mu)$, i.e $T$ est bijectif,
    %donc inversible par l'inégalité $\forall\xi, \norm{T(\xi)}\geq\varepsilon^2\norm{\xi}$\footnote{Ou bien par le théorème de l'application ouverte.}.
    %Or, on a :
    %\begin{equation*}
    %    T = (\lambda(\gamma)-\Id)^*\comp(\lambda(\gamma)-\Id) = \lambda(\gamma\inv\gamma)+\Id-(\lambda(\gamma) + \lambda(\gamma)^*) = 2-(\lambda(\gamma) + \lambda(\gamma)^*)
    %\end{equation*}
    %Donc $\Sp T = 2-\Sp(\lambda(\gamma) + \lambda(\gamma)^*)$. Comme $T$ est positif et $0\notin\Sp T$
    %on a $\Sp T\subseteq\R_+^*$ d'où $\Sp(\lambda(\gamma) + \lambda(\gamma)^*)\subseteq]-\infty, 2[$. Or l'opérateur $\lambda(\gamma)+\lambda(\gamma)^*$ est autoadjoint
    %\TODO{Ça ne marche pas !}
\end{proof}

%\begin{proposition}
%    Soit $\pi$ une représentation unitaire non-nulle du groupe localement compact $\Gamma$. La représentation triviale $\indic_\Gamma$ de $\Gamma$
%    est faiblement contenue dans $\pi$ \ssi $\forall\varepsilon>0, \forall K\in\mathfrak{K}(\Gamma), \exists\xi\in V_\pi, \norm{\xi} = 1 \land \forall g\in K, \norm{\pi(g)(\xi) - \xi}<\varepsilon$.
%\end{proposition}
%
%Lorsque cette condition est vérifiée, on dit que $\pi$ \emph{admet presque des vecteurs invariants}.
%
%\begin{lemma}
%    $\pi$ admet presque des vecteurs invariants \ssi $\forall\varepsilon>0, \forall f\in\mathscr{L}^1(\Gamma, \mu)_{1, +}\cap C_c(\Gamma), \exists\xi\in V_\pi, 
%        \norm{\xi} = 1\land \norm{f\cdot (\ev_\xi\comp\pi - \xi)}_{\mathrm{L}^1}<\varepsilon$, cette dernière expression étant bien définie 
%        en vertu du théorème \ref{strong_measurable_crit} appliquée à la fonction $g\mapsto f(g)(\pi(g)(\xi) - \xi)$ continue à support compact.
%\end{lemma}
%
%\begin{proof}
%    En supposant d'abord que $\pi$ admet presque des vecteurs invariants, soient $\varepsilon>0$ et $f\in\mathscr{L}^1(\Gamma, \mu)_{1, +}\cap C_c(\Gamma)$,
%    et notons $K := \Supp f$. L'hypothèse fournit alors $\xi\in V_\pi$ de norme $1$ tel que $\forall g\in K, \norm{\pi(g)(\xi) - \xi}<\varepsilon$. De 
%    manière immédiate, on a donc $\norm{f\cdot (\ev_\xi\comp\pi - \xi)}_{\mathrm{L}^1} = \integral{}{}{f(g)\norm{\pi(g)(\xi) - \xi}}{\mu(g)} < \varepsilon\integral{}{}{f}{\mu} = \varepsilon$.
%
%    Supposons désormais le contraire, ce qui fournit des $\varepsilon$ et $K$ tels que : 
%    \begin{equation*}
%        \forall\xi\in V_\pi, \norm{\xi} = 1\implies\exists g\in K, \norm{\pi(g)(\xi) - \xi}\geq\varepsilon
%    \end{equation*}
%\end{proof}
%
%\begin{proof}
%    Commençons par étudier la négation du critère qui nous intéresse, à savoir 
%    $\exists\varepsilon>0, \exists K\in\mathfrak{K}(\Gamma), \forall\xi\in V_\pi, \norm{\xi} = 1 \implies \exists g\in K, \norm{\pi(g)(\xi) - \xi}\geq\varepsilon$.
%\end{proof}

\newpage

\appendix

\section{Résultats usuels de théorie de la mesure}

Cette section regroupe, avec ou sans démonstration, des résultats de théorie de la mesure utilisés au cours du document.
Nous nous cantonnons ici strictement aux mesures usuelles ($\sigma$-additives), le cas moins usuel des contenu et de 
l'intégration associée étant traitée dans le corps du texte. Enfin, nous supposons connus les définitions de tribu et de mesure,
la construction de l'intégrale associée à une mesure, ainsi que les théorèmes de convergence monotone et dominée.
Si $(X, \mathcal{A})$ est un espace mesurable, on note $\mathscr{S}(X, \mathcal{A})$ l'ensemble des \emph{fonctions simples}
sur $X$, c'est à dire des fonctions mesurables $f : (X, \mathcal{A})\to(\C, \Bor(\C))$ d'image finie. 

Avant toute chose, rappelons l'inégalité de Markov et une de ses conséquences élémentaires, qui nous servira à de nombreuses reprises.
\begin{proposition}\label{markov_and_consequence}
    Soit $(X, \mathcal{A}, \mu)$ un espace mesuré et $f:X\to\overline{\R}_+$ une fonction mesurable positive.
    On a $\forall a>0, a\cdot\mu\left(f\inv([a, +\infty])\right) \le \integral{}{}{f}{\mu}$. \\
    Par conséquent, si $\integral{}{}{f}{\mu} = 0$, on a $f =_\mu 0$, où $=_\mu$ désigne la relation 
    d'égalité $\mu$-presque-partout.
\end{proposition}

\begin{proof}
    L'inégalité de Markov est une conséquence directe de la croissance de l'intégrale appliquée à 
    l'inégalité $a\cdot\indic_{\set{x\tq f(x)\le a}} \le f$. Pour sa conséquence, notons que le petit théorème de convergence monotone donne :
    \begin{equation*}
        \mu(f\inv([2^{-n}, +\infty])) \xrightarrow[n\to+\infty]{} \mu\left(\bigcup_{i\in\N} f\inv([2^{-i}, +\infty])\right) = \mu(f\inv(\set{0}\compl))
    \end{equation*}
    Mais l'hypothèse $\integral{}{}{f}{\mu} = 0$ implique, \emph{via} l'inégalité de Markov, que pour tout $n\in\N$, $\mu(f\inv([2^{-n}, +\infty]))=0$,
    ce qui entraîne $f =_\mu 0$.
\end{proof}

\subsection*{Espaces \texorpdfstring{$\mathrm{L}^p$}{L\textsuperscript{p}}}

Nous allons maintenant énoncer les résultats fondamentaux sur les espaces $\mathrm{L}^p$. Nous allons en fait considérer une variante subtile de 
la définition, qui est plus adaptée à l'analyse fonctionelle en ce que l'isométrie $\mathrm{L}^\infty \simeq (\mathrm{L}^1)^*$
sera valable \emph{sans hypothèse de $\sigma$-finitude}. Pour cela, on introduit la notion suivante.

\begin{definition}\label{loc_negligible}
    Soit $(X, \mathcal{A}, \mu)$ un espace mesuré, et soit $A\in\parts{X}$. On dit que $A$ est \emph{$\mu$-localement-négligeable} 
    si, pour tout $B\in\mathcal{A}$ \emph{de mesure finie}, $B\cap A$ est $\mu$-négligeable. \\
    Par analogie avec le cas des ensembles négligeables, on dit qu'un prédicat $P$ est valable 
    \emph{$\mu$-localement-presque-partout} si l'ensemble associé est $\mu$-localement-négligeable,
    et on note $f =_\mu^{loc} g$ si les fonctions $f$ et $g$ sont égales $\mu$-localement-presque-partout.
    Enfin, on définit le \emph{supremum $\mu$-localement-essentiel} d'une fonction mesurable $f:(X, \mathcal{A})\to(\overline{\R},\Bor(\overline{\R}))$ par 
    $\sup\ess_\mu^{loc} f := \inf\set{y\tq f\le y\text{ $\mu$-localement-presque-partout}}$.
\end{definition}

Remarquons directement que, si l'espace mesuré $(X, \mathcal{A}, \mu)$ est $\sigma$-fini, 
tout ensemble $\mu$-localement-négligeable est en fait $\mu$-négligeable. Cela garantit en particulier que, 
pour des fonctions intégrables, les relations $=_\mu$ et $=_\mu^{loc}$ coïncident, en vertu du lemme suivant.

\begin{lemma}
    Soit $f:X\to\C$ intégrable. L'ensemble $\set{x\tq f(x)\ne 0}$ est $\sigma$-fini\footnote{C'est à dire qu'il est $\sigma$-fini comme espace mesuré, 
    lorsqu'on le munit de la mesure restreinte}. 
\end{lemma}

\begin{proof}
    Chaque $\set{x\tq \abs{f(x)}\ge 2^{-n}}$ est de mesure finie par l'inégalité de Markov, et l'union de ces ensembles pour 
    $n\in\N$ recouvre $\set{x\tq f(x)\ne 0}$.
\end{proof}

\TODO{Définitions et propriétés des espaces $\mathrm{L}^p$ avec le cette notion de localement négligeable, 
sans preuve mais petite explication de ce qui change dans le cas $p=+\infty$.}

%Rappelons maintenant les définitions et propriétés usuelles des espaces $\mathrm{L}^p$, d'abord pour $p<+\infty$. Pour $(X, \mathcal{A}, \mu)$ espace mesuré et $p\in[1, +\infty[$, on 
%note $\mathscr{L}^p(X, \mu)$ l'espace vectoriel semi-normé des fonctions mesurables $f:(X, \mathcal{A})\to(\C, \Bor(\C))$ telles que $\abs{f}^p$ soit intégrable,
%muni de la seminorme $\norm{\cdot}_{\mathscr{L}^p} : f\mapsto\left(\integral{}{}{\abs{f}^p}{\mu}\right)^{\frac1p}$; on note alors $\mathrm{L}^p(X, \mu)$ l'espace 
%vectoriel normé associé, c'est à dire le quotient de $\mathscr{L}^p(X, \mu)$ par le sous-espace fermé $\set{f\tq\norm{f}_{\mathscr{L}^p}=0}$,
%muni de la norme induite notée $\norm{\cdot}_{\mathrm{L}^p}$. Comme il est usuel de le faire dans la littérature, on oublie l'application de 
%projection sur le quotient : si $f\in\mathscr{L}^p(X, \mu)$, on note aussi $f$ son image dans $\mathrm{L}^p(X, \mu)$.
%
%La proposition \ref{markov_and_consequence} assure que l'ensemble $\set{f\tq\norm{f}_{\mathscr{L}^p}=0}$ est exactement l'ensemble des fonctions 
%$\mu$-négligeables. En effet, la fonction $\abs{f}^p$ étant positive, son intégrale est nulle \ssi
%$\abs{f}^p =_\mu 0$, ce qui équivaut à $f =_\mu 0$. On peut donc voir $\mathrm{L}^p(X, \mu)$ comme le quotient de 
%$\mathscr{L}^p(X, \mu)$ par la relation d'égalité $\mu$-presque partout.
%
%Le théorème suivant rassemble les résultats dont nous aurons besoin sur les espaces $\mathrm{L}^p$.
%
%\begin{theorem}
%    Soit $(X, \mathcal{A}, \mu)$ un espace mesuré et $p\in[1, +\infty[$. Notons $q$ l'unique élément de $]1, +\infty]$
%    vérifiant $\frac1p + \frac1q = 1$.
%    \begin{enumerate}[(i)]
%        \item L'espace $\mathscr{S}(X, \mathcal{A})\cap\mathscr{L}^1(X, \mu)$ des fonctions simples intégrables 
%            est (inclu et) dense dans $\mathscr{L}^p(X, \mu)$. Son image dans $\mathrm{L}^p(X, \mu)$ est donc 
%            toujours dense.
%        \item L'espace $\mathrm{L}^p(X, \mu)$ est complet.
%        \item L'application bilinéaire $\fundef{B}{\mathcal{F}(X, \C)\times\mathcal{F}(X, \C)&\to&\mathcal{F}(X, \C)}{(f, g)&\mapsto&fg}$ 
%            se restreint et passe au quotient en une application bilinéaire continue $\widetilde{B} : \mathcal{}.
%    \end{enumerate}
%\end{theorem}

%\begin{definition}
%    Soit $(X, \mathcal{A})$ un espace mesurable. Une mesure $\mu$ sur cet espace est dite \emph{semi-finie} si tout ensemble
%    $A\in\mathcal{A}$ de mesure non-nulle contient un ensemble $B\in\mathcal{A}$ de mesure 
%    \emph{finie} non-nulle.
%\end{definition}

\subsection*{Mesures de Radon}

\begin{definition}
    Soit $X$ un espace topologique localement compact et séparé. Une \emph{mesure de Radon} sur $X$
    est une mesure $\mu$ sur l'espace mesuré $(X, \Bor(X))$ vérifiant les trois conditions suivantes :
    \begin{itemize}
        \item Pour tout $K\subseteq X$ compact, $\mu(X)<+\infty$.
        \item Régularité extérieure : 
        \begin{equation*}
            \forall A\in\Bor(X), \mu(A) = \inf \set{\mu(U) \tq U\text{ ouvert}, A\subseteq U}
        \end{equation*}
        \item Régularité intérieure pour les ouverts : 
        \begin{equation*}
            \forall U\text{ ouvert}, \mu(U) = \sup \set{\mu(K) \tq K\text{ compact}, K\subseteq U}
        \end{equation*}
    \end{itemize}
\end{definition}

La conjonction des deux conditions de régularité donne, pour tout borélien $A$ de mesure finie et tout $\varepsilon>0$,
un ouvert $U\supseteq A$ de mesure finie et un compact $K\subseteq U$ tels que $\mu(U)<\mu(A)+\varepsilon$ et 
$\mu(U)<\mu(K)+\varepsilon$. Bien sûr, on ne peut ici pas supposer $K\subseteq A$, mais cela permet 
tout de même d'obtenir des résultats intéressants.

On peut notamment appliquer le lemme d'Urysohn, pour obtenir $f:X\to[0,1]$ continue à support compact
vérifiant $\forall x\in K, f(x) = 1$ et $\Supp f\subseteq U$. On a alors, pour tout $p\in[1, +\infty[$ : 
\begin{align*}
    \integral{}{}{\abs{f-\indic_A}^p}{\mu} 
        &= \integral{A}{}{\abs{f-1}^p}{\mu} + \integral{U\setminus A}{}{\abs{f}^p}{\mu} \\
        &\le \integral{U}{}{(1-f)^p}{\mu} + \integral{U\setminus A}{}{f^p}{\mu} \\
        &= \mu(U\setminus K) + \mu(U\setminus A) \\
        &< 2\varepsilon
\end{align*}
On est donc capable d'approcher en norme $\mathscr{L}^p$ l'indicatrice de tout ensemble $A$ de mesure finie par des fonctions continues 
à support compact. Ces indicatrices engendrant un sous-espace dense de $\mathrm{L}^p(X, \mu)$ (\TODO{ref}), on vient de montrer le résultat suivant.

\begin{proposition}\label{cont_supp_compact_dense_Lp}
    Si $\mu$ est de Radon, l'espace $C_c(X)$ est dense dans $\mathrm{L}^p(X, \mu)$ pour tout $p\in[1, +\infty[$.
\end{proposition}

%Enfin, on dispose du critère suivant pour restreindre des mesures de Radon.
%
%\begin{proposition}
%    Soit $X$ un espace localement compact et séparé, $\mu$ une mesure de Radon sur $X$, et $Y\in\Bor(X)$ d'intérieur non vide. 
%    On munit $Y$ de la topologie induite, et on note $i:Y\hookrightarrow X$ l'application 
%    d'inclusion qui est un plongement d'espaces mesurables puisque $\Bor(Y) = i^*\Bor(X)$ et $i$ est injective. 
%     
%    Supposons que $Y$ est encore localement compact (c'est notamment le cas si $Y$ est ouvert ou fermé). Alors la mesure 
%    $i^*\mu:A\mapsto \mu(i(A))$, bien définie sur $\Bor(Y)$ car $i$ est un plongement mesurable, est une mesure de Radon sur $Y$.
%\end{proposition}
%
%\begin{proof}
%    
%\end{proof}

\subsection*{Un petit peu d'intégration vectorielle}

\TODO{Décrire intégration de Bochner}

%\begin{lemma}\label{strong_measurable_of_tendsto}
%    Soit $(X, \mathcal{A}, \mu)$ un espace mesuré, $E$ un espace de Banach, et $\varphi : (X, \mathcal{A})\to(E, \Bor(E))$
%    une fonction mesurable. Si de plus $\varphi$ est limite presque-partout d'une suite
%    de fonctions \emph{fortement mesurables}, alors $\varphi$ est fortement mesurable.
%\end{lemma}
%
%\begin{proof}
%    \TODO{}
%\end{proof}

En pratique, nous utiliserons le critère suivant pour montrer la mesurabilité forte.

\begin{theorem}\label{strong_measurable_crit}
    Soit $X$ un espace topologique séparé et localement compact muni d'une mesure de Radon $\mu$,
    $E$ un $\K$-espace de Banach, pour $\K\in\set{\R, \C}$. Toute fonction $f:X\to E$ continue à support compact est
    fortement mesurable.
\end{theorem}

Notons d'ailleurs que l'intégrabilité est immédiate une fois la mesurabilité forte établie, 
puisque la fonction scalaire $\norm{\cdot}\comp f$ est encore continue à support compact donc intégrable.

\begin{proof}
    Munissons le compact $K:=\Supp f$ de son unique structure d'espace uniforme, pour laquelle $f$ est automatiquement 
    uniformément continue en vertu du théorème de Heine. La (pré)compacité de l'espace métrique $f(K)$ fournit, 
    pour chaque $n\in\N$, un ensemble fini $S_n\subseteq K$ tel que $f(K)\subseteq\bigcup_{x\in S_n}B(f(x), 2^{-n})$.

    Pour $n$ fixé, choisissons une bijection $\alpha_n : \card{S_n} \to S_n$, où l'on identifie $\card{S_n}$ à l'ensemble 
    $\left\llbracket0, \card{S_n}-1\right\rrbracket$. On définit alors $\chi_n:K\to S_n$ par
    \begin{equation*}
        \forall x\in K, \chi_n(x)=\alpha_n\left(\min\set{k\in\card{S_n}\tq f(x)\in B(f(\alpha_n(k)), 2^{-n})}\right)
    \end{equation*}
    Autrement dit, $\chi_n$ est l'unique fonction vérifiant 
    \begin{equation*}
        \forall k\in\card{S_n}, \chi_n\inv(\alpha_n(k)) = f\inv\left(B(f(\alpha_n(k)), 2^{-n})\setminus\left(\bigcup_{i<k}B(f(\alpha_n(i)), 2^{-n})\right)\right)
    \end{equation*}
    Cela montre que les fibres de $\chi_n$ sont boréliennes, donc que $\chi_n$ est mesurable, et que :
    \begin{align*}
        \forall x\in S_n, \forall y\in\chi_n\inv(x), \norm{f(x)-f(y)}<2^{-n}
    \end{align*}
    On peut donc définir $\psi_n : X\to E$ par ${\psi_n}_{|K\compl}=0$ et ${\psi_n}_{|K}=f\comp\chi_n$, qui est bien une fonction 
    mesurable (par mesurabilité de l'ensemble $K$, et des fonctions $f$ et $\chi_n$) et d'image finie 
    inclue dans $\set{0}\cup f(S_n)$. On a enfin 
    $\forall n\in\N, \norm{\psi_n - f}_\infty \le 2^{-n}$, ce qui assure que la suite $\psi$ converge vers $f$ uniformément, 
    et donc en particulier $\mu$-presque-partout. Cela conclut.
\end{proof}

\begin{remark}
    On aurait aussi pu remarquer que l'espace métrique $f(X) = \set{0}\cup f(K)$ est compact donc séparable,
    et conclure en appliquant le théorème de mesurabilité de Pettis, qui assure qu'une fonction à valeurs dans un Banach 
    qui est mesurable et d'image séparable est automatiquement fortement mesurable.
\end{remark}

\section{Intégration sur les groupes localement compacts}

On fixe un groupe topologique $G$ séparé et localement compact.

\subsection*{Mesures de Haar}

Une \emph{mesure de Haar à gauche (resp. à droite)} sur $G$
est une mesure de Radon $\mu$ vérifiant $\forall g\in G, {\ell_g}_*\mu = \mu$ (resp. ${r_g}_*\mu = \mu$).

Le théorème fondamental concernant ces mesures, que nous utilisons tout au long de ce document, est le suivant :
\begin{theorem}\label{theorem_Haar}
    Tout groupe topologique $G$ séparé et localement compact admet une mesure de Haar à gauche, unique à multiplication par un réel strictement 
    positif près. 
    
    Si $\mu$ est une telle mesure, tout ouvert $U$ non-vide de $G$ est de mesure strictement positive,
    et $\mu(G)<+\infty$ \ssi G est compact.

    Enfin, si $G$ est compact, les mesures de Haar à gauche sont exactement les mesures de Haar à droite.
\end{theorem}

\subsection*{Caractère modulaire}

Soit $\mu$ une mesure de Haar à gauche sur $G$. On vérifie aisément que, pour tout $g\in G$, la
mesure de Radon ${r_g}_*\mu$ est encore de Haar à gauche, ce qui fournit un un réel positif $\Delta(g)$ tel que 
${r_g}_*\mu = \Delta(g)\mu$. La fonction $\Delta$ ainsi définie est le \emph{caractère modulaire} de $G$,
et ses propriétés sont résumées par la propriété suivante :

\begin{proposition}\label{modular_character}
    Soit $G$ un groupe topologique séparé et localement compact, et $\mu$ une mesure de Haar à gauche.
    \begin{enumerate}[(i)]
        \item\label{modular_character/def} Par définition, on a $\forall g\in G, {r_g}_*\mu = \Delta(g)\mu$.
        \item\label{modular_character/indep} $\Delta$ est indépendant du choix de la mesure de Haar $\mu$.
        \item\label{modular_character/continuous_group_hom} $\Delta:G\to\R_+^*$ est un morphisme de groupes topologiques.
        \item\label{modular_character/right_haar} $\frac{1}{\Delta}\mu$ est de Haar à droite.
        \item\label{modular_character/inv_change_of_variable} $\invop_*\mu = \frac{1}{\Delta}\mu$
    \end{enumerate}
\end{proposition}

Nous utilisons la propriété $(\ref{modular_character/inv_change_of_variable})$ à de nombreuses reprises
pour effectuer des changements de variables de la forme $g\mapsto g\inv$. C'est notamment fondamental pour développer 
la théorie de la convolution sur un groupe topologique localement compact.

\subsection*{Convolution}

\section{Quelques résultats sur les espaces uniformes}
\subsection*{Sur la convergence uniforme des translatées d'une fonction}

Soient $X$ un ensemble et $Y$ un espace uniforme. On note $\mathcal{F}_u(X, Y)$ l'ensemble des 
fonctions de $X$ dans $Y$ muni de la structure uniforme de la convergence uniforme, dont le filtre des entourages 
$\mathcal{U}(\mathcal{F}_u(X, Y))$ admet pour base les ensembles de la forme 
$\mathbf{S}_X(V) := \set{(f, g)\tq\forall x\in X, (f(x),g(x))\in V}$ lorsque $V$ parcourt (une base de) $\mathcal{U}Y$.

Le but de cette section est de démontrer le théorème suivante.
\begin{theorem}\label{uniform_continuous_iff_postcomp}
    Soient $G$ un groupe topologique, $X$ un espace uniforme, et $f:G\to X$. Les assertions suivantes sont équivalentes.
    \begin{enumerate}[(i)]
        \item $f:G_d\to X$ est uniformément continue \emph{pour la structure uniforme droite}.
        \item L'application $\fundef{\ev_f\comp\lambda}{G_s&\to&\mathcal{F}_u(G,X)}{g&\mapsto&\lambda(g)(f)}$ est uniformément continue
            \emph{pour la structure uniforme gauche}.
        \item L'application $\ev_f\comp\lambda:G\to\mathcal{F}_u(G,X)$ est continue.
    \end{enumerate}
\end{theorem}

\begin{lemma}
    Soit $G$ un groupe topologique. L'application $\ell : G_d\mapsto\mathcal{F}_u(G, G_d)$
    \footnote{On écrit $\mathcal{F}_u(G, G_d)$ au lieu de $\mathcal{F}_u(G_d, G_d)$ car l'espace
    ne dépend pas de la structure uniforme du premier facteur (en fait, on devrait même écrire l'ensemble $G$ 
    à la place du groupe topologique $G$).} est un plongement d'espaces uniformes.
\end{lemma}

\begin{proof}
    Il est évident que $\ell$ est injective, puisque $\ell(g)(1)=g$. Il s'agit
    donc de montrer que $\mathcal{U}G_d = (\ell\times\ell)^*\mathcal{U}(\mathcal{F}_u(G, G_d))$,
    et on va pour cela remarquer que la famille $\mathcal{B}:=\set{\divop_d\inv(V)}_{V\in\nhds_1^G}$ est une base commune pour ces deux filtres.

    L'égalité $\mathcal{U}G_d=(\divop_d)^*\nhds_1^G$ garantit que $\mathcal{B}$ est effectivement une base de $\mathcal{U}G_d$,
    et donc que $\mathcal{C}:=\set{(\ell\times\ell)\inv\left(\mathbf{S}_G(\divop_d\inv(V))\right)}_{V\in\nhds_1^G}$ est une base de $(\ell\times\ell)^*\mathcal{U}(\mathcal{F}_u(G, G_d))$.
    Or, pour $V\in\nhds_1^G$ et $g, h\in G$, on a :
    \begin{align*}
        (g, h)\in(\ell\times\ell)\inv\left(\mathbf{S}_G(\divop_d\inv(V))\right) 
            &\iff (\ell_g,\ell_h)\in \mathbf{S}_G(\divop_d\inv(V)) \\
            &\iff \forall x\in G, (\ell_g(x),\ell_h(x))\in\divop_d\inv(V) \\
            &\iff \forall x\in G, g x x\inv h\inv \in V \\
            &\iff g h\inv \in V \\
            &\iff (g, h) \in\divop_d\inv(V)
    \end{align*}
    Ceci montre que $\mathcal{B}=\mathcal{C}$ est bien une base de $(\ell\times\ell)^*\mathcal{U}(\mathcal{F}_u(G, G_d))$, ce qui conclut.
\end{proof}

Remarquons que l'application $\ev_f\comp\lambda$ peut encore s'écrire comme la composition suivante :
\begin{center}
    \begin{tikzcd}
        G_s \arrow[r, , "\invop", "\sim"' inner sep=.4mm] &G_d \arrow[r, hook, "\ell"] &\mathcal{F}_u(G, G_d) \arrow[r, , "(f\comp\blank)"] &\mathcal{F}_u(G, X)
    \end{tikzcd}
\end{center}
Or, $\invop$ et $\ell$ étant respectivement un isomorphisme et un plongement d'espaces uniformes (et donc d'espaces topologiques), 
la continuité (resp. la continuité uniforme) de $\ev_f\comp\lambda$ équivaut à la continuité (resp. la continuité uniforme) de 
$(f\comp\blank):\mathcal{F}_u(G, G_d)\to\mathcal{F}_u(G, X)$. Le théorème \ref{uniform_continuous_iff_postcomp} découle donc 
du lemme suivant. 

\begin{lemma}
    Soient $X$, $Y$ deux espaces uniformes et $f:X\to Y$. Les propositions suivantes sont équivalentes.
    \begin{enumerate}[(i)]
        \item $f$ est uniformément continue
        \item $(f\comp\blank) : \mathcal{F}_u(X, X)\to\mathcal{F}_u(X, Y)$ est uniformément continue
        \item $(f\comp\blank)$ est continue
        \item $(f\comp\blank)$ est continue en $\id_X$
    \end{enumerate}
\end{lemma}

\begin{proof}
    Montrons \framebox{$(i)\implies(ii)$}. On suppose donc $f$ uniformément continue, et on se donne $V\in\mathcal{U}Y$. On a alors :
    \begin{align*}
        ((f\comp\blank)\times(f\comp\blank))\inv(\mathbf{S}_X(V)) 
            &= \set{(\varphi_1, \varphi_2)\tq\forall x, (f(\varphi_1(x)), f(\varphi_2(x)))\in V} \\
            &= \mathbf{S}_X((f\times f)\inv(V))
    \end{align*}
    L'uniforme continuité de $f$ garantit que $(f\times f)\inv(V)\in\mathcal{U}X$, ce qui entraîne :
    \begin{equation*}
        \mathbf{S}_X((f\times f)\inv(V))\in\mathcal{U}(\mathcal{F}_u(X, X))
    \end{equation*}
    Ceci montre que $(f\comp\blank)$ est uniformément continue.

    Les implications \framebox{$(ii)\implies(iii)$} et \framebox{$(iii)\implies(iv)$} étant claires, montrons
    \framebox{$(iv)\implies(i)$}. On suppose donc $(f\comp\blank)$ continue en $\id_X$, et on se donne 
    $V\in\mathcal{U}Y$. La définition de la convergence uniforme assure que 
    $\mathbf{T}:=\set{h : X\to Y\tq \forall x, (h(x), f(x))\in V}$ est un voisinage de $f = (f\comp\blank)(\id_X)$ dans 
    $\mathcal{F}_u(X, Y)$. Par continuité, $(f\comp\blank)\inv(\mathbf{T})$ est donc un voisinage de $\id_X$ dans $\mathcal{F}_u(X, X)$,
    ce qui signifie qu'il contient un ensemble de la forme $\mathbf{U}:=\set{\varphi : X\to X\tq \forall x, (\varphi(x), x)\in W}$ pour 
    un $W\in\mathcal{U}X$. 
    
    Or, pour $(x_1, x_2)\in W$, il est clair qu'il existe une fonction $\varphi\in\mathbf{U}$ telle que $\varphi(x_2) = x_1$
    \footnote{On peut par exemple poser $\varphi(x_2)=x_1$ et $\varphi(x) = x$ pour $x\ne x_2$.}.
    Mais alors $f\comp\varphi\in\mathbf{T}$, d'où, en évaluant en $x_2$, $(f(x_1),f(x_2))\in V$.
    On a donc montré que $W\subseteq (f\times f)\inv(W)$, ce qui garantit l'uniforme continuité de $f$.

    Ceci conclut la preuve du lemme, ainsi que celle du théorème \ref{uniform_continuous_iff_postcomp}.
\end{proof}

\end{document}