\documentclass[a4paper,12pt]{article}
\usepackage[utf8]{inputenc}
\usepackage[french]{babel}
\usepackage{amssymb}
\usepackage{amsmath}
\usepackage{bbm}
\usepackage{amsthm}
\usepackage{a4wide}
\usepackage{mathrsfs}
\usepackage{stmaryrd}
\usepackage{mathtools}
\usepackage{graphicx}
\usepackage{hyperref}
\usepackage{faktor}
\usepackage{enumerate}
\usepackage[T1]{fontenc}

\newtheorem{theorem}{Théorème}[section]
\newtheorem{prop}[theorem]{Proposition}
\newtheorem{definition}[theorem]{Définition}
\newtheorem{corollary}[theorem]{Corollaire}
\newtheorem{lem}[theorem]{Lemme}

\newcommand{\R}{\mathbb{R}}
\newcommand{\N}{\mathbb{N}}
\newcommand{\Q}{\mathbb{Q}}
\newcommand{\Z}{\mathbb{Z}}
\newcommand{\C}{\mathbb{C}}
\newcommand{\K}{\mathbb{K}}
\newcommand{\F}{\mathcal{F}}
\newcommand{\G}{\mathcal{G}}
\newcommand{\U}{\mathcal{U}}
\newcommand{\Bor}{\mathcal{B}}
\newcommand{\norm}[1]{\left\Vert #1\right\Vert}
\newcommand{\abs}[1]{\left\vert#1\right\vert}
\newcommand{\ket}[1]{\left\langle #1 \right\rangle}
\newcommand{\halfilon}{{\frac\varepsilon2}}
\newcommand{\set}[1]{\left\{ #1 \right\}}
\newcommand{\indic}{\mathbbm{1}}
\newcommand{\integral}[2]{\int #1~\mathrm{d}#2}
\newcommand\fundef[3]{#1: \left\{\begin{array}{ccc}#2\\#3\end{array}\right.}
\newcommand\funlam[2]{\left\{\begin{array}{ccc}#1\\#2\end{array}\right.}
\newcommand{\tq}{\;\middle|\;}
\newcommand{\interior}[1]{\mathring{#1}}
\newcommand{\closure}[1]{\overline{#1}}
\newcommand{\transpose}[1]{\prescript{t}{}{#1}{}{}}
\newcommand{\inv}{^{-1}}
\newcommand{\infi}{\bigwedge}
\newcommand{\supr}{\bigvee}
\newcommand{\comp}{\circ}
\newcommand{\nhds}{\mathcal{N}}
\renewcommand{\implies}{\Rightarrow}
\renewcommand{\iff}{\Leftrightarrow}
\newcommand{\blank}{{-}}
\newcommand{\invop}{\mathrm{inv}}

\DeclareMathOperator{\sgn}{sgn}
\DeclareMathOperator{\card}{Card}
\DeclareMathOperator{\Id}{Id}
\DeclareMathOperator{\Mat}{Mat}
\DeclareMathOperator{\Vect}{Vect}
\DeclareMathOperator{\Ima}{Im}
\DeclareMathOperator{\solset}{Sol}
\DeclareMathOperator{\Sp}{Sp}

\begin{document}

\begin{titlepage}
\title{Groupes Moyennables}
\author{Anatole \textsc{Dedecker}}
\maketitle
\thispagestyle{empty}
\end{titlepage}

\tableofcontents
\thispagestyle{empty}

\clearpage

\pagenumbering{arabic}

\section*{Conventions et remarques préliminaires}

Dans ce mémoire, nous utiliserons le terme de \textit{mesure} pour désigner une mesure \textbf{finiment}-additive sur un 
\textit{espace mesurable}, c'est à dire un couple $(X, \mathcal{A})$ où $\mathcal{A}$ est une $\sigma$-algèbre. 
Lorsqu'une telle mesure est de plus $\sigma$-additive (ce qui est souvent inclus dans la définition de \og{}mesure\fg{}),
nous dirons qu'il s'agit d'une \textit{$\sigma$-mesure}. 

Nous admettrons qu'il est possible de définir une théorie de l'intégration pour toute mesure, les théorèmes de convergence
monotone et dominée n'étant bien sûr valables que dans le cas des $\sigma$-mesures. 

Si $\varphi:(X,\mathcal{A})\to(Y,\mathcal{B})$ est une application mesurable et $m$ est une mesure sur $(X, \mathcal{A})$, 
nous noterons $\varphi_*m$ la mesure image sur $(Y, \mathcal{B})$. C'est une $\sigma$-mesure si $m$ en est une, et intégrer 
une fonction $f$ selon $\varphi_*m$ revient exactement à intégrer $f\comp\varphi$ selon $m$. Enfin, c'est une construction
fonctorielle, au sens où l'on a $(\psi\comp\varphi)_*m = \psi_*\varphi_*m$ pour toute application mesurable 
$\psi : (Y, \mathcal{B})\to(Z, \mathcal{C})$. 

TODO mesure à densité

\paragraph{}
Si $G$ est un groupe et $g\in G$, nous noterons $\invop$ l'inversion et $\ell_g$, $r_g$ les applications de tranlations:
\begin{equation*}
    \fundef{\invop = (\blank)\inv}{G&\to& G}{x&\mapsto& x\inv}\text{;}\quad\fundef{\ell_g=(g\blank)}{G&\to& G}{x&\mapsto& gx}\text{;}\quad\fundef{r_g=(\blank g)}{G&\to& G}{x&\mapsto& xg}
\end{equation*}

Si le groupe $G$ est muni d'une $\sigma$-algèbre stable par les tranlations droite et gauche, ce qui sera notamment le cas
si $G$ est un groupe topologique muni de sa tribu borélienne $\Bor(G)$, nous dirons qu'une mesure $m$ sur $G$ est \textit{invariante
(par translations) à gauche (resp. à droite)} si $\forall g\in G, {\ell_g}_*m = m$ (resp. ${r_g}_*m = m$). Si les deux conditions
sont vérifiées, nous parlerons simplement de mesure \textit{invariante par translations}.

Une \textit{mesure de Haar à gauche (resp. à droite)} sur un groupe topologique $G$ séparé et localement compact est une mesure de
Radon sur $G$ invariante par tranlations à gauche (resp. à droite).

TODO rappeler théorème

\section{Premières définitions}

Fixons $\Gamma$ un groupe topologique séparé et localement compact. La notion qui va nous intéresser dans ces notes est la suivante.

\begin{definition}
    On dit qu'un groupe topologique séparé et localement compact $\Gamma$ est \emph{moyennable} s'il admet une 
    mesure de probabilité borélienne et invariante par translations.
\end{definition}

Notons tout de suite que l'invariance bilatère n'impose pas de restriction suplémentaire. 
\begin{prop}
    Supposons qu'il existe une mesure de probabilité borélienne $m$ sur $\Gamma$, invariante par translations à gauche.
    Alors $\Gamma$ est un groupe moyennable.
\end{prop} 

\begin{proof}
    Il s'agit donc de construire une \textbf{autre} mesure $n$ sur $\Gamma$ qui soit cette-fois invariante des deux côtés.
    Posons d'abord, pour $A\in\Bor(G)$, $\fundef{f_A}{\Gamma&\to& \Gamma}{g&\mapsto& m(Ag\inv)}$. Chaque $f_A$ est bornée par $1 = m(\Gamma)$, 
    donc intégrable pour la mesure de probabilité $\invop_*m$. Posons alors $n(A) := \integral{f_A}{(\invop_*m)}$. \\
    Notons que, pour $A$ et $B$ boréliens disjoints, ainsi que $g, x\in\Gamma$, on a :
    \begin{gather*}
        f_\Gamma = \indic \\
        f_{A\cup B} = f_A + f_B \\
        f_{g\inv A}(x) = m(g\inv Ax\inv) = ({\ell_g}_*m)(Ax\inv) = m(Ax\inv) = f_A(x) \\
        f_{Ag\inv}(x) = m(Ag\inv x\inv) = f_A(xg) = (f_A \comp r_g)(x)
    \end{gather*}
    En intégrant ces relations, on obtient bien :
    \begin{gather*}
        n(\Gamma) = (\invop_*m)(\Gamma) = 1 \\
        n(A\cup B) = n(A) + n(B) \\
        ({\ell_g}_*n)(A) = n(g\inv A) = n(A) \\
        ({r_g}_*n)(A) = \integral{f_A\comp r_g\comp\invop}{m} = \integral{f_A\comp\invop\comp \ell_g}{m} = n(A)
    \end{gather*}
    Ce qui conlut.
\end{proof}

Les premiers exemples de groupes moyennables sont les groupes compacts (et en particulier les groupes finis discrets) séparés.
En effet, la mesure de Haar normalisée d'un tel groupe est une mesure de probabilité invariante par tranlation. Les phénomènes
plus intéressants vont donc se produire dans le cas non-compact, et il convient de noter que ces groupes ne peuvent pas avoir
de $\sigma$-mesure de probabilité invariante par translations (TODO ref). C'est donc l'affaibilissement de la 
$\sigma$-additivité en additivité finie qui engendre la complexité. \\

Comme constaté dans la preuve ci-dessus, [TODO forme linéaire]

On peut maintenant donner notre premier exemple de groupe non-compact moyennable.

\begin{theorem}\label{Z_amenable}
    $\Z$ est moyennable.
\end{theorem}

On va pour cela utiliser la variation suivante sur le théorème de Hahn-Banach.

TODO

\begin{proof}[Démonstration du théorème \ref{Z_amenable}]
    Considérons le sous-espace vectoriel $C$ de $\mathrm{L}^\infty(\Z) = \ell^\infty(\Z)$ formé 
    des suites $u:\Z\to\R$ convergentes en $+\infty$. Posons alors 
    $\fundef{\varphi}{C&\to&\R}{u&\mapsto&\lim_{n\to+\infty}u_n}$.

    Il est clair que $\varphi$ est linéaire. De plus, pour $u\in C$, on a $\forall n, \abs{u_n}\le\norm{u_n}_\infty$
    donc en passant à la limite $\abs{\varphi(u)}\le\norm{u}_\infty$, donc $\varphi$ est continue. $\varphi$ est enfin clairement positive.
    Par le résultat ci-dessus (TODO ref) on obtient donc $\widetilde{\varphi}:$

\end{proof}

\end{document}